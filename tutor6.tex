
\chapter{BOOLEAN EXPRESSIONS}

31st Aug 1988

\section{Introduction}

In Chapter 5 of this series, we took a look at control constructs, and developed parsing  routines  to  translate  them  into object code. We  ended up with a  nice, relatively  rich set of constructs.

As we left  the  parser, though, there  was one big hole in our capabilities:  we  did  not  address  the  issue  of  the  branch condition. To fill the void, I  introduced to you a dummy parse routine called Condition, which only served as a place-keeper for the real thing.

One of the things we'll do in this session is  to  plug that hole by expanding Condition into a true parser/translator.

\section{The Plan}

We're going to  approach  this installment a bit differently than any of the others. In those other installments, we started out immediately with experiments  using the Pascal compiler, building up the parsers from  very  rudimentary  beginnings to their final forms, without spending much time in planning  beforehand. That's called coding without specs, and it's usually frowned  upon. We could get away with it before because the rules of arithmetic are pretty well established ... we  know what a '+' sign is supposed to mean without having to discuss it at length. The same is true for branches and  loops. But  the  ways  in  which programming languages  implement  logic  vary quite a bit  from  language  to language. So before we begin serious coding, we'd  better first make up our minds what it is we want. And the way to do  that is at the level of the BNF syntax rules (the GRAMMAR).

\section{The Grammar}

For some time  now, we've been implementing BNF syntax equations for arithmetic expressions, without  ever  actually  writing them down all in one place. It's time that we did so. They are:

\begin{verbatim}
  <expression> ::= <unary op> <term> [<addop> <term>]*
  <term>       ::= <factor> [<mulop> factor]*
  <factor>     ::= <integer> | <variable> | ( <expression> )
\end{verbatim}

(Remember, the nice thing about  this grammar is that it enforces the operator precedence hierarchy  that  we  normally  expect for algebra.)

Actually, while we're on the subject, I'd  like  to  amend  this grammar a bit right now. The  way we've handled the unary minus is  a  bit  awkward. I've found that it's better  to  write  the grammar this way:

{\small
\begin{verbatim}
  <expression>    ::= <term> [<addop> <term>]*
  <term>          ::= <signed factor> [<mulop> factor]*
  <signed factor> ::= [<addop>] <factor>
  <factor>        ::= <integer> | <variable> | (<expression>)
\end{verbatim} }

This puts the job of handling the unary minus onto  Factor, which is where it really belongs.

This  doesn't  mean  that  you  have  to  go  back and recode the programs you've already written, although you're free to do so if you like. But I will be using the new syntax from now on.

Now, it probably won't come as  a  shock  to you to learn that we can define an analogous grammar for Boolean algebra. A typical set or rules is:

{\small
\begin{verbatim}
 <b-expression>::= <b-term> [<orop> <b-term>]*
 <b-term>      ::= <not-factor> [AND <not-factor>]*
 <not-factor>  ::= [NOT] <b-factor>
 <b-factor>    ::= <b-literal> | <b-variable> | (<b-expression>)
\end{verbatim} }

Notice that in this  grammar, the  operator  AND is analogous to '*', and  OR  (and exclusive OR) to '+'. The  NOT  operator  is analogous to a unary  minus. This  hierarchy is not absolutely standard ... some  languages, notably  Ada, treat  all logical operators  as  having  the same precedence level ... but it seems natural.

Notice also the slight difference between the way the NOT and the unary  minus  are  handled. In  algebra, the unary  minus  is considered to go with the whole term, and so  never  appears  but once in a given term. So an expression like

\begin{verbatim}
                    a * -b
or worse yet,
                    a - -b
\end{verbatim}

is not allowed. In Boolean algebra, though, the expression

\begin{verbatim}
                  a AND NOT b
\end{verbatim}

makes perfect sense, and the syntax shown allows for that.

\section{Relops}

OK, assuming that you're willing to accept the grammar I've shown here, we  now  have syntax rules for both arithmetic and Boolean algebra. The  sticky part comes in when we have to combine the two. Why do we have to do that?  Well, the whole subject came up because of the  need  to  process  the  ``predicates'' (conditions) associated with control statements such as the IF. The predicate is required to have a Boolean value; that is, it must evaluate to either TRUE or FALSE. The branch is  then  taken  or  not taken, depending  on  that  value. What we expect to see  going  on  in procedure  Condition, then, is  the  evaluation  of  a  Boolean expression.

But there's more to it than that. A pure Boolean  expression can indeed be the predicate of a control statement ... things like

\begin{verbatim}
          IF a AND NOT b THEN ....
\end{verbatim}

But more often, we see Boolean algebra show up in such things as

\begin{verbatim}
     IF (x >= 0) and (x <= 100) THEN ...
\end{verbatim}

Here, the  two  terms in parens are Boolean expressions, but the individual terms being compared:  x, 0, and 100, are NUMERIC in nature. The RELATIONAL OPERATORS {\tt >=} and {\tt <=} are the  catalysts by which the  Boolean  and  the  arithmetic  ingredients  get merged together.

Now, in the example above, the terms  being  compared  are  just that:  terms. However, in  general  each  side  can be a math expression. So we can define a RELATION to be:

\begin{verbatim}
     <relation> ::= <expression> <relop> <expression>,
\end{verbatim}

where  the  expressions  we're  talking  about here are  the  old numeric type, and the relops are any of the usual symbols

\begin{verbatim}
             =, <> (or !=), <, >, <=, and >=
\end{verbatim}

If you think about it a  bit, you'll agree that, since this kind of predicate has a single Boolean value, TRUE or  FALSE, as  its result, it is  really  just  another  kind  of factor. So we can expand the definition of a Boolean factor above to read:

\begin{verbatim}
    <b-factor> ::=    <b-literal>
                    | <b-variable>
                    | (<b-expression>)
                    | <relation>
\end{verbatim}

THAT's the connection!  The relops and the  relation  they define serve to wed the two kinds of algebra. It  is  worth noting that this implies a hierarchy  where  the  arithmetic expression has a {\bfseries higher} precedence that  a  Boolean factor, and therefore than all the  Boolean operators. If you write out the precedence levels for all the operators, you arrive at the following list:

\begin{verbatim}
Level	Syntax Element	Operator
0	Factor	Literal, variable
1	Signed factor	Unary minus
2	Term	*,  /
3	Expression	+, -
4	b-factor	Literal, variable, relop
5	Not-factor	NOT
6	b-term	AND
7	b-expression	OR, XOR
\end{verbatim}

If  we're willing to accept that  many  precedence  levels, this grammar seems reasonable. Unfortunately, it  won't  work!   The grammar may be great in theory, but  it's  no good at all in the practice of a top-down parser. To see the problem, consider the code fragment:

\begin{verbatim}
     IF ((((((A + B + C) < 0 ) AND ....
\end{verbatim}

When the parser is parsing this code, it knows after it  sees the IF token that a Boolean expression is supposed to be next. So it can set up to begin evaluating such an expression. But the first expression in the example is an {\bfseries arithmetic} expression, A + B + C. What's worse, at the point that the parser has read this  much of the input line:

\begin{verbatim}
     IF ((((((A   ,
\end{verbatim}

it  still has no way of knowing which  kind  of  expression  it's dealing  with. That won't do, because  we  must  have  different recognizers  for the two cases. The  situation  can  be  handled without  changing  any  of  our  definitions, but only  if  we're willing to accept an arbitrary amount of backtracking to work our way out of bad guesses. No compiler  writer  in  his  right mind would agree to that.

What's going  on  here  is  that  the  beauty and elegance of BNF grammar  has  met  face  to  face with the realities of  compiler technology.

To  deal  with  this situation, compiler writers have had to make compromises  so  that  a  single  parser can handle  the  grammar without backtracking.

\section{Fixing The Grammar}

The  problem  that  we've  encountered  comes   up   because  our definitions of both arithmetic and Boolean factors permit the use of   parenthesized  expressions. Since  the  definitions   are recursive, we  can  end  up  with  any  number   of   levels  of parentheses, and the  parser  can't know which kind of expression it's dealing with.

The  solution is simple, although it  ends  up  causing  profound changes to our  grammar. We  can only allow parentheses in one kind  of factor. The way to do  that  varies  considerably  from language  to  language. This is one  place  where  there  is  NO agreement or convention to help us.

When Niklaus Wirth designed Pascal, the desire was  to  limit the number of levels of precedence (fewer parse routines, after all). So the OR  and  exclusive  OR  operators are treated just like an {\tt Addop}  and  processed   at   the  level  of  a  math  expression. Similarly, the AND is  treated  like  a  Mulop and processed with Term. The precedence levels are

\begin{verbatim}
Level	Syntax Element	Operator
0	Factor	Literal, variable
1	Signed factor	Unary minux, NOT
2	Term	*, /, AND
3	Expression	+, -, OR
\end{verbatim}

Notice that there is only ONE set of syntax  rules, applying  to both  kinds  of  operators. According to this  grammar, then, expressions like

\begin{verbatim}
     x + (y AND NOT z) DIV 3
\end{verbatim}

are perfectly legal. And, in  fact, they {\bfseries are} ... as far as the parser  is  concerned. Pascal  doesn't  allow  the  mixing  of arithmetic and Boolean variables, and things like this are caught at the {\bfseries semantic} level, when it comes time to  generate  code  for them, rather than at the syntax level.

The authors of C took  a  diametrically  opposite  approach: they treat the operators as  different, and  have something much more akin  to our seven levels of precedence. In fact, in C there are no fewer than 17 levels!  That's because C also has the operators `=', `+=' and its kin, `<<', `>>', `++', `--', etc. Ironically, although in C the  arithmetic  and  Boolean operators are treated separately, the variables are {\bfseries not}  ... there  are no Boolean or logical variables in  C, so  a  Boolean  test can be made on any integer value.

We'll do something that's  sort  of  in-between. I'm tempted to stick  mostly  with  the Pascal approach, since  that  seems  the simplest from an implementation point  of view, but it results in some funnies that I never liked very much, such as the fact that, in the expression

\begin{verbatim}
     IF (c >= 'A') and (c <= 'Z') then ...
\end{verbatim}

the  parens  above  are {\bfseries required}. I never understood why before, and  neither my compiler nor any human  ever  explained  it  very well, either. But now, we  can  all see that the 'and' operator, having the precedence of a multiply, has a higher  one  than  the relational operators, so without  the  parens  the  expression is equivalent to

\begin{verbatim}
     IF c >= ('A' and c) <= 'Z' then
\end{verbatim}

which doesn't make sense.

In  any  case, I've  elected  to  separate  the  operators  into different levels, although not as many as in C.

\begin{verbatim}
 <b-expression> ::= <b-term> [<orop> <b-term>]*
 <b-term>       ::= <not-factor> [AND <not-factor>]*
 <not-factor>   ::= [NOT] <b-factor>
 <b-factor>     ::= <b-literal> | <b-variable> | <relation>
 <relation>     ::= | <expression> [<relop> <expression]
 <expression>   ::= <term> [<addop> <term>]*
 <term>         ::= <signed factor> [<mulop> factor]*
 <signed factor>::= [<addop>] <factor>
 <factor>       ::= <integer> | <variable> | (<b-expression>)
\end{verbatim}

This grammar  results  in  the  same  set  of seven levels that I showed earlier. Really, it's almost the same grammar ... I just removed the option of parenthesized b-expressions  as  a possible b-factor, and added the relation as a legal form of b-factor.

There is one subtle but crucial difference, which  is  what makes the  whole  thing  work. Notice  the  square brackets  in  the definition  of a relation. This means that  the relop and the second expression are {\bfseries optional}.

A strange consequence of this grammar (and one shared  by  C)  is that {\bfseries every} expression  is  potentially a Boolean expression. The parser will always be looking  for a Boolean expression, but will ``settle'' for an arithmetic one. To be honest, that's  going  to slow down the parser, because it has to wade through  more layers of procedure calls. That's  one reason why Pascal compilers tend to compile faster than C compilers. If it's raw speed  you want, stick with the Pascal syntax.

\section{The Parser}

Now that we've gotten through the decision-making process, we can press on with development of a parser. You've done this  with me several times now, so you know  the  drill: we begin with a fresh copy of the cradle, and begin  adding  procedures one by one. So let's do it.

We begin, as we did in the arithmetic case, by dealing  only with Boolean literals rather than variables. This gives us a new kind of input token, so we're also going to need a new recognizer, and a  new procedure to read instances of that  token  type. Let's start by defining the two new procedures:

\begin{verbatim}
{------------------------------------------------------}
{ Recognize a Boolean Literal }

function IsBoolean(c: char): Boolean;
begin
   IsBoolean := UpCase(c) in ['T', 'F'];
end;

{------------------------------------------------------}
{ Get a Boolean Literal }

function GetBoolean: Boolean;
var c: char;
begin
   if not IsBoolean(Look) then Expected('Boolean Literal');
   GetBoolean := UpCase(Look) = 'T';
   GetChar;
end;
{------------------------------------------------------}
\end{verbatim}

Type  these routines into your program. You  can  test  them  by
adding into the main program the print statement

\begin{verbatim}
   WriteLn(GetBoolean);
\end{verbatim}

OK, compile the program and test it. As  usual, it's  not very impressive so far, but it soon will be.

Now, when we were dealing with numeric data we had to  arrange to generate code to load the values into D0. We need to do the same for Boolean data. The  usual way to encode Boolean variables is to let 0 stand for FALSE, and  some  other value for TRUE. Many languages, such as C, use an  integer  1  to represent it. But I prefer FFFF hex  (or -1), because  a bitwise NOT also becomes a Boolean  NOT. So now we need to emit the right assembler code to load  those  values. The  first cut at the Boolean  expression parser ({\tt BoolExpression}, of course) is:

\begin{verbatim}
{---------------------------------------------------------------}
{ Parse and Translate a Boolean Expression }

procedure BoolExpression;
begin
   if not IsBoolean(Look) then Expected('Boolean Literal');
   if GetBoolean then
      EmitLn('MOVE #-1,D0')
   else
      EmitLn('CLR D0');
end;
{---------------------------------------------------------------}
\end{verbatim}

Add  this procedure to your parser, and call  it  from  the  main program (replacing the  print  statement you had just put there). As you  can  see, we  still don't have much of a parser, but the output code is starting to look more realistic.

Next, of course, we have to expand the definition  of  a  Boolean expression. We already have the BNF rule:

\begin{verbatim}
 <b-expression> ::= <b-term> [<orop> <b-term>]*
\end{verbatim}

I prefer the Pascal versions of the ``orops'', OR  and  XOR. But since we are keeping to single-character tokens here, I'll encode those with `|' and  `~'. The  next  version of {\tt BoolExpression} is almost a direct copy of the arithmetic procedure Expression:

\begin{verbatim}
{------------------------------------------------------}
{ Recognize and Translate a Boolean OR }

procedure BoolOr;
begin
   Match('|');
   BoolTerm;
   EmitLn('OR (SP)+,D0');
end;

{------------------------------------------------------}
{ Recognize and Translate an Exclusive Or }

procedure BoolXor;
begin
   Match('~');
   BoolTerm;
   EmitLn('EOR (SP)+,D0');
end;

{---------------------------------------------------------------}
{ Parse and Translate a Boolean Expression }

procedure BoolExpression;
begin
   BoolTerm;
   while IsOrOp(Look) do begin
      EmitLn('MOVE D0,-(SP)');
      case Look of
       '|': BoolOr;
       '~': BoolXor;
      end;
   end;
end;
{---------------------------------------------------------------}
\end{verbatim}

Note the new recognizer  {\tt IsOrOp}, which is also a copy, this time of {\tt IsAddOp}:

\begin{verbatim}
{------------------------------------------------------}
{ Recognize a Boolean Orop }

function IsOrop(c: char): Boolean;
begin
   IsOrop := c in ['|', '~'];
end;
{------------------------------------------------------}
\end{verbatim}

OK, rename the old  version  of {\tt  BoolExpression} to {\tt BoolTerm}, then enter  the  code  above. Compile and test this version. At this point, the  output  code  is  starting  to  look pretty good. Of course, it doesn't make much sense to do a lot of Boolean algebra on  constant values, but we'll soon be  expanding  the  types  of Booleans we deal with.

You've  probably  already  guessed  what  the next step  is:  The Boolean version of Term.

Rename the current procedure {\tt BoolTerm} to {\tt NotFactor}, and enter the following new version of {\tt BoolTerm}. Note that is is  much simpler than  the  numeric  version, since  there  is  no equivalent  of division.

\begin{verbatim}
{---------------------------------------------------------------}
{ Parse and Translate a Boolean Term }

procedure BoolTerm;
begin
   NotFactor;
   while Look = '&' do begin
      EmitLn('MOVE D0,-(SP)');
      Match('&');
      NotFactor;
      EmitLn('AND (SP)+,D0');
   end;
end;
{------------------------------------------------------}
\end{verbatim}

Now, we're  almost  home. We are  translating  complex  Boolean expressions, although only for constant values. The next step is to allow for the NOT. Write the following procedure:

\begin{verbatim}
{------------------------------------------------------}
{ Parse and Translate a Boolean Factor with NOT }

procedure NotFactor;
begin
   if Look = '!' then begin
      Match('!');
      BoolFactor;
      EmitLn('EOR #-1,D0');
      end
   else
      BoolFactor;
end;
{------------------------------------------------------}
\end{verbatim}

And  rename  the  earlier procedure to {\tt BoolFactor}. Now try that. At this point  the  parser  should  be able to handle any Boolean expression you care to throw at it. Does it?  Does it trap badly formed expressions?

If you've  been  following  what  we  did  in the parser for math expressions, you know  that  what  we  did next was to expand the definition of a factor to include variables and parens. We don't have  to do that for the Boolean  factor, because  those  little items get taken care of by the next step. It  takes  just  a one line addition to {\tt BoolFactor} to take care of relations:

\begin{verbatim}
{------------------------------------------------------}
{ Parse and Translate a Boolean Factor }

procedure BoolFactor;
begin
   if IsBoolean(Look) then
      if GetBoolean then
         EmitLn('MOVE #-1,D0')
      else
         EmitLn('CLR D0')
      else Relation;
end;
{------------------------------------------------------}
\end{verbatim}

You  might be wondering when I'm going  to  provide  for  Boolean variables and parenthesized Boolean expressions. The  answer is, I'm NOT!   Remember, we  took  those out of the grammar earlier. Right now all I'm  doing  is  encoding  the grammar we've already agreed  upon. The compiler itself can't  tell  the  difference between a Boolean variable  or  expression  and an arithmetic one ... all of those will be handled by Relation, either way.

Of course, it would help to have some code for Relation. I don't feel comfortable, though, adding  any  more  code  without first checking out what we already have. So for now let's just write a dummy  version  of  Relation  that  does nothing except  eat  the current character, and write a little message:

\begin{verbatim}
{------------------------------------------------------}
{ Parse and Translate a Relation }

procedure Relation;
begin
   WriteLn('<Relation>');
   GetChar;
end;
{------------------------------------------------------}
\end{verbatim}

OK, key  in  this  code  and  give  it a try. All the old things should still work ... you should be able to generate the code for ANDs, ORs, and  NOTs. In  addition, if you type any alphabetic character you should get a little <Relation>  place-holder, where a  Boolean factor should be. Did you get that?  Fine, then let's move on to the full-blown version of Relation.

To  get  that, though, there is a bit of groundwork that we must lay first. Recall that a relation has the form

\begin{verbatim}
 <relation>   ::= | <expression> [<relop> <expression]
\end{verbatim}

Since  we have a new kind of operator, we're also going to need a new Boolean function to  recognize  it. That function is shown below. Because of the single-character limitation, I'm sticking to the four operators  that  can be encoded with such a character (the ``not equals'' is encoded by `\#').

\begin{verbatim}
{------------------------------------------------------}
{ Recognize a Relop }

function IsRelop(c: char): Boolean;
begin
   IsRelop := c in ['=', '#', '<', '>'];
end;
{------------------------------------------------------}
\end{verbatim}

Now, recall  that  we're  using  a zero or a -1 in register D0 to represent  a Boolean value, and also  that  the  loop  constructs expect the flags to be set to correspond. In  implementing  all this on the 68000, things get a a little bit tricky.

Since the loop constructs operate only on the flags, it  would be nice (and also quite  efficient)  just to set up those flags, and not load  anything  into  D0  at all. This would be fine for the loops  and  branches, but remember that the relation can be used {\bfseries anywhere} a Boolean factor could be  used. We may be storing its result to a Boolean variable. Since we can't know at  this point how the result is going to be used, we must allow for BOTH cases.

Comparing numeric data  is  easy  enough  ... the  68000  has an operation  for  that ... but it sets  the  flags, not  a  value. What's more, the  flags  will  always  be  set the same (zero if equal, etc.), while we need the zero flag set differently for the each of the different relops.

The solution is found in the 68000 instruction Scc, which  sets a byte value to 0000 or FFFF (funny how that works!) depending upon the  result  of  the  specified   condition. If  we  make  the destination byte to be D0, we get the Boolean value needed.

Unfortunately, there's one  final  complication:  unlike  almost every other instruction in the 68000 set, Scc does NOT  reset the condition flags to match the data being stored. So we have to do one last step, which is to test D0 and set the flags to match it. It must seem to be a trip around the moon to get what we want: we first perform the test, then test the flags to set data  into D0, then test D0 to set the flags again. It  is  sort of roundabout, but it's the most straightforward way to get the flags right, and after all it's only a couple of instructions.

I  might  mention  here that this area is, in my opinion, the one that represents the biggest difference between the  efficiency of hand-coded assembler language and  compiler-generated  code. We have  seen  already  that  we  lose   efficiency   in  arithmetic operations, although later I plan to show you how to improve that a  bit. We've also seen that the control constructs themselves can be done quite efficiently  ... it's usually very difficult to improve  on  the  code generated for an  IF  or  a  WHILE. But virtually every compiler I've ever seen generates  terrible code, compared to assembler, for the computation of a Boolean function, and particularly for relations. The  reason  is just what I've hinted at above. When I'm writing code in assembler, I  go ahead and perform the test the most convenient way I can, and  then set up the branch so that it goes the way it should. In  effect, I ``tailor''  every  branch  to the situation. The compiler can't do that (practically), and it also can't know that we don't  want to store the result of the test as a Boolean variable. So it must generate  the  code  in a very strict order, and it often ends up loading  the  result  as  a  Boolean  that  never gets  used  for anything.

In  any  case, we're now ready to look at the code for Relation. It's shown below with its companion procedures:

\begin{verbatim}
{---------------------------------------------------------------}
{ Recognize and Translate a Relational "Equals" }

procedure Equals;
begin
   Match('=');
   Expression;
   EmitLn('CMP (SP)+,D0');
   EmitLn('SEQ D0');
end;

{---------------------------------------------------------------}
{ Recognize and Translate a Relational "Not Equals" }

procedure NotEquals;
begin
   Match('#');
   Expression;
   EmitLn('CMP (SP)+,D0');
   EmitLn('SNE D0');
end;

{-------------------------------------------------------}
{ Recognize and Translate a Relational "Less Than" }

procedure Less;
begin
   Match('<');
   Expression;
   EmitLn('CMP (SP)+,D0');
   EmitLn('SGE D0');
end;

{-------------------------------------------------------}
{ Recognize and Translate a Relational "Greater Than" }

procedure Greater;
begin
   Match('>');
   Expression;
   EmitLn('CMP (SP)+,D0');
   EmitLn('SLE D0');
end;

{-------------------------------------------------------}
{ Parse and Translate a Relation }

procedure Relation;
begin
   Expression;
   if IsRelop(Look) then begin
      EmitLn('MOVE D0,-(SP)');
      case Look of
       '=': Equals;
       '#': NotEquals;
       '<': Less;
       '>': Greater;
      end;
   EmitLn('TST D0');
   end;
end;
{-------------------------------------------------------}
\end{verbatim}

Now, that call to  Expression  looks familiar!  Here is where the editor of your system comes in handy. We have  already generated code  for  Expression  and its buddies in previous sessions. You can  copy  them  into your file now. Remember to use the single-character  versions. Just to be  certain, I've  duplicated  the arithmetic procedures below. If  you're  observant, you'll also see that I've changed them a little to make  them  correspond  to the latest version of the syntax. This change is  NOT necessary, so  you  may  prefer  to  hold  off  on  that  until you're  sure everything is working.

\begin{verbatim}
{-------------------------------------------------------}
{ Parse and Translate an Identifier }

procedure Ident;
var Name: char;
begin
   Name:= GetName;
   if Look = '(' then begin
      Match('(');
      Match(')');
      EmitLn('BSR ' + Name);
      end
   else
      EmitLn('MOVE ' + Name + '(PC),D0');
end;

{-------------------------------------------------------}
{ Parse and Translate a Math Factor }

procedure Expression; Forward;

procedure Factor;
begin
   if Look = '(' then begin
      Match('(');
      Expression;
      Match(')');
      end
   else if IsAlpha(Look) then
      Ident
   else
      EmitLn('MOVE #' + GetNum + ',D0');
end;

{-------------------------------------------------------}
{ Parse and Translate the First Math Factor }


procedure SignedFactor;
begin
   if Look = '+' then
      GetChar;
   if Look = '-' then begin
      GetChar;
      if IsDigit(Look) then
         EmitLn('MOVE #-' + GetNum + ',D0')
      else begin
         Factor;
         EmitLn('NEG D0');
      end;
   end
   else Factor;
end;

{------------------------------------------------------}
{ Recognize and Translate a Multiply }

procedure Multiply;
begin
   Match('*');
   Factor;
   EmitLn('MULS (SP)+,D0');
end;

{------------------------------------------------------}
{ Recognize and Translate a Divide }

procedure Divide;
begin
   Match('/');
   Factor;
   EmitLn('MOVE (SP)+,D1');
   EmitLn('EXS.L D0');
   EmitLn('DIVS D1,D0');
end;

{-------------------------------------------------------}
{ Parse and Translate a Math Term }

procedure Term;
begin
   SignedFactor;
   while Look in ['*', '/'] do begin
      EmitLn('MOVE D0,-(SP)');
      case Look of
       '*': Multiply;
       '/': Divide;
      end;
   end;
end;

{-------------------------------------------------------}
{ Recognize and Translate an Add }

procedure Add;
begin
   Match('+');
   Term;
   EmitLn('ADD (SP)+,D0');
end;

{-------------------------------------------------------}
{ Recognize and Translate a Subtract }

procedure Subtract;
begin
   Match('-');
   Term;
   EmitLn('SUB (SP)+,D0');
   EmitLn('NEG D0');
end;


{-------------------------------------------------------}
{ Parse and Translate an Expression }

procedure Expression;
begin
   Term;
   while IsAddop(Look) do begin
      EmitLn('MOVE D0,-(SP)');
      case Look of
       '+': Add;
       '-': Subtract;
      end;
   end;
end;
{-------------------------------------------------------}
\end{verbatim}

There you have it ... a parser that can  handle  both  arithmetic AND Boolean algebra, and things  that combine the two through the use of relops. I suggest you file away a copy of this parser in a safe place for future reference, because in our next step we're going to be chopping it up.

\section{Merging With Control Constructs}

At this point, let's go back to the file we had  previously built that parses control  constructs. Remember  those  little dummy procedures called Condition and  Expression?    Now you know what goes in their places!

I  warn you, you're going to have to  do  some  creative  editing here, so take your time and get it right. What you need to do is to copy all of  the  procedures from the logic parser, from {\tt Ident} through  {\tt BoolExpression}, into the parser for control  constructs. Insert  them  at  the current location of Condition. Then delete that  procedure, as  well as the dummy Expression. Next, change every call  to  Condition  to  refer  to  {\tt BoolExpression} instead. Finally, copy the procedures {\tt IsMulop}, {\tt IsOrOp}, {\tt IsRelop}, {\tt IsBoolean}, and {\tt GetBoolean} into place. That should do it.

Compile  the  resulting program and give it  a  try. Since  we haven't  used  this  program in awhile, don't forget that we used single-character tokens for IF, WHILE, etc. Also don't forget that any letter not a keyword just gets echoed as a block.

Try

\begin{verbatim}
     ia=bxlye
\end{verbatim}

which stands for ``\verb|IF a=b X ELSE Y ENDIF|''.

What do you think?  Did it work?  Try some others.

\section{Adding Assignment}

As long as we're this far, and  we already have the routines for expressions in place, we might  as well replace the ``blocks'' with real assignment statements. We've already done that before, so it won't be too hard. Before  taking that step, though, we need to fix something else.

We're soon going to find  that the one-line ``programs'' that we're having to write here will really cramp our style. At  the moment we  have  no  cure for that, because our parser doesn't recognize the end-of-line characters, the carriage return (CR) and the line feed (LF). So before going any further let's plug that hole.

There are  a  couple  of  ways to deal with the CR/LFs. One (the C/Unix approach) is just to  treat them as additional white space characters  and  ignore  them. That's actually not such a  bad approach, but  it  does  sort  of produce funny results for  our parser as  it  stands  now. If it were reading its input from a source file as  any  self-respecting  REAL  compiler  does, there would be no problem. But we're reading input from  the keyboard, and we're sort of conditioned  to expect something to happen when we hit the return key. It won't, if we just skip over the CR and LF  (try it). So I'm going to use a different method here, which is NOT necessarily the  best  approach in the long run. Consider it a temporary kludge until we're further along.

Instead of skipping the CR/LF, We'll let the parser go ahead and catch them, then  introduce  a  special  procedure, analogous to {\tt SkipWhite}, that skips them only in specified ``legal'' spots.

Here's the procedure:

\begin{verbatim}
{------------------------------------------------------}
{ Skip a CRLF }

procedure Fin;
begin
   if Look = CR then GetChar;
   if Look = LF then GetChar;
end;

{------------------------------------------------------}
\end{verbatim}

Now, add two calls to Fin in procedure Block, like this:

\begin{verbatim}
{------------------------------------------------------}
{ Recognize and Translate a Statement Block }

procedure Block(L: string);
begin
   while not(Look in ['e', 'l', 'u']) do begin
      Fin;
      case Look of
       'i': DoIf(L);
       'w': DoWhile;
       'p': DoLoop;
       'r': DoRepeat;
       'f': DoFor;
       'd': DoDo;
       'b': DoBreak(L);
       else Other;
      end;
      Fin;
 end;
end;
{------------------------------------------------------}
\end{verbatim}

Now, you'll find that you  can use multiple-line ``programs''.  The only restriction is that you can't separate an IF or  WHILE token from its predicate.

Now we're ready to include  the  assignment  statements. Simply change  that  call  to  Other  in  procedure  Block  to a call to Assignment, and add  the  following procedure, copied from one of our  earlier  programs. Note   that   Assignment   now  calls {\tt BoolExpression}, so that we can assign Boolean variables.

\begin{verbatim}
{------------------------------------------------------}
{ Parse and Translate an Assignment Statement }

procedure Assignment;
var Name: char;
begin
   Name := GetName;
   Match('=');
   BoolExpression;
   EmitLn('LEA ' + Name + '(PC),A0');
   EmitLn('MOVE D0,(A0)');
end;
{------------------------------------------------------}
\end{verbatim}

With  that change, you should now be  able  to  write  reasonably realistic-looking  programs, subject  only  to our limitation on single-character tokens. My original intention was to get rid of that limitation for you, too. However, that's going to require a fairly major change to what we've  done  so  far. We need a true lexical scanner, and that requires some structural changes. They are not BIG changes that require us to  throw  away  all  of what we've done so far ... with care, it can be done with very minimal changes, in fact. But it does require that care.

This installment  has already gotten pretty long, and it contains some pretty heavy stuff, so I've decided to leave that step until next  time, when you've had a little more  time  to  digest  what we've done and are ready to start fresh.

In the next installment, then, we'll build a lexical scanner and eliminate the single-character  barrier  once and for all. We'll also write our first complete  compiler, based on what we've done in this session. See you then.
