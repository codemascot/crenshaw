
\chapter{A TOP VIEW}

16th Apr 1989

\section{Introduction}

In  the  previous  installments, we  have  learned  many of  the techniques required to  build  a full-blown compiler. We've done both  assignment   statements   (with   Boolean   and  arithmetic expressions), relational operators, and control constructs. We still haven't  addressed procedure or function calls, but even so we  could  conceivably construct a  mini-language  without  them. I've  always  thought  it would be fun to see just  how  small  a language  one  could  build  that  would still be useful. We're ALMOST in a position to do that now. The  problem  is: though we know  how  to  parse and translate the constructs, we still don't know quite how to put them all together into a language.

In those earlier installments, the  development  of  our programs had  a decidedly bottom-up flavor. In  the  case  of  expression parsing, for  example, we  began  with  the  very lowest  level constructs, the individual constants  and  variables, and worked our way up to more complex expressions.

Most people regard  the  top-down design approach as being better than  the  bottom-up  one. I do too, but  the  way  we  did  it certainly seemed natural enough for the kinds of  things  we were parsing.

You must not get  the  idea, though, that the incremental approach that  we've  been  using  in  all these tutorials  is  inherently bottom-up. In  this  installment  I'd  like to show you that the approach can work just as well when applied from the top down ... maybe better. We'll consider languages such as C and Pascal, and see how complete compilers can be built starting from the top.

In the next installment, we'll  apply the same technique to build a  complete  translator  for a subset of the KISS language, which I'll be  calling  TINY. But one of my goals for this series is that you will  not only be able to see how a compiler for TINY or KISS  works, but  that you will also be able to design and build compilers for your own languages. The C and Pascal examples will help. One  thing I'd like you  to  see  is  that  the  natural structure of the compiler depends very much on the language being translated, so the simplicity and  ease  of  construction  of the compiler  depends  very  much  on  letting the language  set  the program structure.

It's  a bit much to produce a full C or Pascal compiler here, and we won't try. But we can flesh out the top levels far enough so that you can see how it goes.

Let's get started.

\section{The Top Level}

One of the biggest  mistakes  people make in a top-down design is failing  to start at the true top. They think they know what the overall structure of the  design  should be, so they go ahead and write it down.

Whenever  I  start a new design, I always like to do  it  at  the absolute beginning. In  program design language (PDL), this top level looks something like:

\begin{verbatim}
     begin
        solve the problem
     end
\end{verbatim}

OK, I grant  you that this doesn't give much of a hint as to what the next level is, but I  like  to  write it down anyway, just to give me that warm feeling that I am indeed starting at the top.

For our problem, the overall function of a compiler is to compile a complete program. Any definition of the  language, written in BNF, begins here. What does the top level BNF look like?  Well, that depends quite a bit on the language to be translated. Let's take a look at Pascal.

\section{The Structure Of Pascal}

Most  texts  for  Pascal  include  a   BNF   or ``railroad-track'' definition of the language. Here are the first few lines of one:

\begin{verbatim}
     <program>        ::= <program-header> <block> '.'
     <program-header> ::= PROGRAM <ident>
     <block>          ::= <declarations> <statements>
\end{verbatim}


We can write recognizers  to  deal  with  each of these elements, just as we've done before. For each one, we'll use  our familiar single-character tokens to represent the input, then flesh things out a little at a time. Let's begin with the first recognizer: the program itself.

To translate this, we'll  start  with a fresh copy of the Cradle. Since we're back to single-character  names, we'll just use a `p' to stand for `PROGRAM.'

To a fresh copy of the cradle, add the following code, and insert a call to it from the main program:

\begin{verbatim}
{------------------------------------------------------}
{ Parse and Translate A Program }

procedure Prog;
var  Name: char;
begin
   Match('p');            { Handles program header part }
   Name := GetName;
   Prolog(Name);
   Match('.');
   Epilog(Name);
end;
{------------------------------------------------------}
\end{verbatim}

The procedures  Prolog and Epilog perform whatever is required to let the program interface with the operating system, so  that it can execute as a program. Needless to  say, this  part  will be VERY OS-dependent. Remember, I've been emitting code for a 68000 running under the OS I use, which is SK*DOS. I  realize most of you are using PC's  and  would rather see something else, but I'm in this thing too deep to change now!

Anyhow, SK*DOS is a  particularly  easy OS to interface to. Here is the code for Prolog and Epilog:

\begin{verbatim}
{------------------------------------------------------}
{ Write the Prolog }

procedure Prolog;
begin
   EmitLn('WARMST EQU $A01E');
end;

{------------------------------------------------------}
{ Write the Epilog }

procedure Epilog(Name: char);
begin
   EmitLn('DC WARMST');
   EmitLn('END ' + Name);
end;
{------------------------------------------------------}
\end{verbatim}

As usual, add  this  code  and  try  out the ``compiler''.  At this point, there is only one legal input:

\begin{verbatim}
     px. (where x is any single letter, the program name)
\end{verbatim}

Well, as  usual  our first effort is rather unimpressive, but by now  I'm sure you know that things  will  get  more  interesting. There is one important thing to  note:   THE OUTPUT IS A WORKING, COMPLETE, AND EXECUTABLE PROGRAM (at least after it's assembled).

This  is  very  important. The  nice  feature  of  the  top-down approach is that at any stage you can  compile  a  subset  of the complete language and get  a  program that will run on the target machine. From here on, then, we  need  only  add  features  by fleshing out the language constructs. It's all  very  similar to what we've been doing all along, except that we're approaching it from the other end.

\section{Fleshing It Out}

To flesh out  the  compiler, we  only have to deal with language features  one by one. I like to start with a stub procedure that does  nothing, then add detail in  incremental  fashion. Let's begin  by  processing  a block, in accordance with its PDL above. We can do this in two stages. First, add the null procedure:

\begin{verbatim}
{------------------------------------------------------}
{ Parse and Translate a Pascal Block }

procedure DoBlock(Name: char);
begin
end;
{------------------------------------------------------}
\end{verbatim}

and modify Prog to read:

\begin{verbatim}
{------------------------------------------------------}
{ Parse and Translate A Program }

procedure Prog;
var  Name: char;
begin
   Match('p');
   Name := GetName;
   Prolog;
   DoBlock(Name);
   Match('.');
   Epilog(Name);
end;
{------------------------------------------------------}
\end{verbatim}

That certainly  shouldn't change the behavior of the program, and it doesn't. But now the  definition  of Prog is complete, and we can proceed to flesh out DoBlock. That's done right from its BNF definition:

\begin{verbatim}
{------------------------------------------------------}
{ Parse and Translate a Pascal Block }

procedure DoBlock(Name: char);
begin
   Declarations;
   PostLabel(Name);
   Statements;
end;
{------------------------------------------------------}
\end{verbatim}

The  procedure {\tt PostLabel}  was  defined  in  the  installment  on branches. Copy it into your cradle.

I probably need to  explain  the  reason  for inserting the label where I have. It has to do with the operation of SK*DOS. Unlike some OS's, SK*DOS allows the entry point to the main  program to be  anywhere in the program. All you have to do is to give  that point a name. The call  to  {\tt PostLabel} puts that name just before the first executable statement  in  the  main  program. How does SK*DOS know which of the many labels is the entry point, you ask?  It's the one that matches the END statement  at  the  end  of the program.

OK, now  we  need  stubs  for  the  procedures Declarations  and Statements. Make them null procedures as we did before.

Does the program  still run the same?  Then we can move on to the next stage.

\section{Declarations}

The BNF for Pascal declarations is:

\begin{verbatim}
     <declarations> ::= ( <label list>    |
                          <constant list> |
                          <type list>     |
                          <variable list> |
                          <procedure>     |
                          <function>         )*
\end{verbatim}

(Note  that  I'm  using the more liberal definition used by Turbo Pascal. In the standard Pascal definition, each  of  these parts must be in a specific order relative to the rest.)

As  usual, let's  let a single character represent each of these declaration types. The new form of Declarations is:

\begin{verbatim}
{------------------------------------------------------}
{ Parse and Translate the Declaration Part }

procedure Declarations;
begin
   while Look in ['l', 'c', 't', 'v', 'p', 'f'] do
      case Look of
       'l': Labels;
       'c': Constants;
       't': Types;
       'v': Variables;
       'p': DoProcedure;
       'f': DoFunction;
      end;
end;
{------------------------------------------------------}
\end{verbatim}

Of course, we need stub  procedures for each of these declaration types. This time, they  can't  quite  be null procedures, since otherwise we'll end up with an infinite While loop. At  the very least, each recognizer must  eat  the  character that invokes it. Insert the following procedures:

\begin{verbatim}
{------------------------------------------------------}
{ Process Label Statement }

procedure Labels;
begin
   Match('l');
end;

{------------------------------------------------------}
{ Process Const Statement }

procedure Constants;
begin
   Match('c');
end;

{------------------------------------------------------}
{ Process Type Statement }
procedure Types;
begin
   Match('t');
end;

{------------------------------------------------------}
{ Process Var Statement }

procedure Variables;
begin
   Match('v');
end;

{------------------------------------------------------}
{ Process Procedure Definition }

procedure DoProcedure;
begin
   Match('p');
end;

{------------------------------------------------------}
{ Process Function Definition }

procedure DoFunction;
begin
   Match('f');
end;
{------------------------------------------------------}
\end{verbatim}

Now try out the  compiler  with a few representative inputs. You can  mix  the  declarations any way you like, as long as the last character  in  the  program is'.' to  indicate  the  end  of  the program. Of course, none  of  the declarations actually declare anything, so you don't need  (and can't use) any characters other than those standing for the keywords.

We can flesh out the statement  part  in  a similar way. The BNF for it is:

{\small
\begin{verbatim}
<statements>         ::= <compound statement>
<compound statement> ::= BEGIN <statement> (';' <statement>) END
\end{verbatim} }

Note that statements can  begin  with  any identifier except END. So the first stub form of procedure Statements is:

\begin{verbatim}
{------------------------------------------------------}
{ Parse and Translate the Statement Part }

procedure Statements;
begin
   Match('b');
   while Look <> 'e' do
      GetChar;
   Match('e');
end;
{------------------------------------------------------}
\end{verbatim}

At  this  point  the  compiler   will   accept   any   number  of declarations, followed by the  BEGIN  block  of the main program. This  block  itself  can contain any characters at all (except an END), but it must be present.

The simplest form of input is now

\begin{verbatim}
     'pxbe.'
\end{verbatim}

Try  it. Also  try  some  combinations  of  this. Make  some deliberate errors and see what happens.

At this point you should be beginning to see the drill. We begin with a stub translator to process a program, then  we  flesh  out each procedure in turn, based  upon its BNF definition. Just as the lower-level BNF definitions add detail and elaborate upon the higher-level ones, the lower-level  recognizers  will  parse more detail  of  the  input  program. When  the  last stub has been expanded, the  compiler  will  be  complete. That's  top-down design/implementation in its purest form.

You might note that even though we've been adding procedures, the output of the program hasn't changed. That's as  it  should  be. At these  top  levels  there  is  no  emitted code required. The recognizers are  functioning as just that: recognizers. They are accepting input sentences, catching bad ones, and channeling good input to the right places, so  they  are  doing their job. If we were to pursue this a bit longer, code would start to appear.

The  next  step  in our expansion should  probably  be  procedure Statements. The Pascal definition is:

{\small
\begin{verbatim}
    <statement>            ::= <simple statement> |
                               <structured statement>
    <simple statement>     ::= <assignment> | <procedure call> |
                               null
    <structured statement> ::= <compound statement> |
                               <if statement>       |
                               <case statement>     |
                               <while statement>    |
                               <repeat statement>   |
                               <for statement>      |
                               <with statement>
\end{verbatim} }

These  are  starting  to look familiar. As a matter of fact, you have already gone  through  the process of parsing and generating code for both assignment statements and control structures. This is where the top level meets our bottom-up  approach  of previous sessions. The constructs will be a little  different  from those we've  been  using  for KISS, but the differences are nothing you can't handle.

I  think  you can get the picture now as to the  procedure. We begin with a complete BNF  description of the language. Starting at  the  top  level, we code  up  the  recognizer  for  that  BNF statement, using stubs  for  the next-level recognizers. Then we flesh those lower-level statements out one by one.

As it happens, the definition of Pascal is  very  compatible with the  use of BNF, and BNF descriptions  of  the  language  abound. Armed  with  such   a   description, you  will  find  it  fairly straightforward to continue the process we've begun.

You  might  have  a go at fleshing a few of these constructs out, just  to get a feel for it. I don't expect you  to  be  able  to complete a Pascal compiler  here  ... there  are too many things such  as  procedures  and types that we haven't addressed yet ... but  it  might  be helpful to try some of the more familiar ones. It will do  you  good  to  see executable programs coming out the other end.

If I'm going to address those issues that we haven't covered yet, I'd rather  do  it  in  the context of KISS. We're not trying to build a complete Pascal  compiler  just yet, so I'm going to stop the expansion of Pascal here. Let's  take  a  look  at  a very different language.

\section{The Structure Of C}

The C language is quite another matter, as you'll see. Texts on C  rarely  include  a BNF definition of  the  language. Probably that's because the language is quite hard to write BNF for.

One reason I'm showing you these structures now is so that  I can impress upon you these two facts:

(1)	The definition of  the  language drives the structure of the compiler. What works for one language may be a disaster for another. It's  a very bad idea to try to  force  a  given structure upon the compiler. Rather, you should let the BNF drive the structure, as we have done here.
(2)	A language that is hard to write BNF for  will  probably  be hard  to  write  a compiler for, as well. C  is  a  popular language, and  it  has  a  reputation  for  letting you  do virtually  anything that is possible to  do. Despite  the success of Small C, C is  NOT  an easy language to parse.

A C program has  less  structure than its Pascal counterpart. At the top level, everything in C is a static declaration, either of data or of a function. We can capture this thought like this:

\begin{verbatim}
     <program>            ::= ( <global declaration> )*

     <global declaration> ::= <data declaration>  |
                              <function>
\end{verbatim}

In Small C, functions  can  only have the default type int, which is not declared. This makes  the  input easy to parse: the first token is either ``int,'' ``char,'' or the name  of  a  function. In Small  C, the preprocessor commands are  also  processed  by  the compiler proper, so the syntax becomes:

{\small
\begin{verbatim}
     <global declaration> ::= '#' <preprocessor command>  |
                              'int' <data list>           |
                              'char' <data list>          |
                              <ident> <function body>     |
\end{verbatim} }

Although we're really more interested in full C  here, I'll show you the  code corresponding to this top-level structure for Small C.

\begin{verbatim}
{------------------------------------------------------}
{ Parse and Translate A Program }

procedure Prog;
begin
   while Look <> ^Z do begin
      case Look of
       '#': PreProc;
       'i': IntDecl;
       'c': CharDecl;
      else DoFunction(Int);
      end;
   end;
end;
{------------------------------------------------------}
\end{verbatim}

Note that I've had to use a \begin{verbatim}^Z\end{verbatim} to indicate the end of the source. C has no keyword such as END or the '.' to otherwise indicate the end.

With full C, things  aren't  even  this easy. The problem comes about because in full C, functions can also have types. So when the compiler sees a  keyword  like  ``int,''  it still doesn't know whether to expect a  data  declaration  or a function definition. Things get more  complicated  since  the  next token may not be a name  ... it may start with an `*' or `(', or combinations of the two.

More specifically, the BNF for full C begins with:

\begin{verbatim}
     <program>        ::= ( <top-level decl> )*
     <top-level decl> ::= <function def> | <data decl>
     <data decl>      ::= [<class>] <type> <decl-list>
     <function def>   ::= [<class>] [<type>] <function decl>
\end{verbatim}

You  can  now  see the problem:   The  first  two  parts  of  the declarations for data and functions can be the same. Because of the  ambiguity  in  the grammar as  written  above, it's  not  a suitable  grammar  for  a  recursive-descent  parser. Can  we transform it into one that is suitable?  Yes, with a little work. Suppose we write it this way:

\begin{verbatim}
     <top-level decl> ::= [<class>] <decl>
     <decl>           ::= <type> <typed decl> | <function decl>
     <typed decl>     ::= <data list> | <function decl>
\end{verbatim}

We  can  build  a  parsing  routine  for  the   class   and  type definitions, and have them store away their findings  and  go on, without their ever having to ``know'' whether a function or  a data declaration is being processed.

To begin, key in the following version of the main program:

\begin{verbatim}
{------------------------------------------------------}
{ Main Program }

begin
   Init;
   while Look <> ^Z do begin
      GetClass;
      GetType;
      TopDecl;
   end;
end.

{------------------------------------------------------}
\end{verbatim}

For the first round, just make the three procedures stubs that do nothing  BUT  call {\tt GetChar}.

Does this program work?  Well, it would be hard put NOT to, since we're not really asking it to do anything. It's been said that a C compiler will accept virtually any input without choking. It's certainly true of THIS  compiler, since in effect all it does is to eat input characters until it finds a \begin{verbatim}^Z\end{verbatim}.

Next, let's make  {\tt GetClass}  do something worthwhile. Declare the global variable

\begin{verbatim}
     var Class: char;
\end{verbatim}

and change {\tt GetClass} to do the following:

\begin{verbatim}
{------------------------------------------------------}
{  Get a Storage Class Specifier }

Procedure GetClass;
begin
   if Look in ['a', 'x', 's'] then begin
      Class := Look;
      GetChar;
      end
   else Class := 'a';
end;
{------------------------------------------------------}
\end{verbatim}

Here, I've used three  single  characters  to represent the three storage classes ``auto,'' ``extern,''  and  ``static.''   These are not the only three possible classes ... there are also ``register'' and ``typedef,'' but this should  give  you the picture. Note that the default class is ``auto''.

We  can  do  a  similar  thing  for  types. Enter the following procedure next:

\begin{verbatim}
{------------------------------------------------------}
{  Get a Type Specifier }

procedure GetType;
begin
   Typ := ' ';
   if Look = 'u' then begin
      Sign := 'u';
      Typ := 'i';
      GetChar;
      end
   else Sign := 's';
   if Look in ['i', 'l', 'c'] then begin
      Typ := Look;
      GetChar;
   end;
end;
{------------------------------------------------------}
\end{verbatim}

Note that you must add two more global variables, Sign and Typ.

With these two procedures in place, the compiler will process the class and type definitions and store away their findings. We can now process the rest of the declaration.

We  are by no means out of the woods yet, because there are still many complexities just in the definition of the  type, before we even get to the actual data or function names. Let's pretend for the moment that we have passed all those gates, and that the next thing in the  input stream is a name. If the name is followed by a left paren, we have a function declaration. If not, we have at least one data item, and  possibly a list, each element of which can have an initializer.

Insert the following version of {\tt TopDecl}:

\begin{verbatim}
{------------------------------------------------------}
{ Process a Top-Level Declaration }

procedure TopDecl;
var Name: char;
begin
   Name := Getname;
   if Look = '(' then
      DoFunc(Name)
   else
      DoData(Name);
end;
{------------------------------------------------------}
\end{verbatim}

(Note that, since we have already read the name, we must  pass it along to the appropriate routine.)

Finally, add the two procedures {\tt DoFunc} and {\tt DoData}:

\begin{verbatim}
{------------------------------------------------------}
{ Process a Function Definition }

procedure DoFunc(n: char);
begin
   Match('(');
   Match(')');
   Match('{');
   Match('}');
   if Typ = ' ' then Typ := 'i';
   Writeln(Class, Sign, Typ, ' function ', n);
end;

{------------------------------------------------------}
{ Process a Data Declaration }

procedure DoData(n: char);
begin
   if Typ = ' ' then Expected('Type declaration');
   Writeln(Class, Sign, Typ, ' data ', n);
   while Look = ',' do begin
      Match(',');
      n := GetName;
      WriteLn(Class, Sign, Typ, ' data ', n);
   end;
   Match(';');
end;
{------------------------------------------------------}
\end{verbatim}

Since  we're  still  a long way from producing executable code, I decided to just have these two routines tell us what they found.

OK, give this program a try. For data declarations, it's OK to give a list separated by commas. We  can't  process initializers as yet. We also can't process argument lists for  the functions, but the \verb|(){}| characters should be there.

We're still a {\tt very}  long way from having a C compiler, but what we have is starting to process the right kinds of inputs, and is recognizing both good  and  bad  inputs. In  the  process, the natural structure of the compiler is starting to take form.

Can we continue this until we have something that acts  more like a compiler. Of course we can. Should we?  That's another matter. I don't know about you, but I'm beginning to get dizzy, and we've still  got  a  long  way  to  go  to  even  get  past   the  data declarations.

At  this  point, I think you can see how the  structure  of  the compiler evolves from the language  definition. The structures we've seen for our  two  examples, Pascal and C, are as different as night and day. Pascal was designed at least partly to be easy to parse, and that's  reflected  in the compiler. In general, in Pascal there is more structure and we have a better idea  of what kinds of constructs to expect at any point. In  C, on the other hand, the  program  is  essentially  a  list   of  declarations, terminated only by the end of file.

We  could  pursue  both  of  these structures much  farther, but remember that our purpose here is  not  to  build a Pascal or a C compiler, but rather to study compilers in general. For those of you  who DO want to deal with Pascal or C, I hope I've given  you enough of a start so that you can  take  it  from  here (although you'll soon need some of the stuff we still haven't  covered yet, such as typing and procedure calls). For the rest of you, stay with me through the next installment. There, I'll be leading you through the development of a complete compiler for TINY, a subset of KISS.

See you then.
