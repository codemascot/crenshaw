
\chapter{INTRODUCING ``TINY''}

21st May 1989

\section{Introduction}

In the last installment, I showed you the general  idea  for  the top-down development of  a  compiler. I gave you the first few steps  of  the process for compilers for  Pascal  and  C, but  I stopped  far  short  of  pushing  it through to completion. The reason was simple: if we're going to produce  a  real, functional compiler  for  any  language, I'd rather  do  it  for  KISS, the language that I've been defining in this tutorial series.

In this installment, we're going to do just that, for a subset of KISS which I've chosen to call TINY.

The process  will be essentially that outlined in Installment IX, except  for  one  notable  difference. In that  installment, I suggested  that  you  begin  with  a full BNF description of  the language. That's fine for something like Pascal or C, for which the language definition is  firm. In the case of TINY, however, we don't yet have a full  description  ... we seem to be defining the language as we go. That's OK. In  fact, it's preferable, since we can tailor the language  slightly  as we go, to keep the parsing easy.

So in the development  that  follows, we'll  actually be doing a top-down development of BOTH the  language and its compiler. The BNF description will grow along with the compiler.

In this process, there will be a number of decisions to  be made, each of which will influence the BNF and therefore the  nature of the language. At  each  decision  point I'll try to remember to explain  the  decision  and the rationale behind my choice. That way, if you happen to hold a different opinion and would prefer a different option, you can choose it instead. You  now  have  the background  to  do  that. I guess the important thing to note is that  nothing  we  do  here  is  cast  in  concrete. When YOU'RE designing YOUR language, you should feel free to do it YOUR way.

Many of you may be asking at this point: Why bother starting over from  scratch?  We had a working subset of KISS as the outcome of Installment  VII  (lexical  scanning). Why not just extend it as needed?  The  answer  is  threefold. First of all, I have been making  a  number  of changes to further simplify the program ... changes  like  encapsulating  the  code generation procedures, so that  we  can  convert to a different target machine more easily. Second, I want you to see how the development can indeed  be done from the top down as outlined in the last installment. Finally, we both need the practice. Each time I go through this exercise, I get a little better at it, and you will, also.

\section{Getting Started}

Many  years  ago  there were languages called  Tiny  BASIC, Tiny Pascal, and Tiny C, each of which was a subset of its parent full language. Tiny BASIC, for  example, had  only single-character variable names and global variables. It supported only a single data type. Sound familiar?  At this point we have almost all the tools we need to build a compiler like that.

Yet a language called Tiny-anything  still  carries  some baggage inherited from its parent language. I've often wondered if this is a  good  idea. Granted, a  language based upon some parent language will have the  advantage  of  familiarity, but there may also  be  some  peculiar syntax carried over from the parent that may tend  to add unnecessary complexity to the compiler. (Nowhere is this more true than in Small C.)

I've wondered just how small and simple a compiler could  be made and  still  be  useful, if it were designed from the outset to be both easy to use and to  parse. Let's find out. This language will just be called "TINY," period. It's a subset of KISS, which I  also  haven't  fully  defined, so  that  at  least  makes  us consistent (!). I suppose you could call it TINY KISS. But that opens  up a whole can of worms involving  cuter  and  cuter  (and perhaps more risque) names, so let's just stick with TINY.

The main limitations  of  TINY  will  be because of the things we haven't yet covered, such as data types. Like its cousins Tiny C and Tiny BASIC, TINY  will  have  only one data type, the 16-bit integer. The  first  version  we  develop  will also  have  no procedure  calls  and  will  use single-character variable names, although as you will see we can remove these restrictions without much effort.

The language I have in mind will share some of the  good features of  Pascal, C, and Ada. Taking a lesson from the comparison of the Pascal and  C  compilers in the previous installment, though, TINY will have a decided Pascal flavor. Wherever  feasible, a language structure will  be  bracketed by keywords or symbols, so that  the parser will know where it's  going  without  having  to guess.

One other ground rule:  As we go, I'd like  to  keep the compiler producing real, executable code. Even though it may not  DO much at the beginning, it will at least do it correctly.

Finally, I'll  use  a couple of Pascal  restrictions  that  make sense:  All data and procedures must be declared before  they are used. That makes good sense, even  though for now the only data type we'll use  is a word. This rule in turn means that the only reasonable place to put the  executable code for the main program is at the end of the listing.

The top-level definition will be similar to Pascal:

\begin{verbatim}
     <program> ::= PROGRAM <top-level decl> <main> '.'
\end{verbatim}

Already, we've reached a decision point. My first thought was to make the main block optional. It  doesn't seem to make sense to write a "program" with no main program, but it does make sense if we're  allowing  for  multiple modules, linked together. As  a matter of fact, I intend to allow for this in KISS. But then we begin  to open up a can of worms that I'd rather leave closed for now. For example, the  term "PROGRAM" really becomes a misnomer. The MODULE of Modula-2 or the Unit of Turbo Pascal would  be more appropriate. Second, what  about  scope  rules?    We'd  need a convention for  dealing  with  name  visibility  across  modules. Better  for  now  to  just  keep  it  simple  and ignore the idea altogether.

There's also a decision in choosing to require  the  main program to  be  last. I  toyed  with  the idea of making its  position optional, as  in  C. The nature of SK*DOS, the OS I'm compiling for, make this very easy to do. But  this  doesn't  really make much sense in view of the Pascal-like requirement  that  all data and procedures  be declared before they're referenced. Since the main  program can only call procedures  that  have  already  been declared, the only position that makes sense is at the end, a la Pascal.

Given  the  BNF  above, let's write a parser that just recognizes the brackets:

\begin{verbatim}
{--------------------------------------------------------------}
{  Parse and Translate a Program }

procedure Prog;
begin
   Match('p');
   Header;
   Prolog;
   Match('.');
   Epilog;
end;
{--------------------------------------------------------------}
\end{verbatim}

The procedure Header just emits  the startup code required by the assembler:

\begin{verbatim}
{------------------------------------------------------}
{ Write Header Info }

procedure Header;
begin
   WriteLn('WARMST', TAB, 'EQU $A01E');
end;
{------------------------------------------------------}
\end{verbatim}

The procedures Prolog and  Epilog  emit  the code for identifying the main program, and for returning to the OS:

\begin{verbatim}
{------------------------------------------------------}
{ Write the Prolog }

procedure Prolog;
begin
   PostLabel('MAIN');
end;

{------------------------------------------------------}
{ Write the Epilog }

procedure Epilog;
begin
   EmitLn('DC WARMST');
   EmitLn('END MAIN');
end;
{------------------------------------------------------}
\end{verbatim}

The  main program just calls Prog, and then  looks  for  a  clean ending:

\begin{verbatim}
{------------------------------------------------------}
{ Main Program }

begin
   Init;
   Prog;
   if Look <> CR then Abort('Unexpected data after ''.''');
end.
{------------------------------------------------------}
\end{verbatim}

At this point, TINY  will  accept  only  one input "program," the null program:

\begin{verbatim}
     PROGRAM . (or 'p.' in our shorthand.)
\end{verbatim}

Note, though, that the  compiler  DOES  generate correct code for this program. It will run, and do  what  you'd  expect  the null program to do, that is, nothing but return gracefully to the OS.

As a matter of interest, one of my  favorite  compiler benchmarks is to compile, link, and  execute  the  null program in whatever language   is   involved. You  can  learn  a  lot  about  the implementation by measuring  the  overhead  in  time  required to compile what should be a trivial case. It's also  interesting to measure the amount of code produced. In many compilers, the code can be fairly large, because they always include  the  whole run- time  library whether they need it or not. Early  versions  of Turbo Pascal produced a 12K object file for  this  case. VAX C generates 50K!

The  smallest  null  programs  I've  seen are those  produced  by Modula-2 compilers, and they run about 200-800 bytes.

In the case of TINY, we {\bfseries have} no run-time library  as  yet, so the object code is indeed tiny:  two  bytes. That's  got  to  be a record, and it's  likely  to  remain  one since it is the minimum size required by the OS.

The  next step is to process the code for the main program. I'll use the Pascal BEGIN-block:

\begin{verbatim}
     <main> ::= BEGIN <block> END
\end{verbatim}

Here, again, we  have made a decision. We could have chosen to require a "PROCEDURE MAIN" sort of declaration, similar to C. I must  admit  that  this  is  not  a bad idea at all ... I  don't particularly  like  the  Pascal  approach  since I tend  to  have trouble locating the main  program  in a Pascal listing. But the alternative is a little awkward, too, since you have to deal with the  error condition where the user omits  the  main  program  or misspells its name. Here I'm taking the easy way out.

Another solution to the "where is the main program" problem might be to require a name for  the  program, and then bracket the main by

\begin{verbatim}
     BEGIN <name>
     END <name>
\end{verbatim}

similar to the convention of  Modula  2. This  adds  a  bit of ``syntactic sugar'' to the language. Things like this are  easy to add or change to your liking, if the language is your own design.

To parse this definition of a main block, change  procedure Prog to read:

\begin{verbatim}
{------------------------------------------------------}
{  Parse and Translate a Program }

procedure Prog;
begin
   Match('p');
   Header;
   Main;
   Match('.');
end;
{------------------------------------------------------}
\end{verbatim}

and add the new procedure:

\begin{verbatim}
{------------------------------------------------------}
{ Parse and Translate a Main Program }

procedure Main;
begin
   Match('b');
   Prolog;
   Match('e');
   Epilog;
end;
{------------------------------------------------------}
\end{verbatim}

Now, the only legal program is:

\begin{verbatim}
     PROGRAM BEGIN END . (or 'pbe.')
\end{verbatim}

Aren't we making progress???  Well, as usual it gets better. You might try some deliberate errors here, like omitting  the  'b' or the 'e', and see what happens. As always, the  compiler  should flag all illegal inputs.


\section{Declarations}

The obvious next step is to decide what we mean by a declaration. My  intent  here  is to have two kinds of declarations: variables and  procedures/functions. At  the  top  level, only  global declarations are allowed, just as in C.

For now, there  can  only be variable declarations, identified by the keyword VAR (abbreviated 'v'):

\begin{verbatim}
     <top-level decls> ::= ( <data declaration> )*
     <data declaration> ::= VAR <var-list>
\end{verbatim}

Note that since there is only one variable type, there is no need to  declare the type. Later on, for full KISS, we can easily add a type description.

The procedure Prog becomes:

\begin{verbatim}
{------------------------------------------------------}
{  Parse and Translate a Program }

procedure Prog;
begin
   Match('p');
   Header;
   TopDecls;
   Main;
   Match('.');
end;
{------------------------------------------------------}
\end{verbatim}

Now, add the two new procedures:

\begin{verbatim}
{------------------------------------------------------}
{ Process a Data Declaration }

procedure Decl;
begin
   Match('v');
   GetChar;
end;

{------------------------------------------------------}
{ Parse and Translate Global Declarations }

procedure TopDecls;
begin
   while Look <> 'b' do
      case Look of
        'v': Decl;
      else Abort('Unrecognized Keyword ''' + Look + '''');
      end;
end;
{------------------------------------------------------}
\end{verbatim}

Note that at this point, Decl  is  just  a stub. It generates no code, and it doesn't process a list ... every variable must occur in a separate VAR statement.

OK, now  we  can have any  number  of  data  declarations, each starting with a `v' for VAR, before  the BEGIN-block. Try a few cases and see what happens.

\section{Declarations And Symbols}

That looks pretty good, but  we're still only generating the null program  for  output. A  real compiler would  issue  assembler directives to allocate storage for  the  variables. It's about time we actually produced some code.

With  a  little  extra  code, that's  an  easy  thing to do from procedure Decl. Modify it as follows:

\begin{verbatim}
{------------------------------------------------------}
{ Parse and Translate a Data Declaration }

procedure Decl;
var Name: char;
begin
   Match('v');
   Alloc(GetName);
end;
{------------------------------------------------------}
\end{verbatim}

The procedure Alloc just  issues  a  command  to the assembler to allocate storage:

\begin{verbatim}
{------------------------------------------------------}
{ Allocate Storage for a Variable }

procedure Alloc(N: char);
begin
   WriteLn(N, ':', TAB, 'DC 0');
end;
{------------------------------------------------------}
\end{verbatim}

Give  this  one  a  whirl. Try  an  input  that declares  some variables, such as:

\begin{verbatim}
     pvxvyvzbe.
\end{verbatim}

See how the storage is allocated?    Simple, huh?  Note also that the entry point, "MAIN," comes out in the right place.

For the record, a "real" compiler would also have a  symbol table to record the variables being used. Normally, the  symbol table is necessary to record the type  of  each variable. But since in this case  all  variables  have  the  same  type, we don't need a symbol  table  for  that reason. As it turns out, we're going to find a symbol  necessary  even without different types, but let's postpone that need until it arises.

Of course, we haven't really parsed the correct syntax for a data declaration, since it involves a variable list. Our version only permits a single variable. That's easy to fix, too.

\begin{verbatim}
The BNF for <var-list> is

     <var-list> ::= <ident> (, <ident>)*
\end{verbatim}

Adding this syntax to Decl gives this new version:

\begin{verbatim}
{------------------------------------------------------}
{ Parse and Translate a Data Declaration }

procedure Decl;
var Name: char;
begin
   Match('v');
   Alloc(GetName);
   while Look = ',' do begin
      GetChar;
      Alloc(GetName);
   end;
end;
{------------------------------------------------------}
\end{verbatim}

OK, now compile this code and give it  a  try. Try a number of lines of VAR declarations, try a list of several variables on one line, and try combinations of the two. Does it work?

\section{Initializers}

As long as we're dealing with data declarations, one thing that's always  bothered  me  about  Pascal  is  that  it  doesn't  allow initializing  data items in the declaration. That  feature  is admittedly sort of a frill, and it  may  be  out  of  place  in a language that purports to  be  a minimal language. But it's also SO easy to add that it seems a shame not  to  do  so. The  BNF becomes:

\begin{verbatim}
     <var-list> ::= <var> ( <var> )*
     <var>      ::= <ident> [ = <integer> ]
\end{verbatim}

Change Alloc as follows:

\begin{verbatim}
{------------------------------------------------------}
{ Allocate Storage for a Variable }

procedure Alloc(N: char);
begin
   Write(N, ':', TAB, 'DC ');
   if Look = '=' then begin
      Match('=');
      WriteLn(GetNum);
      end
   else
      WriteLn('0');
end;
{------------------------------------------------------}
\end{verbatim}

There you are: an initializer with six added lines of Pascal.

OK, try this  version  of  TINY  and verify that you can, indeed, give the variables initial values.

By golly, this thing is starting to look  real!    Of  course, it still doesn't DO anything, but it looks good, doesn't it?

Before leaving this section, I should point out  that  we've used two versions of function GetNum. One, the earlier one, returns a character value, a single digit. The other accepts a multi-digit integer and returns an integer value. Either one will work here, since WriteLn will handle either type. But there's no  reason to limit ourselves  to  single-digit  values  here, so  the correct version to use is the one that returns an integer. Here it is:

\begin{verbatim}
{------------------------------------------------------}
{ Get a Number }

function GetNum: integer;
var Val: integer;
begin
   Val := 0;
   if not IsDigit(Look) then Expected('Integer');
   while IsDigit(Look) do begin
      Val := 10 * Val + Ord(Look) - Ord('0');
      GetChar;
   end;
   GetNum := Val;
end;
{------------------------------------------------------}
\end{verbatim}

As a matter  of  fact, strictly  speaking  we  should  allow for expressions in the data field of the initializer, or at  the very least  for  negative  values. For  now, let's  just  allow  for negative values by changing the code for Alloc as follows:

\begin{verbatim}
{------------------------------------------------------}
{ Allocate Storage for a Variable }

procedure Alloc(N: char);
begin
   if InTable(N) then Abort('Duplicate Variable Name ' + N);
   ST[N] := 'v';
   Write(N, ':', TAB, 'DC ');
   if Look = '=' then begin
      Match('=');
      If Look = '-' then begin
         Write(Look);
         Match('-');
      end;
      WriteLn(GetNum);
      end
   else
      WriteLn('0');
end;
{------------------------------------------------------}
\end{verbatim}

Now  you should be able to  initialize  variables  with  negative and/or multi-digit values.

\section{The Symbol Table}

There's one problem  with  the  compiler  as it stands so far: it doesn't do anything to record a variable when we declare it. So the compiler is perfectly content to allocate storage for several variables with the same name. You can easily verify this with an input like

\begin{verbatim}
     pvavavabe.
\end{verbatim}

Here we've declared the variable A three times. As you  can see, the compiler will  cheerfully  accept  that, and  generate three identical labels. Not good.

Later on, when we start referencing variables, the compiler will also let us reference variables  that don't exist. The assembler will  catch  both  of these error conditions, but it doesn't seem friendly at all to pass such errors along to the assembler. The compiler should catch such things at the source language level.

So even though we don't need a symbol table to record data types, we ought to install  one  just to check for these two conditions. Since at this  point  we are still restricted to single-character variable names, the symbol table can be trivial. To  provide for it, first add the following  declaration at the beginning of your program:

\begin{verbatim}
     var ST: array['A'..'Z'] of char;
\end{verbatim}

and insert the following function:

\begin{verbatim}
{------------------------------------------------------}
{ Look for Symbol in Table }

function InTable(n: char): Boolean;
begin
   InTable := ST[n] <> ' ';
end;
{-----------------------------------------------------}
\end{verbatim}

We  also  need  to initialize the  table  to  all  blanks. The following lines in Init will do the job:

\begin{verbatim}
var i: char;
begin
   for i := 'A' to 'Z' do
      ST[i] := ' ';
   ...
\end{verbatim}

Finally, insert  the  following two lines at  the  beginning  of Alloc:

\begin{verbatim}
   if InTable(N) then Abort('Duplicate Variable Name ' + N);
   ST[N] := 'v';
\end{verbatim}

That  should  do  it. The  compiler  will  now  catch  duplicate declarations. Later, we can  also  use  InTable  when generating references to the variables.

\section{Executable Statements}

At this point, we can generate a null program that has  some data variables  declared  and  possibly initialized. But  so  far  we haven't arranged to generate the first line of executable code.

Believe  it or not, though, we almost  have  a  usable  language!  What's missing is the executable code that must go into  the main program. But that code is just assignment statements and control statements ... all stuff we have done before. So  it  shouldn't take us long to provide for them, as well.

The BNF definition given earlier  for the main program included a statement block, which we have so far ignored:

\begin{verbatim}
     <main> ::= BEGIN <block> END
\end{verbatim}

For now, we  can  just  consider  a  block  to  be  a  series of assignment statements:

\begin{verbatim}
     <block> ::= (Assignment)*
\end{verbatim}

Let's start things off by adding  a  parser for the block. We'll begin with a stub for the assignment statement:

\begin{verbatim}
{------------------------------------------------------}
{ Parse and Translate an Assignment Statement }

procedure Assignment;
begin
   GetChar;
end;

{------------------------------------------------------}
{ Parse and Translate a Block of Statements }

procedure Block;
begin
   while Look <> 'e' do
      Assignment;
end;
{------------------------------------------------------}
\end{verbatim}


Modify procedure Main to call Block as shown below:

\begin{verbatim}
{------------------------------------------------------}
{ Parse and Translate a Main Program }

procedure Main;
begin
   Match('b');
   Prolog;
   Block;
   Match('e');
   Epilog;
end;
{------------------------------------------------------}
\end{verbatim}

This version still won't generate any code for  the  ``assignment statements'' ... all it does is to eat characters  until  it  sees the 'e' for 'END.'  But it sets the stage for what is to follow.

The  next  step, of  course, is  to  flesh out the code for  an assignment statement. This  is  something  we've done many times before, so  I  won't belabor it. This time, though, I'd like to deal with the code generation a little differently. Up till now, we've always just inserted the Emits that generate output code in line with  the parsing routines. A little unstructured, perhaps, but it seemed the most straightforward approach, and made it easy to see what kind of code would be emitted for each construct.

However, I realize that most of you are using an  80x86 computer, so  the 68000 code generated is of little use to you. Several of you have asked me if the CPU-dependent code couldn't be collected into one spot  where  it  would  be easier to retarget to another CPU. The answer, of course, is yes.

To  accomplish  this, insert  the  following  "code  generation" routines:

\begin{verbatim}
{-------------------------------------------------------}
{ Clear the Primary Register }

procedure Clear;
begin
   EmitLn('CLR D0');
end;


{-------------------------------------------------------}
{ Negate the Primary Register }

procedure Negate;
begin
   EmitLn('NEG D0');
end;

{-------------------------------------------------------}
{ Load a Constant Value to Primary Register }

procedure LoadConst(n: integer);
begin
   Emit('MOVE #');
   WriteLn(n, ',D0');
end;


{-------------------------------------------------------}
{ Load a Variable to Primary Register }

procedure LoadVar(Name: char);
begin
   if not InTable(Name) then Undefined(Name);
   EmitLn('MOVE ' + Name + '(PC),D0');
end;

{-------------------------------------------------------}
{ Push Primary onto Stack }

procedure Push;
begin
   EmitLn('MOVE D0,-(SP)');
end;


{-------------------------------------------------------}
{ Add Top of Stack to Primary }

procedure PopAdd;
begin
   EmitLn('ADD (SP)+,D0');
end;

{-------------------------------------------------------}
{ Subtract Primary from Top of Stack }

procedure PopSub;
begin
   EmitLn('SUB (SP)+,D0');
   EmitLn('NEG D0');
end;


{-------------------------------------------------------}
{ Multiply Top of Stack by Primary }

procedure PopMul;
begin
   EmitLn('MULS (SP)+,D0');
end;

{-------------------------------------------------------}
{ Divide Top of Stack by Primary }

procedure PopDiv;
begin
   EmitLn('MOVE (SP)+,D7');
   EmitLn('EXT.L D7');
   EmitLn('DIVS D0,D7');
   EmitLn('MOVE D7,D0');
end;

{-------------------------------------------------------}
{ Store Primary to Variable }

procedure Store(Name: char);
begin
   if not InTable(Name) then Undefined(Name);
   EmitLn('LEA ' + Name + '(PC),A0');
   EmitLn('MOVE D0,(A0)')
end;
{-------------------------------------------------------}
\end{verbatim}

The  nice  part  of  this  approach, of  course, is that we can retarget  the compiler to a new CPU  simply  by  rewriting  these "code generator" procedures. In  addition, we  will  find later that we can improve the code quality by tweaking these routines a bit, without having to modify the compiler proper.

Note that both LoadVar  and  Store check the symbol table to make sure that the variable is defined. The  error  handler Undefined simply calls Abort:

\begin{verbatim}
{------------------------------------------------------}
{ Report an Undefined Identifier }

procedure Undefined(n: string);
begin
   Abort('Undefined Identifier ' + n);
end;
{------------------------------------------------------}
\end{verbatim}

OK, we are now finally ready to begin processing executable code. We'll  do  that  by  replacing  the  stub  version  of  procedure Assignment.

We've been down this  road  many times before, so this should all be familiar to you. In fact, except for the changes associated with the code generation, we  could just copy the procedures from Section 7. Since we are making some changes, I won't just copy them, but we will go a little faster than usual.

The BNF for the assignment statement is:

\begin{verbatim}
     <assignment>   ::= <ident> = <expression>
     <expression>   ::= <first term> ( <addop> <term> )*
     <first term>   ::= <first factor> <rest>
     <term>         ::= <factor> <rest>
     <rest>         ::= ( <mulop> <factor> )*
     <first factor> ::= [ <addop> ] <factor>
     <factor>       ::= <var> | <number> | ( <expression> )
\end{verbatim}

This version of the BNF is  also  a bit different than we've used before ... yet another "variation on the theme of an expression." This particular version  has  what  I  consider  to  be  the best treatment  of  the  unary minus. As you'll see later, it lets us handle   negative  constant  values  efficiently. It's   worth mentioning  here  that  we  have  often  seen  the advantages  of "tweaking"  the  BNF  as we go, to help make the language easy to parse. What  you're looking at here is a bit different:  we've tweaked  the  BNF  to make the CODE  GENERATION  more  efficient!  That's a first for this series.

Anyhow, the following code implements the BNF:

\begin{verbatim}
{-------------------------------------------------------}
{ Parse and Translate a Math Factor }

procedure Expression; Forward;

procedure Factor;
begin
   if Look = '(' then begin
      Match('(');
      Expression;
      Match(')');
      end
   else if IsAlpha(Look) then
      LoadVar(GetName)
   else
      LoadConst(GetNum);
end;

{------------------------------------------------------}
{ Parse and Translate a Negative Factor }

procedure NegFactor;
begin
   Match('-');
   if IsDigit(Look) then
      LoadConst(-GetNum)
   else begin
      Factor;
      Negate;
   end;
end;

{------------------------------------------------------}
{ Parse and Translate a Leading Factor }

procedure FirstFactor;
begin
   case Look of
     '+': begin
             Match('+');
             Factor;
          end;
     '-': NegFactor;
   else  Factor;
   end;
end;

{------------------------------------------------------}
{ Recognize and Translate a Multiply }

procedure Multiply;
begin
   Match('*');
   Factor;
   PopMul;
end;

{-------------------------------------------------------------}
{ Recognize and Translate a Divide }

procedure Divide;
begin
   Match('/');
   Factor;
   PopDiv;
end;

{-------------------------------------------------------}
{ Common Code Used by Term and FirstTerm }

procedure Term1;
begin
   while IsMulop(Look) do begin
      Push;
      case Look of
       '*': Multiply;
       '/': Divide;
      end;
   end;
end;

{-------------------------------------------------------}
{ Parse and Translate a Math Term }

procedure Term;
begin
   Factor;
   Term1;
end;

{-------------------------------------------------------}
{ Parse and Translate a Leading Term }

procedure FirstTerm;
begin
   FirstFactor;
   Term1;
end;

{------------------------------------------------------}
{ Recognize and Translate an Add }

procedure Add;
begin
   Match('+');
   Term;
   PopAdd;
end;

{-------------------------------------------------------------}
{ Recognize and Translate a Subtract }

procedure Subtract;
begin
   Match('-');
   Term;
   PopSub;
end;

{-------------------------------------------------------}
{ Parse and Translate an Expression }

procedure Expression;
begin
   FirstTerm;
   while IsAddop(Look) do begin
      Push;
      case Look of
       '+': Add;
       '-': Subtract;
      end;
   end;
end;


{------------------------------------------------------}
{ Parse and Translate an Assignment Statement }

procedure Assignment;
var Name: char;
begin
   Name := GetName;
   Match('=');
   Expression;
   Store(Name);
end;
{------------------------------------------------------}
\end{verbatim}

OK, if you've  got  all  this  code inserted, then compile it and check  it out. You should  be  seeing  reasonable-looking  code, representing a complete program that will  assemble  and execute. We have a compiler!

\section{Booleans}

The next step should also  be  familiar  to  you. We  must add Boolean  expressions  and relational operations. Again, since we've already dealt with them more than once, I  won't elaborate much on them, except  where  they  are  different from what we've done before. Again, we won't just copy from other  files because I've changed a few things just a bit. Most  of  the changes just involve encapsulating the machine-dependent parts as  we  did for the   arithmetic  operations. I've  also  modified   procedure {\tt NotFactor}  somewhat, to  parallel  the structure of FirstFactor. Finally, I  corrected  an  error  in  the  object code  for  the relational operators:  The Scc instruction I used  only  sets the low 8 bits of D0. We want all 16 bits set for a logical true, so I've added an instruction to sign-extend the low byte.

To begin, we're going to need some more recognizers:

\begin{verbatim}
{------------------------------------------------------}
{ Recognize a Boolean Orop }

function IsOrop(c: char): boolean;
begin
   IsOrop := c in ['|', '~'];
end;

{------------------------------------------------------}
{ Recognize a Relop }

function IsRelop(c: char): boolean;
begin
   IsRelop := c in ['=', '#', '<', '>'];
end;
{------------------------------------------------------}

Also, we're going to need some more code generation routines:

{-------------------------------------------------------}
{ Complement the Primary Register }

procedure NotIt;
begin
   EmitLn('NOT D0');
end;
{-------------------------------------------------------}
.
.
.
{-------------------------------------------------------}
{ AND Top of Stack with Primary }

procedure PopAnd;
begin
   EmitLn('AND (SP)+,D0');
end;

{-------------------------------------------------------}
{ OR Top of Stack with Primary }

procedure PopOr;
begin
   EmitLn('OR (SP)+,D0');
end;

{-------------------------------------------------------}
{ XOR Top of Stack with Primary }

procedure PopXor;
begin
   EmitLn('EOR (SP)+,D0');
end;

{-------------------------------------------------------}
{ Compare Top of Stack with Primary }

procedure PopCompare;
begin
   EmitLn('CMP (SP)+,D0');
end;

{-------------------------------------------------------}
{ Set D0 If Compare was = }

procedure SetEqual;
begin
   EmitLn('SEQ D0');
   EmitLn('EXT D0');
end;

{-------------------------------------------------------}
{ Set D0 If Compare was != }

procedure SetNEqual;
begin
   EmitLn('SNE D0');
   EmitLn('EXT D0');
end;

{-------------------------------------------------------}
{ Set D0 If Compare was > }

procedure SetGreater;
begin
   EmitLn('SLT D0');
   EmitLn('EXT D0');
end;

{-------------------------------------------------------}
{ Set D0 If Compare was < }

procedure SetLess;
begin
   EmitLn('SGT D0');
   EmitLn('EXT D0');
end;
{-------------------------------------------------------}
\end{verbatim}

All of this  gives us the tools we need. The BNF for the Boolean expressions is:

\begin{verbatim}
     <bool-expr>   ::= <bool-term> ( <orop> <bool-term> )*
     <bool-term>   ::= <not-factor> ( <andop> <not-factor> )*
     <not-factor>  ::= [ '!' ] <relation>
     <relation>    ::= <expression> [ <relop> <expression> ]
\end{verbatim}

Sharp-eyed readers might  note  that this syntax does not include the non-terminal  "bool-factor" used in earlier versions. It was needed then because I also allowed for the Boolean constants TRUE and FALSE. But  remember  that  in TINY there is no distinction made between Boolean and arithmetic  types ... they can be freely intermixed. So there is really no  need  for  these  predefined values ... we can just use -1 and 0, respectively.

In C terminology, we could always use the defines:

\begin{verbatim}
     #define TRUE   -1
     #define FALSE   0
\end{verbatim}

(That is, if TINY had a  preprocessor.)   Later on, when we allow for  declarations  of  constants, these  two   values   will  be predefined by the language.

The reason that I'm harping on this is that  I've  already  tried the alternative, which is to  include TRUE and FALSE as keywords. The problem with that approach is that it  then  requires lexical scanning for EVERY variable name  in every expression. If you'll recall, I pointed out in Installment VII  that  this  slows  the compiler  down considerably. As long as  keywords  can't  be  in expressions, we need to do the scanning only at the  beginning of every  new  statement  ... quite  an improvement. So using  the syntax above not only simplifies the parsing, but  speeds  up the scanning as well.

OK, given that we're  all  satisfied  with  the syntax above, the corresponding code is shown below:

\begin{verbatim}
{-------------------------------------------------------}
{ Recognize and Translate a Relational "Equals" }

procedure Equals;
begin
   Match('=');
   Expression;
   PopCompare;
   SetEqual;
end;

{-------------------------------------------------------}
{ Recognize and Translate a Relational "Not Equals" }

procedure NotEquals;
begin
   Match('#');
   Expression;
   PopCompare;
   SetNEqual;
end;


{-------------------------------------------------------}
{ Recognize and Translate a Relational "Less Than" }

procedure Less;
begin
   Match('<');
   Expression;
   PopCompare;
   SetLess;
end;


{-------------------------------------------------------}
{ Recognize and Translate a Relational "Greater Than" }

procedure Greater;
begin
   Match('>');
   Expression;
   PopCompare;
   SetGreater;
end;


{-------------------------------------------------------}
{ Parse and Translate a Relation }

procedure Relation;
begin
   Expression;
   if IsRelop(Look) then begin
      Push;
      case Look of
       '=': Equals;
       '#': NotEquals;
       '<': Less;
       '>': Greater;
      end;
   end;
end;

{-------------------------------------------------------}
{ Parse and Translate a Boolean Factor with Leading NOT }

procedure NotFactor;
begin
   if Look = '!' then begin
      Match('!');
      Relation;
      NotIt;
      end
   else
      Relation;
end;

{-------------------------------------------------------}
{ Parse and Translate a Boolean Term }

procedure BoolTerm;
begin
   NotFactor;
   while Look = '&' do begin
      Push;
      Match('&');
      NotFactor;
      PopAnd;
   end;
end;

{------------------------------------------------------}
{ Recognize and Translate a Boolean OR }

procedure BoolOr;
begin
   Match('|');
   BoolTerm;
   PopOr;
end;

{------------------------------------------------------}
{ Recognize and Translate an Exclusive Or }

procedure BoolXor;
begin
   Match('~');
   BoolTerm;
   PopXor;
end;

{-------------------------------------------------------}
{ Parse and Translate a Boolean Expression }

procedure BoolExpression;
begin
   BoolTerm;
   while IsOrOp(Look) do begin
      Push;
      case Look of
       '|': BoolOr;
       '~': BoolXor;
      end;
   end;
end;
{------------------------------------------------------}
\end{verbatim}

To tie it all together, don't forget to change the  references to Expression in  procedures Factor and Assignment so that they call BoolExpression instead.

OK, if  you've  got  all  that typed in, compile it and give it a whirl. First, make  sure  you  can  still parse  an  ordinary arithmetic expression. Then, try a Boolean one. Finally, make sure  that you can assign the results of  relations. Try, for example:

\begin{verbatim}
     pvx,y,zbx=z>ye.
\end{verbatim}

which stands for:

\begin{verbatim}
     PROGRAM
     VAR X,Y,Z
     BEGIN
     X = Z > Y
     END.
\end{verbatim}

See how this assigns a Boolean value to X?

\section{Control Structures}

We're almost home. With  Boolean  expressions  in place, it's a simple  matter  to  add control structures. For TINY, we'll only allow two kinds of them, the IF and the WHILE:

{\small
\begin{verbatim}
     <if>     ::= IF <bool-expression> <block> [ ELSE <block>] ENDIF
     <while>  ::= WHILE <bool-expression> <block> ENDWHILE
\end{verbatim} }

Once  again, let  me  spell  out the decisions implicit in  this syntax, which departs strongly from that of C or Pascal. In both of those languages, the "body" of an IF or WHILE is regarded as a single  statement. If you intend to use a block of more than one statement, you have to build a compound statement using BEGIN-END (in Pascal) or  '{}' (in C). In TINY (and KISS) there is no such thing as a compound statement  ... single or multiple they're all just blocks to these languages.

In KISS, all the control structures will have explicit and unique keywords  bracketing  the  statement block, so there  can  be  no confusion as to where things begin  and  end. This is the modern approach, used in such respected languages as Ada  and  Modula 2, and it completely eliminates the problem of the "dangling else."

Note  that I could have chosen to use the same keyword END to end all  the constructs, as is done in Pascal. (The closing `\verb|}|' in C serves the same purpose.)  But this has always led  to confusion, which is why Pascal programmers tend to write things like

\begin{verbatim}
     end { loop }
or         end { if }
\end{verbatim}

As I explained in Section 5, using  unique terminal keywords does increase  the  size  of the keyword list and therefore slows down the  scanning, but in this case it seems a small price to pay for the added insurance. Better  to find the errors at compile time rather than run time.

One last thought:  The two constructs above each  have  the  non- terminals

\begin{verbatim}
      <bool-expression> and <block>
\end{verbatim}

juxtaposed with no separating keyword. In Pascal we would expect the keywords THEN and DO in these locations.

I have no problem with leaving out these keywords, and the parser has no trouble either, ON CONDITION that we make no errors in the bool-expression part. On  the  other hand, if we were to include these extra keywords we would get yet one more level of insurance at very little  cost, and  I  have no problem with that, either. Use your best judgment as to which way to go.

OK, with that bit of explanation let's proceed. As  usual, we're going to need some new  code generation routines. These generate the code for conditional and unconditional branches:

\begin{verbatim}
{-------------------------------------------------------}
{ Branch Unconditional  }

procedure Branch(L: string);
begin
   EmitLn('BRA ' + L);
end;

{-------------------------------------------------------}
{ Branch False }

procedure BranchFalse(L: string);
begin
   EmitLn('TST D0');
   EmitLn('BEQ ' + L);
end;
{------------------------------------------------------}
\end{verbatim}

Except for the encapsulation of  the code generation, the code to parse the control constructs is the same as you've seen before:

\begin{verbatim}
{-------------------------------------------------------}
{ Recognize and Translate an IF Construct }

procedure Block; Forward;

procedure DoIf;
var L1, L2: string;
begin
   Match('i');
   BoolExpression;
   L1 := NewLabel;
   L2 := L1;
   BranchFalse(L1);
   Block;
   if Look = 'l' then begin
      Match('l');
      L2 := NewLabel;
      Branch(L2);
      PostLabel(L1);
      Block;
   end;
   PostLabel(L2);
   Match('e');
end;

{------------------------------------------------------}
{ Parse and Translate a WHILE Statement }

procedure DoWhile;
var L1, L2: string;
begin
   Match('w');
   L1 := NewLabel;
   L2 := NewLabel;
   PostLabel(L1);
   BoolExpression;
   BranchFalse(L2);
   Block;
   Match('e');
   Branch(L1);
   PostLabel(L2);
end;
{------------------------------------------------------}
\end{verbatim}

To tie everything  together, we need only modify procedure Block to recognize the "keywords" for the  IF  and WHILE. As usual, we expand the definition of a block like so:

\begin{verbatim}
     <block> ::= ( <statement> )*

where

     <statement> ::= <if> | <while> | <assignment>
\end{verbatim}

The corresponding code is:

\begin{verbatim}
{------------------------------------------------------}
{ Parse and Translate a Block of Statements }

procedure Block;
begin
   while not(Look in ['e', 'l']) do begin
      case Look of
       'i': DoIf;
       'w': DoWhile;
      else Assignment;
      end;
   end;
end;
{------------------------------------------------------}
\end{verbatim}

OK, add the routines I've given, compile and  test  them. You should be able to parse the single-character versions  of  any of the control constructs. It's looking pretty good!

As a matter  of  fact, except for the single-character limitation we've got a virtually complete version of TINY. I call  it, with tongue planted firmly in cheek, TINY Version 0.1.

\section{Lexical Scanning}

Of course, you know what's next:  We have to convert  the program so that  it can deal with multi-character keywords, newlines, and whitespace. We have just gone through all  that  in Section 7. We'll use the distributed scanner  technique that I showed you in that  installment. The  actual  implementation  is   a  little different because the way I'm handling newlines is different.

To begin with, let's simply  allow for whitespace. This involves only adding calls to SkipWhite at the end of the  three routines, GetName, GetNum, and Match. A call to SkipWhite in Init primes the pump in case there are leading spaces.

Next, we need to deal with  newlines. This is really a two-step process, since  the  treatment  of  the  newlines  with  single- character tokens is different from that for multi-character ones. We can eliminate some work by doing both  steps  at  once, but I feel safer taking things one step at a time.

Insert the new procedure:

\begin{verbatim}
{------------------------------------------------------}
{ Skip Over an End-of-Line }

procedure NewLine;
begin
   while Look = CR do begin
      GetChar;
      if Look = LF then GetChar;
      SkipWhite;
   end;
end;
{------------------------------------------------------}
\end{verbatim}

Note that  we  have  seen  this  procedure  before in the form of Procedure Fin. I've changed the name since this  new  one  seems more descriptive of the actual function. I've  also  changed the code  to  allow  for multiple newlines and lines with nothing but white space.

The next step is to insert calls to NewLine wherever we  decide a newline is permissible. As I've pointed out before, this  can be very different in different languages. In TINY, I've decided to allow them virtually anywhere. This means that we need  calls to NewLine at the BEGINNING (not the end, as with SkipWhite)  of the procedures GetName, GetNum, and Match.

For procedures that have while loops, such as TopDecl, we  need a call  to NewLine at the beginning of the  procedure  AND  at  the bottom  of  each  loop. That way, we can be assured that NewLine has just been called at the beginning of each  pass  through  the loop.

If you've got all this done, try the program out and  verify that it will indeed handle white space and newlines.

If it does, then we're  ready to deal with multi-character tokens and keywords. To begin, add the additional declarations (copied almost verbatim from Section 7):

\begin{verbatim}
{------------------------------------------------------}
{ Type Declarations }

type Symbol = string[8];

     SymTab = array[1..1000] of Symbol;

     TabPtr = ^SymTab;


{------------------------------------------------------}
{ Variable Declarations }

var Look : char;             { Lookahead Character }
    Token: char;             { Encoded Token       }
    Value: string[16];       { Unencoded Token     }

    ST: Array['A'..'Z'] of char;

{------------------------------------------------------}
{ Definition of Keywords and Token Types }

const NKW =   9;
      NKW1 = 10;

const KWlist: array[1..NKW] of Symbol =
              ('IF', 'ELSE', 'ENDIF', 'WHILE', 'ENDWHILE',
               'VAR', 'BEGIN', 'END', 'PROGRAM');

const KWcode: string[NKW1] = 'xilewevbep';
{------------------------------------------------------}

Next, add the three procedures, also from Section 7:

{------------------------------------------------------}
{ Table Lookup }

function Lookup(T: TabPtr; s: string; n: integer): integer;
var i: integer;
    found: Boolean;
begin
   found := false;
   i := n;
   while (i > 0) and not found do
      if s = T^[i] then
         found := true
      else
         dec(i);
   Lookup := i;
end;
{------------------------------------------------------}
.
.
{------------------------------------------------------}
{ Get an Identifier and Scan it for Keywords }

procedure Scan;
begin
   GetName;
   Token := KWcode[Lookup(Addr(KWlist), Value, NKW) + 1];
end;
{------------------------------------------------------}
.
.
{------------------------------------------------------}
{ Match a Specific Input String }

procedure MatchString(x: string);
begin
   if Value <> x then Expected('''' + x + '''');
end;
{------------------------------------------------------}
\end{verbatim}

Now, we have to make a  fairly  large number of subtle changes to the remaining procedures. First, we  must  change  the function GetName to a procedure, again as we did in Section 7:


\begin{verbatim}
{------------------------------------------------------}
{ Get an Identifier }

procedure GetName;
begin
   NewLine;
   if not IsAlpha(Look) then Expected('Name');
   Value := '';
   while IsAlNum(Look) do begin
      Value := Value + UpCase(Look);
      GetChar;
   end;
   SkipWhite;
end;
{------------------------------------------------------}
\end{verbatim}

Note that this procedure leaves its result in  the  global string Value.

Next, we have to change every reference to GetName to reflect its new form. These occur in Factor, Assignment, and Decl:

\begin{verbatim}
{-------------------------------------------------------}
{ Parse and Translate a Math Factor }

procedure BoolExpression; Forward;

procedure Factor;
begin
   if Look = '(' then begin
      Match('(');
      BoolExpression;
      Match(')');
      end
   else if IsAlpha(Look) then begin
      GetName;
      LoadVar(Value[1]);
      end
   else
      LoadConst(GetNum);
end;
{------------------------------------------------------}
.
.
{------------------------------------------------------}
{ Parse and Translate an Assignment Statement }

procedure Assignment;
var Name: char;
begin
   Name := Value[1];
   Match('=');
   BoolExpression;
   Store(Name);
end;
{-------------------------------------------------------}
.
.
{------------------------------------------------------}
{ Parse and Translate a Data Declaration }

procedure Decl;
begin
   GetName;
   Alloc(Value[1]);
   while Look = ',' do begin
      Match(',');
      GetName;
      Alloc(Value[1]);
   end;
end;
{------------------------------------------------------}
\end{verbatim}

(Note that we're still  only  allowing  single-character variable names, so we take the easy way out here and simply use the first character of the string.)

Finally, we must make the changes to use Token instead of Look as the  test  character  and to call Scan at the appropriate places. Mostly, this  involves  deleting  calls  to  Match, occasionally replacing calls to  Match  by calls to MatchString, and Replacing calls  to  NewLine  by  calls  to  Scan. Here are the affected routines:

\begin{verbatim}
{-------------------------------------------------------}
{ Recognize and Translate an IF Construct }

procedure Block; Forward;


procedure DoIf;
var L1, L2: string;
begin
   BoolExpression;
   L1 := NewLabel;
   L2 := L1;
   BranchFalse(L1);
   Block;
   if Token = 'l' then begin
      L2 := NewLabel;
      Branch(L2);
      PostLabel(L1);
      Block;
   end;
   PostLabel(L2);
   MatchString('ENDIF');
end;


{------------------------------------------------------}
{ Parse and Translate a WHILE Statement }

procedure DoWhile;
var L1, L2: string;
begin
   L1 := NewLabel;
   L2 := NewLabel;
   PostLabel(L1);
   BoolExpression;
   BranchFalse(L2);
   Block;
   MatchString('ENDWHILE');
   Branch(L1);
   PostLabel(L2);
end;


{------------------------------------------------------}
{ Parse and Translate a Block of Statements }

procedure Block;
begin
   Scan;
   while not(Token in ['e', 'l']) do begin
      case Token of
       'i': DoIf;
       'w': DoWhile;
      else Assignment;
      end;
      Scan;
   end;
end;


{------------------------------------------------------}
{ Parse and Translate Global Declarations }

procedure TopDecls;
begin
   Scan;
   while Token <> 'b' do begin
      case Token of
        'v': Decl;
      else Abort('Unrecognized Keyword ' + Value);
      end;
      Scan;
   end;
end;


{------------------------------------------------------}
{ Parse and Translate a Main Program }

procedure Main;
begin
   MatchString('BEGIN');
   Prolog;
   Block;
   MatchString('END');
   Epilog;
end;

{------------------------------------------------------}
{  Parse and Translate a Program }

procedure Prog;
begin
   MatchString('PROGRAM');
   Header;
   TopDecls;
   Main;
   Match('.');
end;

{------------------------------------------------------}
{ Initialize }

procedure Init;
var i: char;
begin
   for i := 'A' to 'Z' do
      ST[i] := ' ';
   GetChar;
   Scan;
end;
{------------------------------------------------------}
\end{verbatim}

That should do  it. If  all  the changes got in correctly, you should now be parsing programs that look like programs. (If you didn't  make  it  through all the  changes, don't  despair. A complete listing of the final form is given later.)

Did it work?  If so, then we're just about home. In fact, with a few minor  exceptions we've already got a compiler that's usable. There are still a few areas that need improvement.

\section{Multi-Character Variable Names}

One of those is  the  restriction  that  we still have, requiring single-character variable names. Now that we can handle multi- character keywords, this one  begins  to  look  very much like an arbitrary  and  unnecessary  limitation. And  indeed   it  is. Basically, its only virtue is  that it permits a trivially simple implementation  of  the   symbol   table. But  that's  just  a convenience to the compiler writers, and needs to be eliminated.

We've done this step before. This time, as usual, I'm doing it a little differently. I think  the approach used here keeps things just about as simple as possible.

The natural  way  to  implement  a  symbol  table in Pascal is by declaring a record type, and making the symbol table an  array of such records. Here, though, we don't really need  a  type  field yet  (there is only one kind of entry allowed so far), so we only need an array of symbols. This has the advantage that we can use the existing procedure Lookup to  search the symbol table as well as the  keyword  list. As it turns out, even when we need more fields we can still use the same approach, simply by  storing the other fields in separate arrays.

OK, here are the changes that  need  to  be made. First, add the new typed constant:

\begin{verbatim}
      NEntry: integer = 0;
\end{verbatim}

Then change the definition of the symbol table as follows:

\begin{verbatim}
const MaxEntry = 100;
var ST   : array[1..MaxEntry] of Symbol;
\end{verbatim}

(Note that ST is  NOT  declared as a SymTab. That declaration is a phony one to get Lookup to work. A SymTab  would  take  up too much RAM space, and so one is never actually allocated.)

Next, we need to replace InTable:

\begin{verbatim}
{------------------------------------------------------}
{ Look for Symbol in Table }

function InTable(n: Symbol): Boolean;
begin
   InTable := Lookup(@ST, n, MaxEntry) <> 0;
end;
{------------------------------------------------------}
\end{verbatim}

We also need a new procedure, AddEntry, that adds a new  entry to the table:

\begin{verbatim}
{------------------------------------------------------}
{ Add a New Entry to Symbol Table }

procedure AddEntry(N: Symbol; T: char);
begin
   if InTable(N) then Abort('Duplicate Identifier ' + N);
   if NEntry = MaxEntry then Abort('Symbol Table Full');
   Inc(NEntry);
   ST[NEntry] := N;
   SType[NEntry] := T;
end;
{------------------------------------------------------}

This procedure is called by Alloc:

{------------------------------------------------------}
{ Allocate Storage for a Variable }

procedure Alloc(N: Symbol);
begin
   if InTable(N) then Abort('Duplicate Variable Name ' + N);
   AddEntry(N, 'v');
.
.
.
{------------------------------------------------------}
\end{verbatim}

Finally, we must change all the routines that currently treat the variable name as a single character. These include   LoadVar and Store (just change the  type  from  char  to string), and Factor, Assignment, and Decl (just change Value[1] to Value).

One  last  thing:  change  procedure  Init to clear the array  as shown:

\begin{verbatim}
{------------------------------------------------------}
{ Initialize }

procedure Init;
var i: integer;
begin
   for i := 1 to MaxEntry do begin
      ST[i] := '';
      SType[i] := ' ';
   end;
   GetChar;
   Scan;
end;
{------------------------------------------------------}
\end{verbatim}

That should do it. Try it out and verify  that  you can, indeed, use multi-character variable names.

\section{More Relops}

We still have one remaining single-character restriction: the one on relops. Some of the relops are indeed single  characters, but others  require two. These are `\verb|<=|' and `\verb|>=|'. I also prefer the Pascal `\verb|<>|' for "not equals,"  instead of `\#'.

If you'll recall, in Section 7 I pointed out that the conventional way  to  deal  with  relops  is  to  include them in the list  of keywords, and let the  lexical  scanner  find  them. But, again, this requires scanning throughout the expression parsing process, whereas so far we've been able to limit the use of the scanner to the beginning of a statement.

I mentioned then that we can still get away with this, since the multi-character relops are so few  and so limited in their usage. It's easy to just treat them as special cases and handle  them in an ad hoc manner.

The changes required affect only the code generation routines and procedures Relation and friends. First, we're going to need two more code generation routines:

\begin{verbatim}
{-------------------------------------------------------}
{ Set D0 If Compare was <= }

procedure SetLessOrEqual;
begin
   EmitLn('SGE D0');
   EmitLn('EXT D0');
end;

{-------------------------------------------------------}
{ Set D0 If Compare was >= }

procedure SetGreaterOrEqual;
begin
   EmitLn('SLE D0');
   EmitLn('EXT D0');
end;
{-------------------------------------------------------}

Then, modify the relation parsing routines as shown below:

{-------------------------------------------------------}
{ Recognize and Translate a Relational "Less Than or Equal" }

procedure LessOrEqual;
begin
   Match('=');
   Expression;
   PopCompare;
   SetLessOrEqual;
end;


{-------------------------------------------------------}
{ Recognize and Translate a Relational "Not Equals" }

procedure NotEqual;
begin
   Match('>');
   Expression;
   PopCompare;
   SetNEqual;
end;

{-------------------------------------------------------}
{ Recognize and Translate a Relational "Less Than" }

procedure Less;
begin
   Match('<');
   case Look of
     '=': LessOrEqual;
     '>': NotEqual;
   else begin
           Expression;
           PopCompare;
           SetLess;
        end;
   end;
end;

{-------------------------------------------------------}
{ Recognize and Translate a Relational "Greater Than" }

procedure Greater;
begin
   Match('>');
   if Look = '=' then begin
      Match('=');
      Expression;
      PopCompare;
      SetGreaterOrEqual;
      end
   else begin
      Expression;
      PopCompare;
      SetGreater;
   end;
end;
{-------------------------------------------------------}
\end{verbatim}

That's all it takes. Now  you  can  process all the relops. Try it.

\section{Input/Output}

We  now  have  a complete, working language, except for one minor embarrassment: we have no way to get data in or out. We need some I/O.

Now, the convention these days, established in C and continued in Ada and Modula 2, is to leave I/O statements out of  the language itself, and  just  include them in the subroutine library. That would  be  fine, except that so far  we  have  no  provision  for subroutines. Anyhow, with this approach you run into the problem of variable-length argument lists. In Pascal, the I/O statements are built into the language because they are the  only  ones  for which  the  argument  list can have a variable number of entries. In C, we settle for kludges like scanf and printf, and  must pass the argument count to the called procedure. In Ada and  Modula 2 we must use the  awkward  (and SLOW!) approach of a separate call for each argument.

So I think I prefer the  Pascal  approach of building the I/O in, even though we don't need to.

As  usual, for  this we need some more code generation routines. These turn out  to be the easiest of all, because all we do is to call library procedures to do the work:

\begin{verbatim}
{-------------------------------------------------------}
{ Read Variable to Primary Register }

procedure ReadVar;
begin
   EmitLn('BSR READ');
   Store(Value);
end;

{-------------------------------------------------------}
{ Write Variable from Primary Register }

procedure WriteVar;
begin
   EmitLn('BSR WRITE');
end;
{------------------------------------------------------}
\end{verbatim}

The idea is that READ loads the value from input  to  the D0, and WRITE outputs it from there.

These two procedures represent  our  first  encounter with a need for library procedures ... the components of a  Run  Time Library (RTL). Of  course, someone (namely  us)  has  to  write  these routines, but they're not  part  of the compiler itself. I won't even bother  showing the routines here, since these are obviously very much OS-dependent. I   WILL   simply  say that for SK*DOS, they  are  particularly  simple ... almost trivial. One reason I won't show them here is that  you  can add all kinds of fanciness to the things, for  example  by prompting in READ for the inputs, and by giving the user a chance to reenter a bad input.

But that is really separate from compiler design, so for now I'll just assume that a library call TINYLIB.LIB exists. Since we now need  it  loaded, we need to add a statement to  include  it  in procedure Header:

\begin{verbatim}
{------------------------------------------------------}
{ Write Header Info }

procedure Header;
begin

   WriteLn('WARMST', TAB, 'EQU $A01E');
   EmitLn('LIB TINYLIB');
end;
{------------------------------------------------------}
\end{verbatim}

That takes care of that part. Now, we also need to recognize the
read  and  write  commands. We can do this by  adding  two  more
keywords to our list:

\begin{verbatim}
{------------------------------------------------------}
{ Definition of Keywords and Token Types }

const NKW =   11;
      NKW1 = 12;

const KWlist: array[1..NKW] of Symbol =
              ('IF', 'ELSE', 'ENDIF', 'WHILE', 'ENDWHILE',
               'READ', 'WRITE', 'VAR', 'BEGIN', 'END',
'PROGRAM');

const KWcode: string[NKW1] = 'xileweRWvbep';
{------------------------------------------------------}
\end{verbatim}

(Note how I'm using upper case codes here to avoid  conflict with the 'w' of WHILE.)
Next, we need procedures for processing the  read/write statement and its argument list:

\begin{verbatim}
{------------------------------------------------------}
{ Process a Read Statement }
procedure DoRead;
begin
   Match('(');
   GetName;
   ReadVar;
   while Look = ',' do begin
      Match(',');
      GetName;
      ReadVar;
   end;
   Match(')');
end;

{------------------------------------------------------}
{ Process a Write Statement }

procedure DoWrite;
begin
   Match('(');
   Expression;
   WriteVar;
   while Look = ',' do begin
      Match(',');
      Expression;
      WriteVar;
   end;
   Match(')');
end;
{------------------------------------------------------}
\end{verbatim}

Finally, we  must  expand  procedure  Block  to  handle the  new statement types:

\begin{verbatim}
{------------------------------------------------------}
{ Parse and Translate a Block of Statements }

procedure Block;
begin
   Scan;
   while not(Token in ['e', 'l']) do begin
      case Token of
       'i': DoIf;
       'w': DoWhile;
       'R': DoRead;
       'W': DoWrite;
      else Assignment;
      end;
      Scan;
   end;
end;
{------------------------------------------------------}
\end{verbatim}

That's all there is to it. {\bfseries Now} we have a language!

\section{Conclusion}

At this point we have TINY completely defined. It's not much ... actually a toy  compiler. TINY  has  only one data type and no subroutines  ... but it's a complete, usable  language. While you're not likely to be able to write another compiler in  it, or do anything else very seriously, you could write programs to read some input, perform calculations, and  output  the results. Not too bad for a toy.

Most importantly, we have a firm base upon which to build further extensions. I know you'll be glad to hear this: this is the last time  I'll  start  over in building a parser ... from  now  on  I intend to just add features to  TINY  until it becomes KISS. Oh, there'll be other times we will  need  to try things out with new copies  of  the  Cradle, but once we've found out how to do those things they'll be incorporated into TINY.

What  will  those  features  be?    Well, for starters  we  need subroutines and functions. Then  we  need to be able to handle different types, including arrays, strings, and other structures. Then we need to deal with the idea of pointers. All this will be upcoming in future installments.

See you then.

For references purposes, the complete listing of TINY Version 1.0 is shown below:

\begin{verbatim}
{------------------------------------------------------}
program Tiny10;

{------------------------------------------------------}
{ Constant Declarations }

const TAB = ^I;
      CR  = ^M;
      LF  = ^J;

      LCount: integer = 0;
      NEntry: integer = 0;

{------------------------------------------------------}
{ Type Declarations }

type Symbol = string[8];

     SymTab = array[1..1000] of Symbol;
     TabPtr = ^SymTab;

{------------------------------------------------------}
{ Variable Declarations }

var Look : char;             { Lookahead Character }
    Token: char;             { Encoded Token       }
    Value: string[16];       { Unencoded Token     }


const MaxEntry = 100;

var ST   : array[1..MaxEntry] of Symbol;
    SType: array[1..MaxEntry] of char;

{------------------------------------------------------}
{ Definition of Keywords and Token Types }

const NKW =   11;
      NKW1 = 12;

const KWlist: array[1..NKW] of Symbol =
              ('IF', 'ELSE', 'ENDIF', 'WHILE', 'ENDWHILE',
               'READ', 'WRITE', 'VAR', 'BEGIN', 'END',
'PROGRAM');

const KWcode: string[NKW1] = 'xileweRWvbep';


{------------------------------------------------------}
{ Read New Character From Input Stream }

procedure GetChar;
begin
   Read(Look);
end;

{------------------------------------------------------}
{ Report an Error }

procedure Error(s: string);
begin
   WriteLn;
   WriteLn(^G, 'Error: ', s, '.');
end;

{------------------------------------------------------}
{ Report Error and Halt }

procedure Abort(s: string);
begin
   Error(s);
   Halt;
end;

{------------------------------------------------------}
{ Report What Was Expected }

procedure Expected(s: string);
begin
   Abort(s + ' Expected');
end;

{------------------------------------------------------}
{ Report an Undefined Identifier }

procedure Undefined(n: string);
begin
   Abort('Undefined Identifier ' + n);
end;

{------------------------------------------------------}
{ Recognize an Alpha Character }

function IsAlpha(c: char): boolean;
begin
   IsAlpha := UpCase(c) in ['A'..'Z'];
end;

{------------------------------------------------------}
{ Recognize a Decimal Digit }

function IsDigit(c: char): boolean;
begin
   IsDigit := c in ['0'..'9'];
end;


{------------------------------------------------------}
{ Recognize an AlphaNumeric Character }

function IsAlNum(c: char): boolean;
begin
   IsAlNum := IsAlpha(c) or IsDigit(c);
end;

{------------------------------------------------------}
{ Recognize an Addop }

function IsAddop(c: char): boolean;
begin
   IsAddop := c in ['+', '-'];
end;


{------------------------------------------------------}
{ Recognize a Mulop }

function IsMulop(c: char): boolean;
begin
   IsMulop := c in ['*', '/'];
end;

{------------------------------------------------------}
{ Recognize a Boolean Orop }

function IsOrop(c: char): boolean;
begin
   IsOrop := c in ['|', '~'];
end;

{------------------------------------------------------}
{ Recognize a Relop }

function IsRelop(c: char): boolean;
begin
   IsRelop := c in ['=', '#', '<', '>'];
end;

{------------------------------------------------------}
{ Recognize White Space }

function IsWhite(c: char): boolean;
begin
   IsWhite := c in [' ', TAB];
end;

{------------------------------------------------------}
{ Skip Over Leading White Space }

procedure SkipWhite;
begin
   while IsWhite(Look) do
      GetChar;
end;

{------------------------------------------------------}
{ Skip Over an End-of-Line }

procedure NewLine;
begin
   while Look = CR do begin
      GetChar;
      if Look = LF then GetChar;
      SkipWhite;
   end;
end;

{------------------------------------------------------}
{ Match a Specific Input Character }

procedure Match(x: char);
begin
   NewLine;
   if Look = x then GetChar
   else Expected('''' + x + '''');
   SkipWhite;
end;

{------------------------------------------------------}
{ Table Lookup }

function Lookup(T: TabPtr; s: string; n: integer): integer;
var i: integer;
    found: Boolean;
begin
   found := false;
   i := n;
   while (i > 0) and not found do
      if s = T^[i] then
         found := true
      else
         dec(i);
   Lookup := i;
end;


{------------------------------------------------------}
{ Locate a Symbol in Table }
{ Returns the index of the entry. Zero if not present. }

function Locate(N: Symbol): integer;
begin
   Locate := Lookup(@ST, n, MaxEntry);
end;

{------------------------------------------------------}
{ Look for Symbol in Table }

function InTable(n: Symbol): Boolean;
begin
   InTable := Lookup(@ST, n, MaxEntry) <> 0;
end;

{------------------------------------------------------}
{ Add a New Entry to Symbol Table }

procedure AddEntry(N: Symbol; T: char);
begin
   if InTable(N) then Abort('Duplicate Identifier ' + N);
   if NEntry = MaxEntry then Abort('Symbol Table Full');
   Inc(NEntry);
   ST[NEntry] := N;
   SType[NEntry] := T;
end;

{------------------------------------------------------}
{ Get an Identifier }

procedure GetName;
begin
   NewLine;
   if not IsAlpha(Look) then Expected('Name');
   Value := '';
   while IsAlNum(Look) do begin
      Value := Value + UpCase(Look);
      GetChar;
   end;
   SkipWhite;
end;

{------------------------------------------------------}
{ Get a Number }

function GetNum: integer;
var Val: integer;
begin
   NewLine;
   if not IsDigit(Look) then Expected('Integer');
   Val := 0;
   while IsDigit(Look) do begin
      Val := 10 * Val + Ord(Look) - Ord('0');
      GetChar;
   end;
   GetNum := Val;
   SkipWhite;
end;

{------------------------------------------------------}
{ Get an Identifier and Scan it for Keywords }

procedure Scan;
begin
   GetName;
   Token := KWcode[Lookup(Addr(KWlist), Value, NKW) + 1];
end;

{------------------------------------------------------}
{ Match a Specific Input String }

procedure MatchString(x: string);
begin
   if Value <> x then Expected('''' + x + '''');
end;

{------------------------------------------------------}
{ Output a String with Tab }

procedure Emit(s: string);
begin
   Write(TAB, s);
end;

{------------------------------------------------------}
{ Output a String with Tab and CRLF }

procedure EmitLn(s: string);
begin
   Emit(s);
   WriteLn;
end;

{------------------------------------------------------}
{ Generate a Unique Label }

function NewLabel: string;
var S: string;
begin
   Str(LCount, S);
   NewLabel := 'L' + S;
   Inc(LCount);
end;

{------------------------------------------------------}
{ Post a Label To Output }

procedure PostLabel(L: string);
begin
   WriteLn(L, ':');
end;

{-------------------------------------------------------}
{ Clear the Primary Register }

procedure Clear;
begin
   EmitLn('CLR D0');
end;

{-------------------------------------------------------}
{ Negate the Primary Register }

procedure Negate;
begin
   EmitLn('NEG D0');
end;

{-------------------------------------------------------}
{ Complement the Primary Register }

procedure NotIt;
begin
   EmitLn('NOT D0');
end;

{-------------------------------------------------------}
{ Load a Constant Value to Primary Register }

procedure LoadConst(n: integer);
begin
   Emit('MOVE #');
   WriteLn(n, ',D0');
end;

{-------------------------------------------------------}
{ Load a Variable to Primary Register }

procedure LoadVar(Name: string);
begin
   if not InTable(Name) then Undefined(Name);
   EmitLn('MOVE ' + Name + '(PC),D0');
end;

{-------------------------------------------------------}
{ Push Primary onto Stack }

procedure Push;
begin
   EmitLn('MOVE D0,-(SP)');
end;

{-------------------------------------------------------}
{ Add Top of Stack to Primary }

procedure PopAdd;
begin
   EmitLn('ADD (SP)+,D0');
end;

{-------------------------------------------------------}
{ Subtract Primary from Top of Stack }

procedure PopSub;
begin
   EmitLn('SUB (SP)+,D0');
   EmitLn('NEG D0');
end;

{-------------------------------------------------------}
{ Multiply Top of Stack by Primary }

procedure PopMul;
begin
   EmitLn('MULS (SP)+,D0');
end;

{-------------------------------------------------------}
{ Divide Top of Stack by Primary }

procedure PopDiv;
begin
   EmitLn('MOVE (SP)+,D7');
   EmitLn('EXT.L D7');
   EmitLn('DIVS D0,D7');
   EmitLn('MOVE D7,D0');
end;

{-------------------------------------------------------}
{ AND Top of Stack with Primary }

procedure PopAnd;
begin
   EmitLn('AND (SP)+,D0');
end;

{-------------------------------------------------------}
{ OR Top of Stack with Primary }

procedure PopOr;
begin
   EmitLn('OR (SP)+,D0');
end;

{-------------------------------------------------------}
{ XOR Top of Stack with Primary }

procedure PopXor;
begin
   EmitLn('EOR (SP)+,D0');
end;

{-------------------------------------------------------}
{ Compare Top of Stack with Primary }

procedure PopCompare;
begin
   EmitLn('CMP (SP)+,D0');
end;

{-------------------------------------------------------}
{ Set D0 If Compare was = }

procedure SetEqual;
begin
   EmitLn('SEQ D0');
   EmitLn('EXT D0');
end;

{-------------------------------------------------------}
{ Set D0 If Compare was != }

procedure SetNEqual;
begin
   EmitLn('SNE D0');
   EmitLn('EXT D0');
end;

{-------------------------------------------------------}
{ Set D0 If Compare was > }

procedure SetGreater;
begin
   EmitLn('SLT D0');
   EmitLn('EXT D0');
end;

{-------------------------------------------------------}
{ Set D0 If Compare was < }

procedure SetLess;
begin
   EmitLn('SGT D0');
   EmitLn('EXT D0');
end;

{-------------------------------------------------------}
{ Set D0 If Compare was <= }

procedure SetLessOrEqual;
begin
   EmitLn('SGE D0');
   EmitLn('EXT D0');
end;

{-------------------------------------------------------}
{ Set D0 If Compare was >= }

procedure SetGreaterOrEqual;
begin
   EmitLn('SLE D0');
   EmitLn('EXT D0');
end;

{-------------------------------------------------------}
{ Store Primary to Variable }

procedure Store(Name: string);
begin
   if not InTable(Name) then Undefined(Name);
   EmitLn('LEA ' + Name + '(PC),A0');
   EmitLn('MOVE D0,(A0)')
end;

{-------------------------------------------------------}
{ Branch Unconditional  }

procedure Branch(L: string);
begin
   EmitLn('BRA ' + L);
end;

{-------------------------------------------------------}
{ Branch False }

procedure BranchFalse(L: string);
begin
   EmitLn('TST D0');
   EmitLn('BEQ ' + L);
end;

{-------------------------------------------------------}
{ Read Variable to Primary Register }

procedure ReadVar;
begin
   EmitLn('BSR READ');
   Store(Value[1]);
end;

{ Write Variable from Primary Register }

procedure WriteVar;
begin
   EmitLn('BSR WRITE');
end;

{------------------------------------------------------}
{ Write Header Info }

procedure Header;
begin
   WriteLn('WARMST', TAB, 'EQU $A01E');
end;

{------------------------------------------------------}
{ Write the Prolog }

procedure Prolog;
begin
   PostLabel('MAIN');
end;

{------------------------------------------------------}
{ Write the Epilog }

procedure Epilog;
begin
   EmitLn('DC WARMST');
   EmitLn('END MAIN');
end;

{-------------------------------------------------------}
{ Parse and Translate a Math Factor }

procedure BoolExpression; Forward;

procedure Factor;
begin
   if Look = '(' then begin
      Match('(');
      BoolExpression;
      Match(')');
      end
   else if IsAlpha(Look) then begin
      GetName;
      LoadVar(Value);
      end
   else
      LoadConst(GetNum);
end;

{------------------------------------------------------}
{ Parse and Translate a Negative Factor }

procedure NegFactor;
begin
   Match('-');
   if IsDigit(Look) then
      LoadConst(-GetNum)
   else begin
      Factor;
      Negate;
   end;
end;

{------------------------------------------------------}
{ Parse and Translate a Leading Factor }

procedure FirstFactor;
begin
   case Look of
     '+': begin
             Match('+');
             Factor;
          end;
     '-': NegFactor;
   else  Factor;
   end;
end;

{------------------------------------------------------}
{ Recognize and Translate a Multiply }

procedure Multiply;
begin
   Match('*');
   Factor;
   PopMul;
end;

{-------------------------------------------------------------}
{ Recognize and Translate a Divide }

procedure Divide;
begin
   Match('/');
   Factor;
   PopDiv;
end;

{-------------------------------------------------------}
{ Common Code Used by Term and FirstTerm }

procedure Term1;
begin
   while IsMulop(Look) do begin
      Push;
      case Look of
       '*': Multiply;
       '/': Divide;
      end;
   end;
end;

{-------------------------------------------------------}
{ Parse and Translate a Math Term }

procedure Term;
begin
   Factor;
   Term1;
end;

{-------------------------------------------------------}
{ Parse and Translate a Leading Term }

procedure FirstTerm;
begin
   FirstFactor;
   Term1;
end;

{------------------------------------------------------}
{ Recognize and Translate an Add }

procedure Add;
begin
   Match('+');
   Term;
   PopAdd;
end;

{-------------------------------------------------------------}
{ Recognize and Translate a Subtract }

procedure Subtract;
begin
   Match('-');
   Term;
   PopSub;
end;

{-------------------------------------------------------}
{ Parse and Translate an Expression }

procedure Expression;
begin
   FirstTerm;
   while IsAddop(Look) do begin
      Push;
      case Look of
       '+': Add;
       '-': Subtract;
      end;
   end;
end;

{-------------------------------------------------------}
{ Recognize and Translate a Relational "Equals" }

procedure Equal;
begin
   Match('=');
   Expression;
   PopCompare;
   SetEqual;
end;

{-------------------------------------------------------}
{ Recognize and Translate a Relational "Less Than or Equal" }

procedure LessOrEqual;
begin
   Match('=');
   Expression;
   PopCompare;
   SetLessOrEqual;
end;

{-------------------------------------------------------}
{ Recognize and Translate a Relational "Not Equals" }

procedure NotEqual;
begin
   Match('>');
   Expression;
   PopCompare;
   SetNEqual;
end;

{-------------------------------------------------------}
{ Recognize and Translate a Relational "Less Than" }

procedure Less;
begin
   Match('<');
   case Look of
     '=': LessOrEqual;
     '>': NotEqual;
   else begin
           Expression;
           PopCompare;
           SetLess;
        end;
   end;
end;

{-------------------------------------------------------}
{ Recognize and Translate a Relational "Greater Than" }

procedure Greater;
begin
   Match('>');
   if Look = '=' then begin
      Match('=');
      Expression;
      PopCompare;
      SetGreaterOrEqual;
      end
   else begin
      Expression;
      PopCompare;
      SetGreater;
   end;
end;

{-------------------------------------------------------}
{ Parse and Translate a Relation }

procedure Relation;
begin
   Expression;
   if IsRelop(Look) then begin
      Push;
      case Look of
       '=': Equal;
       '<': Less;
       '>': Greater;
      end;
   end;
end;

{-------------------------------------------------------}
{ Parse and Translate a Boolean Factor with Leading NOT }

procedure NotFactor;
begin
   if Look = '!' then begin
      Match('!');
      Relation;
      NotIt;
      end
   else
      Relation;
end;

{-------------------------------------------------------}
{ Parse and Translate a Boolean Term }

procedure BoolTerm;
begin
   NotFactor;
   while Look = '&' do begin
      Push;
      Match('&');
      NotFactor;
      PopAnd;
   end;
end;

{------------------------------------------------------}
{ Recognize and Translate a Boolean OR }

procedure BoolOr;
begin
   Match('|');
   BoolTerm;
   PopOr;
end;

{------------------------------------------------------}
{ Recognize and Translate an Exclusive Or }

procedure BoolXor;
begin
   Match('~');
   BoolTerm;
   PopXor;
end;

{-------------------------------------------------------}
{ Parse and Translate a Boolean Expression }

procedure BoolExpression;
begin
   BoolTerm;
   while IsOrOp(Look) do begin
      Push;
      case Look of
       '|': BoolOr;
       '~': BoolXor;
      end;
   end;
end;

{------------------------------------------------------}
{ Parse and Translate an Assignment Statement }

procedure Assignment;
var Name: string;
begin
   Name := Value;
   Match('=');
   BoolExpression;
   Store(Name);
end;

{-------------------------------------------------------}
{ Recognize and Translate an IF Construct }

procedure Block; Forward;

procedure DoIf;
var L1, L2: string;
begin
   BoolExpression;
   L1 := NewLabel;
   L2 := L1;
   BranchFalse(L1);
   Block;
   if Token = 'l' then begin
      L2 := NewLabel;
      Branch(L2);
      PostLabel(L1);
      Block;
   end;
   PostLabel(L2);
   MatchString('ENDIF');
end;

{------------------------------------------------------}
{ Parse and Translate a WHILE Statement }

procedure DoWhile;
var L1, L2: string;
begin
   L1 := NewLabel;
   L2 := NewLabel;
   PostLabel(L1);
   BoolExpression;
   BranchFalse(L2);
   Block;
   MatchString('ENDWHILE');
   Branch(L1);
   PostLabel(L2);
end;

{------------------------------------------------------}
{ Process a Read Statement }

procedure DoRead;
begin
   Match('(');
   GetName;
   ReadVar;
   while Look = ',' do begin
      Match(',');
      GetName;
      ReadVar;
   end;
   Match(')');
end;

{------------------------------------------------------}
{ Process a Write Statement }

procedure DoWrite;
begin
   Match('(');
   Expression;
   WriteVar;
   while Look = ',' do begin
      Match(',');
      Expression;
      WriteVar;
   end;
   Match(')');
end;

{------------------------------------------------------}
{ Parse and Translate a Block of Statements }

procedure Block;
begin
   Scan;
   while not(Token in ['e', 'l']) do begin
      case Token of
       'i': DoIf;
       'w': DoWhile;
       'R': DoRead;
       'W': DoWrite;
      else Assignment;
      end;
      Scan;
   end;
end;

{------------------------------------------------------}
{ Allocate Storage for a Variable }

procedure Alloc(N: Symbol);
begin
   if InTable(N) then Abort('Duplicate Variable Name ' + N);
   AddEntry(N, 'v');
   Write(N, ':', TAB, 'DC ');
   if Look = '=' then begin
      Match('=');
      If Look = '-' then begin
         Write(Look);
         Match('-');
      end;
      WriteLn(GetNum);
      end
   else
      WriteLn('0');
end;

{------------------------------------------------------}
{ Parse and Translate a Data Declaration }

procedure Decl;
begin
   GetName;
   Alloc(Value);
   while Look = ',' do begin
      Match(',');
      GetName;
      Alloc(Value);
   end;
end;

{------------------------------------------------------}
{ Parse and Translate Global Declarations }

procedure TopDecls;
begin
   Scan;
   while Token <> 'b' do begin
      case Token of
        'v': Decl;
      else Abort('Unrecognized Keyword ' + Value);
      end;
      Scan;
   end;
end;

{------------------------------------------------------}
{ Parse and Translate a Main Program }

procedure Main;
begin
   MatchString('BEGIN');
   Prolog;
   Block;
   MatchString('END');
   Epilog;
end;

{------------------------------------------------------}
{  Parse and Translate a Program }

procedure Prog;
begin
   MatchString('PROGRAM');
   Header;
   TopDecls;
   Main;
   Match('.');
end;

{------------------------------------------------------}
{ Initialize }

procedure Init;
var i: integer;
begin
   for i := 1 to MaxEntry do begin
      ST[i] := '';
      SType[i] := ' ';
   end;
   GetChar;
   Scan;
end;

{------------------------------------------------------}
{ Main Program }

begin
   Init;
   Prog;
   if Look <> CR then Abort('Unexpected data after ''.''');
end.
{------------------------------------------------------}
\end{verbatim}
