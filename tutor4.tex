
\chapter{INTERPRETERS}

24th July 1988

\section{Introduction}

In the first three installments of this series, we've  looked at parsing and  compiling math expressions, and worked our way gradually and methodically from dealing  with  very  simple one-term, one-character ``expressions'' up through more general ones, finally arriving at a very complete parser that could parse and translate complete  assignment  statements, with  multi-character  tokens, embedded white space, and function calls. This  time, I'm going to walk you through the process one more time, only with the goal of interpreting rather than compiling object code.

Since this is a series on compilers, why should  we  bother  with interpreters?  Simply because I want you to see how the nature of the  parser changes as we change the goals. I also want to unify the concepts of the two types of translators, so that you can see not only the differences, but also the similarities.

Consider the assignment statement

\begin{verbatim}
               x = 2 * y + 3
\end{verbatim}

In a compiler, we want the target CPU to execute  this assignment at EXECUTION time. The translator itself doesn't  do  any arithmetic ... it only issues the object code that will cause  the CPU to do it when the code is executed. For  the  example above, the compiler would issue code to compute the expression and store the results in variable x.

For an interpreter, on  the  other  hand, no object code is generated. Instead, the arithmetic is computed immediately, as the parsing is going on. For the example, by the time parsing of the statement is complete, x will have a new value.

The approach we've been  taking  in  this  whole series is called ``syntax-driven translation''.  As you are aware by now, the structure of the  parser  is  very  closely  tied to the syntax of the productions we parse. We  have built Pascal procedures that recognize every language  construct. Associated with each of these constructs (and procedures) is  a  corresponding  ``action'', which does  whatever  makes  sense to do  once  a  construct  has  been recognized. In  our  compiler  so far, every  action  involves emitting object code, to be executed later at execution time. In an interpreter, every action  involves  something  to be done immediately.

What I'd like you to see here is that the  layout  ... the structure ... of  the  parser  doesn't  change. It's only the actions that change. So  if  you  can  write an interpreter for a given language, you can also write a compiler, and vice versa. Yet, as you  will  see, there  ARE  differences, and  significant ones. Because the actions are different, the  procedures  that  do the recognizing end up being written differently. Specifically, in the interpreter  the recognizing procedures end up being coded as FUNCTIONS that return numeric values to their callers. None of the parsing routines for our compiler did that.

Our compiler, in fact, is  what we might call a ``pure'' compiler. Each time a construct is recognized, the object  code  is emitted {\bfseries immediately}. (That's one reason the code is not very efficient.) The interpreter we'll be building  here is a pure interpreter, in the sense that there is  no  translation, such  as ``tokenizing'', performed on the source code. These represent  the  two extremes of translation. In  the  real  world, translators are rarely so pure, but tend to have bits of each technique.

I can think of  several  examples. I've already mentioned one: most interpreters, such as Microsoft BASIC, for  example, translate the source code (tokenize it) into an  intermediate  form so that it'll be easier to parse real time.

Another example is an assembler. The purpose of an assembler, of course, is to produce object code, and it normally does that on a one-to-one basis: one object instruction per line of source code. But almost every assembler also permits expressions as arguments. In this case, the expressions  are  always  constant expressions, and  so the assembler isn't supposed to  issue  object  code  for them. Rather, it  ``interprets'' the expressions and computes the corresponding constant result, which is what it actually emits as object code.

As a matter of fact, we  could  use  a bit of that ourselves. The translator we built in the  previous  installment  will dutifully spit out object code  for  complicated  expressions, even though every term in  the  expression  is  a  constant. In that case it would be far better if the translator behaved a bit more  like an interpreter, and just computed the equivalent constant result.

There is  a concept in compiler theory called ``lazy'' translation. The  idea is that you typically don't just  emit  code  at  every action. In fact, at the extreme you don't emit anything  at all, until  you  absolutely  have to. To accomplish this, the actions associated with the parsing routines  typically  don't  just emit code. Sometimes  they  do, but  often  they  simply  return in- formation back to the caller. Armed with  such  information, the caller can then make a better choice of what to do.

For example, given the statement

\begin{verbatim}
               x = x + 3 - 2 - (5 - 4),
\end{verbatim}

our compiler will dutifully spit  out a stream of 18 instructions to load each parameter into  registers, perform  the arithmetic, and store the result. A lazier evaluation  would  recognize that the arithmetic involving constants can  be  evaluated  at compile time, and would reduce the expression to

\begin{verbatim}
               x = x + 0.
\end{verbatim}

An  even  lazier  evaluation would then be smart enough to figure out that this is equivalent to

\begin{verbatim}
               x = x,
\end{verbatim}

which  calls  for  no  action  at  all. We could reduce 18  instructions to zero!

Note that there is no chance of optimizing this way in our translator as it stands, because every action takes place immediately.

Lazy  expression  evaluation  can  produce  significantly  better object code than  we  have  been  able  to  so  far. I warn you, though: it complicates the parser code considerably, because each routine now has to make decisions as to whether  to  emit  object code or not. Lazy evaluation is certainly not named that because it's easier on the compiler writer!

Since we're operating mainly on  the KISS principle here, I won't go  into much more depth on this subject. I just want you to  be aware  that  you  can get some code optimization by combining the techniques of compiling and  interpreting. In  particular, you should know that the parsing  routines  in  a  smarter translator will generally  return  things  to  their  caller, and sometimes expect things as  well. That's  the main reason for going over interpretation in this installment.

\section{The Interpreter}

OK, now that you know {\bfseries why} we're going into all this, let's do it. Just to give you practice, we're going to start over with  a bare cradle and build up the translator all over again. This time, of course, we can go a bit faster.

Since we're now going  to  do arithmetic, the first thing we need to do is to change function {\tt GetNum}, which up till now  has always returned a character  (or  string). Now, it's better for it to return an integer. {\bfseries make a copy} of the cradle (for goodness's sake, don't change the version  in  Cradle  itself!!)  and modify {\tt GetNum} as follows:

\begin{verbatim}
{------------------------------------------------------}
{ Get a Number }

function GetNum: integer;
begin
   if not IsDigit(Look) then Expected('Integer');
   GetNum := Ord(Look) - Ord('0');
   GetChar;
end;
{------------------------------------------------------}
\end{verbatim}

Now, write the following version of Expression:

\begin{verbatim}
{---------------------------------------------------------------}
{ Parse and Translate an Expression }

function Expression: integer;
begin
   Expression := GetNum;
end;
{------------------------------------------------------}
\end{verbatim}

Finally, insert the statement

\begin{verbatim}
   Writeln(Expression);
\end{verbatim}

at the end of the main program. Now compile and test.

All this program  does  is  to  ``parse''  and  translate  a single integer  ``expression''.    As always, you should make sure that it does that with the digits 0..9, and gives an  error  message  for anything else. Shouldn't take you very long!

OK, now let's extend this to include {\tt addops}. Change Expression to read:

\begin{verbatim}
{-------------------------------------------------------}
{ Parse and Translate an Expression }

function Expression: integer;
var Value: integer;
begin
   if IsAddop(Look) then
      Value := 0
   else
      Value := GetNum;
   while IsAddop(Look) do begin
      case Look of
       '+': begin
               Match('+');
               Value := Value + GetNum;
            end;
       '-': begin
               Match('-');
               Value := Value - GetNum;
            end;
      end;
   end;
   Expression := Value;
end;
{------------------------------------------------------}
\end{verbatim}

The structure of Expression, of  course, parallels  what  we did before, so  we  shouldn't have too much  trouble  debugging  it. There's  been  a  {\bfseries significant}  development, though, hasn't there?  Procedures Add and Subtract went away!  The reason  is  that  the action to be taken  requires {\bfseries BOTH} arguments of the operation. I could have chosen to retain the procedures and pass into them the value of the expression to date, which  is Value. But it seemed cleaner to me to  keep  Value as strictly a local variable, which meant that the code for Add and Subtract had to be moved in line. This result suggests  that, while the structure we had developed was nice and  clean  for our simple-minded translation scheme, it probably  wouldn't do for use with lazy  evaluation. That's  a little tidbit we'll probably want to keep in mind for later.

OK, did the translator work?  Then let's  take  the  next  step. It's not hard to  figure  out what procedure Term should now look like. Change every call to {\tt GetNum} in function  Expression  to  a call to Term, and then enter the following form for Term:

\begin{verbatim}
{-------------------------------------------------------}
{ Parse and Translate a Math Term }

function Term: integer;
var Value: integer;
begin
   Value := GetNum;
   while Look in ['*', '/'] do begin
      case Look of
       '*': begin
               Match('*');
               Value := Value * GetNum;
            end;
       '/': begin
               Match('/');
               Value := Value div GetNum;
            end;
      end;
   end;
   Term := Value;
end;
{------------------------------------------------------}
\end{verbatim}

Now, try it out. Don't forget two things: first, we're dealing with integer division, so, for example, 1/3 should come out zero. Second, even  though we can output multi-digit results, our input is still restricted to single digits.

That seems like a silly restriction at this point, since  we have already  seen how easily function {\tt GetNum} can  be  extended. So let's go ahead and fix it right now. The new version is

\begin{verbatim}
{------------------------------------------------------}
{ Get a Number }

function GetNum: integer;
var Value: integer;
begin
   Value := 0;
   if not IsDigit(Look) then Expected('Integer');
   while IsDigit(Look) do begin
      Value := 10 * Value + Ord(Look) - Ord('0');
      GetChar;
   end;
   GetNum := Value;
end;
{------------------------------------------------------}
\end{verbatim}

If you've compiled and  tested  this  version of the interpreter, the  next  step  is to install function Factor, complete with parenthesized  expressions. We'll hold off a  bit  longer  on  the variable  names. First, change the references  to  {\tt GetNum}, in function Term, so that they call Factor instead. Now  code  the following version of Factor:

\begin{verbatim}
{-------------------------------------------------------}
{ Parse and Translate a Math Factor }

function Expression: integer; Forward;

function Factor: integer;
begin
   if Look = '(' then begin
      Match('(');
      Factor := Expression;
      Match(')');
      end
   else
       Factor := GetNum;
end;
{-------------------------------------------------------}
\end{verbatim}

That was pretty easy, huh?  We're rapidly closing in on  a useful interpreter.

\section{A Little Philosophy}

Before going any further, there's something I'd like  to  call to your attention. It's a concept that we've been making use  of in all these sessions, but I haven't explicitly mentioned it up till now. I think it's time, because it's a concept so useful, and so powerful, that  it  makes all the difference  between  a  parser that's trivially easy, and one that's too complex to deal with.

In the early days of compiler technology, people  had  a terrible time  figuring  out  how to deal with things like operator precedence  ... the  way  that  multiply  and  divide operators  take precedence over add and subtract, etc. I remember a colleague of some  thirty years ago, and how excited he was to find out how to do it. The technique used involved building two  stacks, upon which you pushed each operator  or operand. Associated with each operator was a precedence level, and the rules required that you only actually performed an operation  (``reducing''  the  stack) if the precedence level showing on top of the stack was correct. To make life more interesting, an  operator  like ')' had different precedence levels, depending  upon  whether or not it was already on the stack. You  had to give it one value before you put it on the stack, and another to decide when to take it  off. Just for the experience, I worked all of  this  out for myself a few years ago, and I can tell you that it's very tricky.

We haven't  had  to  do  anything like that. In fact, by now the parsing of an arithmetic statement should seem like child's play. How did we get so lucky?  And where did the precedence stacks go?

A similar thing is going on  in  our interpreter above. You just {\bfseries know} that in  order  for  it  to do the computation of arithmetic statements (as opposed to the parsing of them), there have  to be numbers pushed onto a stack somewhere. But where is the stack?

Finally, in compiler textbooks, there are  a  number  of  places where  stacks  and  other structures are discussed. In the other leading parsing method (LR), an explicit stack is used. In fact, the technique is very  much  like the old way of doing arithmetic expressions. Another concept  is  that of a parse tree. Authors like to draw diagrams  of  the  tokens  in a statement, connected into a tree with  operators  at the internal nodes. Again, where are the trees and stacks in our technique?  We haven't seen any. The answer in all cases is that the structures are  implicit, not explicit. In  any computer language, there is a stack involved every  time  you  call  a  subroutine. Whenever a subroutine  is called, the return address is pushed onto the CPU stack. At the end of the subroutine, the address is popped back off and control is  transferred  there. In a recursive language such as Pascal, there can also be local data pushed onto the stack, and  it, too, returns when it's needed.

For example, function  Expression  contains  a  local  parameter called  Value, which it fills by a call to Term. Suppose, in its next call to  Term  for  the  second  argument, that  Term calls Factor, which recursively  calls  Expression  again. That ``instance'' of Expression gets another value for its  copy  of Value. What happens  to  the  first  Value?    Answer: it's still on the stack, and  will  be  there  again  when  we return from our call sequence.

In other words, the reason things look so simple  is  that  we've been making maximum use of the resources of the  language. The hierarchy levels  and  the  parse trees are there, all right, but they're hidden within the  structure  of  the parser, and they're taken care of by the order with which the various  procedures are called. Now that you've seen how we do it, it's probably hard to imagine doing it  any other way. But I can tell you that it took a lot of years for compiler writers to get that smart. The early compilers were too complex  too  imagine. Funny how things get easier with a little practice.

The reason  I've  brought  all  this up is as both a lesson and a warning. The lesson: things can be easy when you do  them right. The warning: take a look at what you're doing. If, as you branch out on  your  own, you  begin to find a real need for a separate stack or tree structure, it may be time to ask yourself if you're looking at things the right way. Maybe you just aren't using the facilities of the language as well as you could be.

The next step is to add variable names. Now, though, we have a slight problem. For  the  compiler, we had no problem in dealing with variable names ... we just issued the names to the assembler and let the rest  of  the program take care of allocating storage for  them. Here, on the other hand, we need to be able to  fetch the values of the variables and return them as the  return values of Factor. We need a storage mechanism for these variables.

Back in the early days of personal computing, Tiny  BASIC lived. It had  a  grand  total  of  26  possible variables: one for each letter of the  alphabet. This  fits nicely with our concept of single-character tokens, so we'll  try  the  same  trick. In the beginning of your  interpreter, just  after  the  declaration of variable Look, insert the line:

\begin{verbatim}
               Table: Array['A'..'Z'] of integer;
\end{verbatim}

We also need to initialize the array, so add this procedure:

\begin{verbatim}
{-------------------------------------------------------}
{ Initialize the Variable Area }

procedure InitTable;
var i: char;
begin
   for i := 'A' to 'Z' do
      Table[i] := 0;
end;
{-------------------------------------------------------}
\end{verbatim}

You must also insert a call to InitTable, in procedure {\tt Init}. {\bfseries don't forget} to do that, or the results may surprise you!

Now that we have an array  of  variables, we can modify Factor to use it. Since we don't have a way (so far) to set the variables, Factor  will always return zero values for  them, but  let's  go ahead and extend it anyway. Here's the new version:

\begin{verbatim}
{-------------------------------------------------------}
{ Parse and Translate a Math Factor }

function Expression: integer; Forward;

function Factor: integer;
begin
   if Look = '(' then begin
      Match('(');
      Factor := Expression;
      Match(')');
      end
   else if IsAlpha(Look) then
      Factor := Table[GetName]
   else
       Factor := GetNum;
end;
{-------------------------------------------------------}
\end{verbatim}

As always, compile and test this version of the  program. Even though all the variables are now zeros, at least we can correctly parse the complete expressions, as well as catch any badly formed expressions.

I suppose you realize the next step: we need to do  an assignment statement so we can  put  something {\bfseries into} the variables. For now, let's  stick  to  one-liners, though  we will soon  be  handling multiple statements.

The assignment statement parallels what we did before:

\begin{verbatim}
{------------------------------------------------------}
{ Parse and Translate an Assignment Statement }

procedure Assignment;
var Name: char;
begin
   Name := GetName;
   Match('=');
   Table[Name] := Expression;
end;
{------------------------------------------------------}
\end{verbatim}

To test this, I  added  a  temporary write statement in the main program, to  print out the value of A. Then I  tested  it  with various assignments to it.

Of course, an interpretive language that can only accept a single line of program  is not of much value. So we're going to want to handle multiple statements. This  merely  means  putting  a loop around  the  call  to Assignment. So let's do that now. But what should be the loop exit criterion?  Glad you  asked, because  it brings up a point we've been able to ignore up till now.

One of the most tricky things  to  handle in any translator is to determine when to bail out of  a  given construct and go look for something else. This hasn't been a problem for us so far because we've only allowed for  a  single kind of construct ... either an expression  or an assignment statement. When  we  start  adding loops and different kinds of statements, you'll find that we have to be very careful that things terminate properly. If we put our interpreter in a loop, we need a way to quit. Terminating on a newline is no good, because that's what sends us back for another line. We could always let an unrecognized character take us out, but that would cause every run to end in an error  message, which certainly seems uncool.

What we need  is  a  termination  character. I vote for Pascal's ending period (`.'). A  minor  complication  is that Turbo ends every normal line  with {\bfseries two} characters, the carriage return (CR) and line feed (LF). At  the  end  of  each line, we need to eat these characters before processing the next one. A  natural way to do this would  be  with  procedure  Match, except that Match's error  message  prints  the character, which of course for the CR and/or  LF won't look so great. What we need is a special procedure for this, which we'll no doubt be using over and over. Here it is:

\begin{verbatim}
{------------------------------------------------------}
{ Recognize and Skip Over a Newline }

procedure NewLine;
begin
   if Look = CR then begin
      GetChar;
      if Look = LF then
         GetChar;
   end;
end;
{------------------------------------------------------}
\end{verbatim}

Insert this procedure at any convenient spot ... I put  mine just after Match. Now, rewrite the main program to look like this:

\begin{verbatim}
{------------------------------------------------------}
{ Main Program }

begin
   Init;
   repeat
      Assignment;
      NewLine;
   until Look = '.';
end.
{------------------------------------------------------}
\end{verbatim}

Note that the  test for a CR is now gone, and that there are also no  error tests within {\tt NewLine} itself. That's  OK, though  ... whatever is left over in terms of bogus characters will be caught at the beginning of the next assignment statement.

Well, we now have a functioning interpreter. It doesn't do  us a lot of  good, however, since  we have no way to read data in or write it out. Sure would help to have some I/O!

Let's wrap this session  up, then, by  adding the I/O routines. Since we're  sticking to single-character tokens, I'll use `?' to stand for a read statement, and  `!'  for a write, with the character  immediately  following  them  to  be used as  a  one-token ``parameter list''.  Here are the routines:

\begin{verbatim}
{------------------------------------------------------}
{ Input Routine }

procedure Input;
begin
   Match('?');
   Read(Table[GetName]);
end;


{------------------------------------------------------}
{ Output Routine }

procedure Output;
begin
   Match('!');
   WriteLn(Table[GetName]);
end;
{------------------------------------------------------}
\end{verbatim}

They aren't very fancy, I admit ... no prompt character on input, for example ... but they get the job done.

The corresponding changes in  the  main  program are shown below. Note that we use the usual  trick  of a case statement based upon the current lookahead character, to decide what to do.

\begin{verbatim}
{------------------------------------------------------}
{ Main Program }

begin
   Init;
   repeat
      case Look of
       '?': Input;
       '!': Output;
       else Assignment;
      end;
      NewLine;
   until Look = '.';
end.
{------------------------------------------------------}
\end{verbatim}

You have now completed a  real, working interpreter. It's pretty sparse, but it works just like the ``big boys.''  It includes three kinds of program statements  (and  can  tell the difference!), 26 variables, and  I/O  statements. The only things that it lacks, really, are control statements, subroutines, and some kind of program editing function. The program editing part, I'm going to pass on. After all, we're  not  here  to build a product, but to learn  things. The control statements, we'll cover in the next installment, and the subroutines soon  after. I'm anxious to get on with that, so we'll leave the interpreter as it stands.

I hope that by  now  you're convinced that the limitation of single-character names  and the processing of white space are easily taken  care  of, as we did in the last session. This  time, if you'd like to play around with these extensions, be my  guest ... they're  ``left as an exercise for the student.''    See  you  next time.
