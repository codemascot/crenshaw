
\chapter{LEXICAL SCANNING}

7th Nov 1988

\section{Introduction}

In the last installment, I left you with a  compiler  that  would {\bfseries almost}  work, except  that  we  were  still  limited to  single- character tokens. The purpose of  this  session is to get rid of that restriction, once and for all. This means that we must deal with the concept of the lexical scanner.

Maybe I should mention why we  need  a lexical scanner at all ... after all, we've been able to manage all right  without  one, up till now, even when we provided for multi-character tokens.

The {\bfseries only} reason, really, has to do with keywords. It's a fact of computer life that the syntax for a keyword has the same  form as that  for  any  other identifier. We can't tell until we get the complete word whether or not it  IS  a keyword. For example, the variable IFILE and the keyword IF look just alike, until  you get to the third character. In the examples to date, we  were always able to make  a  decision  based  upon the first character of the token, but that's  no  longer possible when keywords are present. We  need to know that a given string is a keyword {\bfseries before} we begin to process it. And that's why we need a scanner.

In the last session, I also promised that  we  would  be  able to provide for normal tokens  without  making  wholesale  changes to what we have  already done. I didn't lie ... we can, as you will see later. But every time I set out to install these elements of the software into  the  parser  we  have already built, I had bad feelings about it. The whole thing felt entirely too much like a band-aid. I finally figured out what was causing the  problem: I was installing lexical scanning software without first explaining to you what scanning is all about, and what the alternatives are. Up  till  now, I have studiously avoided  giving  you  a  lot  of theory, and  certainly  not  alternatives. I  generally don't respond well to the textbooks that give you twenty-five different ways  to do something, but no clue as to which way best fits your needs. I've tried to avoid that pitfall by just showing  you {\bfseries one} method, that {\bfseries works}.

But  this is an important area. While  the  lexical  scanner  is hardly the most  exciting  part  of  a compiler, it often has the most  profound  effect  on  the  general  ``look \& feel''  of  the language, since after all it's the  part  closest to the user. I have a particular structure in mind for the scanner  to  be  used with  KISS. It fits the look \&  feel  that  I  want  for  that language. But it may not work at  all  for  the  language {\bfseries you're} cooking  up, so  in this one case I feel that it's important for you to know your options.

So I'm going to depart, again, from my  usual  format. In this session we'll be getting  much  deeper  than usual into the basic theory of languages and  grammars. I'll  also be talking about areas {\bfseries other} than compilers in  which  lexical  scanning  plays an important role. Finally, I will show you  some  alternatives for the structure of the lexical scanner. Then, and only  then, will we get back to our parser  from  the last installment. Bear with me ... I think you'll find it's worth the wait. In fact, since scanners have many applications  outside  of  compilers, you may well find this to be the most useful session for you.

\section{Lexical Scanning}

Lexical scanning is the process of scanning the  stream  of input characters and separating it  into  strings  called tokens. Most compiler  texts  start  here, and  devote  several  chapters  to discussing various ways to build scanners. This approach has its place, but as you have already  seen, there  is a lot you can do without ever even addressing the issue, and in  fact  the scanner we'll  end  up with here won't look  much  like  what  the  texts describe. The reason?    Compiler  theory and, consequently, the programs resulting from it, must  deal with the most general kind of parsing rules. We don't. In the real  world, it is possible to specify the language syntax in such a way that a pretty simple scanner will suffice. And as always, KISS is our motto.

Typically, lexical scanning is  done  in  a  separate part of the compiler, so that the parser per  se  sees only a stream of input tokens. Now, theoretically it  is not necessary to separate this function from the rest of the parser. There is  only  one set of syntax equations that define the  whole language, so in theory we could write the whole parser in one module.

Why  the  separation?  The  answer  has  both  practical  and theoretical bases.

In  1956, Noam Chomsky  defined  the  ``Chomsky Hierarchy'' of grammars. They are:

\begin{itemize}
\item	Type 0:  Unrestricted (e.g., English)
\item	Type 1:  Context-Sensitive
\item	Type 2:  Context-Free
\item	Type 3:  Regular
\end{itemize}

A few features of the typical programming  language (particularly the older ones, such as FORTRAN) are Type  1, but  for  the most part  all  modern  languages can be described using only the last two types, and those are all we'll be dealing with here.

The  neat  part about these two types  is  that  there  are  very specific ways to parse them. It has been shown that  any regular grammar can be parsed using a particular form of abstract machine called the state machine (finite  automaton). We  have already implemented state machines in some of our recognizers.

Similarly, Type 2 (context-free) grammars  can  always  be parsed using  a  push-down  automaton (a state machine  augmented  by  a stack). We have  also  implemented  these  machines. Instead of implementing  a literal stack, we have  relied  on  the  built-in stack associated with recursive coding to do the job, and that in fact is the preferred approach for top-down parsing.

Now, it happens that in  real, practical grammars, the parts that qualify as  regular expressions tend to be the lower-level parts, such as the definition of an identifier:

\begin{verbatim}
     <ident> ::= <letter> [ <letter> | <digit> ]*
\end{verbatim}

Since it takes a different kind of abstract machine to  parse the two  types  of  grammars, it makes sense to separate these lower- level functions into  a  separate  module, the  lexical scanner, which is built around the idea of a state machine. The idea is to use the simplest parsing technique needed for the job.

There is another, more practical  reason  for  separating scanner from  parser. We like to think of the input source  file  as  a stream  of characters, which we process  right  to  left  without backtracking. In practice that  isn't  possible. Almost every language has certain keywords such as  IF, WHILE, and END. As I mentioned  earlier, we  can't  really  know  whether  a  given character string is a keyword, until we've reached the end of it, as defined by a space or other delimiter. So  in  that sense, we {\bfseries must}  save  the  string long enough to find out whether we have a keyword or not. That's a limited form of backtracking.

So the structure of a conventional compiler involves splitting up the functions of  the  lower-level and higher-level parsing. The lexical  scanner  deals  with  things  at  the  character  level, collecting characters into strings, etc., and passing  them along to the parser proper as indivisible tokens. It's also considered normal to let the scanner have the job of identifying keywords.

\section{State Machines And Alternatives}

I  mentioned  that  the regular expressions can be parsed using a state machine. In  most  compiler  texts, and  indeed  in most compilers as well, you will find this taken literally. There is typically  a  real  implementation  of  the  state  machine, with integers used to define the current state, and a table of actions to  take   for  each  combination  of  current  state  and  input character. If you  write  a compiler front end using the popular Unix tools LEX and YACC, that's  what  you'll get. The output of LEX is a state machine implemented in C, plus a table  of actions corresponding to the input grammar given to LEX. The YACC output is  similar  ... a canned table-driven parser, plus  the  table corresponding to the language syntax.

That  is  not  the  only  choice, though. In   our  previous installments, you have seen over and over that it is  possible to implement  parsers  without  dealing  specifically  with  tables, stacks, or state variables. In fact, in Installment V I warned you that if you  find  yourself needing these things you might be doing something wrong, and not taking advantage of  the  power of Pascal. There are basically two ways to define a state machine's state: explicitly, with  a  state number or code, and implicitly, simply by virtue of the fact that I'm at a  certain  place in the code  (if  it's  Tuesday, this  must be Belgium). We've  relied heavily on the implicit approaches  before, and  I  think you'll find that they work well here, too.

In practice, it may not even be necessary to {\bfseries have}  a well-defined lexical scanner. This isn't our first experience at dealing with multi-character tokens. In  Installment  III, we  extended our parser to provide  for  them, and  we didn't even {\bfseries need} a lexical scanner. That  was  because  in that narrow context, we  could always tell, just  by  looking at the single lookahead character, whether  we  were  dealing  with  a  number, a variable, or  an operator. In effect, we  built  a  distributed  lexical scanner, using procedures {\tt GetName} and {\tt GetNum}.

With keywords present, we  can't know anymore what we're dealing with, until the entire token is  read. This leads us to a more localized  scanner; although, as you will see, the  idea  of  a distributed scanner still has its merits.

\section{Some Experiments In Scanning}

Before  getting  back  to our compiler, it  will  be  useful  to experiment a bit with the general concepts.

Let's  begin with the two definitions most  often  seen  in  real programming languages:

\begin{verbatim}
     <ident> ::= <letter> [ <letter> | <digit> ]*
     <number ::= [<digit>]+
\end{verbatim}

(Remember, the `*' indicates zero or more occurrences of the terms in brackets, and the `+', one or more.)

We  have already dealt with similar  items  in  Installment  III. Let's begin (as usual) with a bare cradle. Not  surprisingly, we are going to need a new recognizer:

\begin{verbatim}
{----------------------------------------------}
{ Recognize an Alphanumeric Character }

function IsAlNum(c: char): boolean;
begin
   IsAlNum := IsAlpha(c) or IsDigit(c);
end;
{----------------------------------------------}
\end{verbatim}

Using this let's write the following two routines, which are very similar to those we've used before:

\begin{verbatim}
{----------------------------------------------}
{ Get an Identifier }

function GetName: string;
var x: string[8];
begin
   x := '';
   if not IsAlpha(Look) then Expected('Name');
   while IsAlNum(Look) do begin
     x := x + UpCase(Look);
     GetChar;
   end;
   GetName := x;
end;

{----------------------------------------------}
{ Get a Number }

function GetNum: string;
var x: string[16];
begin
   x := '';
   if not IsDigit(Look) then Expected('Integer');
   while IsDigit(Look) do begin
     x := x + Look;
     GetChar;
   end;
   GetNum := x;
end;
{----------------------------------------------}
\end{verbatim}

(Notice  that this version of {\tt GetNum} returns  a  string, not  an integer as before.)

You  can  easily  verify that these routines work by calling them from the main program, as in

\begin{verbatim}
     WriteLn(GetName);
\end{verbatim}

This  program  will  print any legal name typed in (maximum eight characters, since that's what we told {\tt GetName}). It  will reject anything else.

Test the other routine similarly.


\section{White Space}

We  also  have  dealt with embedded white space before, using the two  routines  {\tt IsWhite} and  {\tt SkipWhite}. Make  sure that  these routines are in your  current  version of the cradle, and add the the line

\begin{verbatim}
     SkipWhite;
\end{verbatim}

at the end of both {\tt GetName} and {\tt GetNum}.

Now, let's define the new procedure:

\begin{verbatim}
{----------------------------------------------}
{ Lexical Scanner }

Function Scan: string;
begin
   if IsAlpha(Look) then
      Scan := GetName
   else if IsDigit(Look) then
      Scan := GetNum
   else begin
      Scan := Look;
      GetChar;
   end;
   SkipWhite;
end;
{----------------------------------------------}
\end{verbatim}

We can call this from the new main program:

\begin{verbatim}
{----------------------------------------------}
{ Main Program }

begin
   Init;
   repeat
      Token := Scan;
      writeln(Token);
   until Token = CR;
end.
{----------------------------------------------}
\end{verbatim}

(You will have to add the declaration of the string Token  at the beginning of the program. Make it any convenient length, say 16 characters.)

Now, run the program. Note how the  input  string  is, indeed, separated into distinct tokens.

\section{State Machines}

For  the  record, a  parse  routine  like  {\tt GetName}  does  indeed implement a state machine. The state is implicit in  the current position in the code. A very useful trick for visualizing what's going on is  the  syntax  diagram, or  ``railroad-track'' diagram. It's a little difficult to draw  one  in this medium, so I'll use them very sparingly, but  the  figure  below  should give you the idea:

As  you  can  see, this  diagram  shows  how  the logic flows as characters  are  read. Things  begin, of course, in the  start state, and end when  a  character  other  than an alphanumeric is found. If  the  first  character  is not alpha, an error occurs. Otherwise the machine will continue looping until the terminating delimiter is found.

Note  that at any point in the flow, our  position  is  entirely dependent on the past  history  of the input characters. At that point, the action to be taken depends only on the  current state, plus the current input character. That's what make this  a state machine.

Because of the difficulty of drawing  railroad-track  diagrams in this medium, I'll continue to  stick to syntax equations from now on. But I highly recommend the diagrams to you for  anything you do that involves parsing. After a little practice you  can begin to  see  how  to  write  a  parser  directly from  the  diagrams. Parallel paths get coded into guarded actions (guarded by IF's or CASE statements), serial  paths  into  sequential  calls. It's almost like working from a schematic.

We didn't even discuss {\tt SkipWhite}, which  was  introduced earlier, but it also is a simple state machine, as is {\tt GetNum}. So is their parent procedure, Scan. Little machines make big machines.

The neat thing that I'd like  you  to note is how painlessly this implicit approach creates these  state  machines. I personally prefer it a lot over the table-driven approach. It  also results is a small, tight, and fast scanner.

\section{Newlines}

Moving right along, let's modify  our scanner to handle more than one line. As I mentioned last time, the most straightforward way to  do  this  is to simply treat the newline characters, carriage return  and line feed, as white space. This is, in fact, the way the  C  standard  library  routine, {\tt IsWhite}, works. We  didn't actually try this  before. I'd like to do it now, so you can get a feel for the results.

To do this, simply modify the single executable  line  of {\tt IsWhite} to read:

\begin{verbatim}
   IsWhite := c in [' ', TAB, CR, LF];
\end{verbatim}

We need to give the main  program  a new stop condition, since it will never see a CR. Let's just use:

\begin{verbatim}
   until Token = '.';
\end{verbatim}

OK, compile this  program  and  run  it. Try a couple of lines, terminated by the period. I used:

\begin{verbatim}
     now is the time
     for all good men.
\end{verbatim}

Hey, what  happened?   When I tried it, I didn't  get  the  last token, the period. The program didn't halt. What's more, when I pressed the  'enter'  key  a  few  times, I still didn't get the period.

If you're still stuck in your program, you'll find that  typing a period on a new line will terminate it.

What's going on here?  The answer is  that  we're  hanging  up in {\tt SkipWhite}. A quick look at  that  routine will show that as long as we're typing null lines, we're going to just continue to loop. After {\tt SkipWhite} encounters an LF, it tries to execute a {\tt GetChar}. But since the input buffer is now empty, {\tt GetChar}'s read statement insists  on  having  another  line. Procedure  Scan  gets  the terminating period, all right, but  it  calls {\tt SkipWhite} to clean up, and {\tt SkipWhite} won't return until it gets a non-null line.

This kind of behavior is not quite as bad as it seems. In a real compiler, we'd  be  reading  from  an input file instead of  the console, and as long  as  we have some procedure for dealing with end-of-files, everything will come out  OK. But for reading data from the console, the behavior is just too bizarre. The  fact of the matter is that the C/Unix convention is  just  not compatible with the structure of  our  parser, which  calls for a lookahead character. The  code that the Bell  wizards  have  implemented doesn't use that convention, which is why they need 'ungetc'.

OK, let's fix the problem. To do that, we need to go back to the old definition of {\tt IsWhite} (delete the CR and  LF  characters) and make  use  of  the procedure Fin that I introduced last time. If it's not in your current version of the cradle, put it there now.

Also, modify the main program to read:

\begin{verbatim}
{----------------------------------------------}
{ Main Program }

begin
   Init;
   repeat
      Token := Scan;
      writeln(Token);
      if Token = CR then Fin;
   until Token = '.';
end.
{----------------------------------------------}
\end{verbatim}

Note the ``guard''  test  preceding  the  call to Fin. That's what makes the whole thing work, and ensures that we don't try to read a line ahead.

Try the code now. I think you'll like it better.

If you refer to the code  we  did in the last installment, you'll find that I quietly sprinkled calls to Fin  throughout  the code, wherever  a line break was appropriate. This  is  one  of  those areas that really affects the look  \&  feel that I mentioned. At this  point  I  would  urge  you  to  experiment  with  different arrangements  and  see  how  you  like  them. If you want your language  to  be  truly  free-field, then  newlines   should  be transparent. In  this  case, the  best  approach is to put the following lines at the BEGINNING of Scan:

\begin{verbatim}
          while Look = CR do
             Fin;
\end{verbatim}

If, on the other  hand, you  want  a line-oriented language like Assembler, BASIC, or FORTRAN  (or  even  Ada... note that it has comments terminated by newlines), then  you'll  need for Scan to return CR's as tokens. It  must  also  eat the trailing LF. The best way to do that is to use this line, again  at the beginning of Scan:

\begin{verbatim}
          if Look = LF then Fin;
\end{verbatim}

For other conventions, you'll  have  to  use  other arrangements. In my example  of  the  last  session, I allowed newlines only at specific places, so I was somewhere in the middle ground. In the rest of these sessions, I'll be picking ways  to  handle newlines that I happen to like, but I want you to know how to choose other ways for yourselves.

\section{Operators}

We  could  stop now and have a  pretty  useful  scanner  for  our purposes. In the fragments of KISS that we've built so  far, the only tokens that have multiple characters are the identifiers and numbers. All  operators  were  single  characters. The  only exception I can think of is the relops <=, >=, and  <>, but they could be dealt with as special cases.

Still, other languages have  multi-character  operators, such as the ':=' of  Pascal or the '++' and '>>' of C. So  while  we may not need multi-character operators, it's  nice to know how to get them if necessary.

Needless to say, we  can  handle operators very much the same way as the other tokens. Let's start with a recognizer:

\begin{verbatim}
{----------------------------------------------}
{ Recognize Any Operator }

function IsOp(c: char): boolean;
begin
   IsOp := c in ['+', '-', '*', '/', '<', '>', ':', '='];
end;
{----------------------------------------------}
\end{verbatim}

It's important to  note  that  we  DON'T  have  to  include every possible  operator in this list. For  example, the paretheses aren't  included, nor is the terminating  period. The  current version of Scan handles single-character operators  just  fine as it is. The list above includes only those  characters  that  can appear in multi-character operators. (For specific languages, of course, the list can always be edited.)

Now, let's modify Scan to read:

\begin{verbatim}
{----------------------------------------------}
{ Lexical Scanner }

Function Scan: string;
begin
   while Look = CR do
      Fin;
   if IsAlpha(Look) then
      Scan := GetName
   else if IsDigit(Look) then
      Scan := GetNum
   else if IsOp(Look) then
      Scan := GetOp
   else begin
      Scan := Look;
      GetChar;
   end;
   SkipWhite;
end;
{----------------------------------------------}
\end{verbatim}

Try the program now. You  will  find that any code fragments you care  to throw at it will be neatly  broken  up  into  individual tokens.

\section{Lists, Commas And Command Lines}

Before getting back to the main thrust of our study, I'd  like to get on my soapbox for a moment.

How many times have you worked with a program or operating system that had rigid rules about how you must separate items in a list?  (Try, the  last  time  you  used MSDOS!)  Some programs  require spaces as delimiters, and  some  require  commas. Worst of all, some  require  both, in  different  places. Most  are  pretty unforgiving about violations of their rules.

I think this is inexcusable. It's too  easy  to  write  a parser that will handle  both  spaces  and  commas  in  a  flexible way. Consider the following procedure:

\begin{verbatim}
{----------------------------------------------}
{ Skip Over a Comma }

procedure SkipComma;
begin
   SkipWhite;
   if Look = ',' then begin
      GetChar;
      SkipWhite;
   end;
end;
{----------------------------------------------}
\end{verbatim}

This eight-line procedure will skip over  a  delimiter consisting of any number (including zero)  of spaces, with zero or one comma embedded in the string.

TEMPORARILY, change the call to {\tt SkipWhite} in Scan to  a  call  to {\tt SkipComma}, and  try  inputting some lists. Works  nicely, eh?  Don't you wish more software authors knew about {\tt SkipComma}?

For the record, I found that adding the  equivalent  of {\tt SkipComma} to my Z80 assembler-language programs took all of  6  (six) extra bytes of  code. Even  in a 64K machine, that's not a very high price to pay for user-friendliness!

I  think  you can see where I'm going here. Even  if  you  never write a line of a compiler code in your life, there are places in every program where  you  can  use  the concepts of parsing. Any program that processes a command line needs them. In  fact, if you  think  about  it for a bit, you'll have to conclude that any time  you  write  a program that processes  user  inputs, you're defining a  language. People communicate with languages, and the syntax implicit in your program  defines that language. The real question  is:  are  you  going  to  define  it  deliberately  and explicitly, or just let it turn out to be  whatever  the  program ends up parsing?

I claim that you'll have  a better, more user-friendly program if you'll take the time to define the syntax explicitly. Write down the syntax equations or  draw  the  railroad-track  diagrams, and code the parser using the techniques I've shown you here. You'll end  up with a better program, and it will be easier to write, to boot.

\section{Getting Fancy}

OK, at this point we have a pretty nice lexical scanner that will break  an  input stream up into tokens. We could use  it  as  it stands and have a serviceable compiler. But there are  some other aspects of lexical scanning that we need to cover.

The main consideration is <shudder> efficiency. Remember when we were dealing  with  single-character  tokens, every  test  was a comparison of a single character, Look, with a byte constant. We also used the Case statement heavily.

With the multi-character tokens being returned by Scan, all those tests now become string comparisons. Much slower. And  not only slower, but more awkward, since  there is no string equivalent of the  Case  statement  in Pascal. It seems especially wasteful to test for what used to be single characters ... the `=', `+', and other operators ... using string comparisons.

Using string comparison is not  impossible ... Ron Cain used just that approach in writing Small C. Since we're  sticking  to  the KISS principle here, we would  be truly justified in settling for this  approach. But then I would have failed to tell you about one of the key approaches used in ``real'' compilers.

You have to remember: the lexical scanner is going to be called a  LOT !   Once for every token in the  whole  source  program, in fact. Experiments  have  indicated  that  the  average compiler spends  anywhere  from 20\% to 40\% of  its  time  in  the  scanner routines. If there were ever a place  where  efficiency deserves real consideration, this is it.

For this reason, most compiler writers ask the lexical scanner to do  a  little  more work, by ``tokenizing''  the input stream. The idea  is  to  match every token  against  a  list  of  acceptable keywords  and operators, and return unique  codes  for  each  one recognized. In the case of ordinary variable  names  or numbers, we  just return a code that says what kind of token they are, and save the actual string somewhere else.

One  of the first things we're going to need is a way to identify keywords. We can always do  it  with successive IF tests, but it surely would be nice  if  we  had  a general-purpose routine that could compare a given string with  a  table of keywords. (By the way, we're also going  to  need such a routine later, for dealing with symbol tables.)  This  usually presents a problem in Pascal, because standard Pascal  doesn't  allow  for  arrays  of variable lengths. It's  a  real  bother  to  have to declare a different search routine for every table. Standard  Pascal  also doesn't allow for initializing arrays, so you tend to see code like

\begin{verbatim}
     Table[1] := 'IF';
     Table[2] := 'ELSE';
     .
     .
     Table[n] := 'END';
\end{verbatim}

which can get pretty old if there are many keywords.

Fortunately, Turbo Pascal 4.0 has extensions that  eliminate both of  these  problems. Constant arrays can be declared using TP's ``typed constant'' facility, and  the  variable  dimensions  can be handled with its C-like extensions for pointers.

First, modify your declarations like this:

\begin{verbatim}
{----------------------------------------------}
{ Type Declarations  }

type Symbol = string[8];

     SymTab = array[1..1000] of Symbol;

     TabPtr = ^SymTab;
{----------------------------------------------}
\end{verbatim}

(The dimension  used  in  SymTab  is  not  real ... no storage is allocated by the declaration itself, and the number need only be ``big enough''.)

Now, just beneath those declarations, add the following:

\begin{verbatim}
{----------------------------------------------}
{ Definition of Keywords and Token Types }

const KWlist: array [1..4] of Symbol =
              ('IF', 'ELSE', 'ENDIF', 'END');

{----------------------------------------------}
\end{verbatim}

Next, insert the following new function:

\begin{verbatim}
{----------------------------------------------}
{ Table Lookup }

{ If the input string matches a table entry, return the entry index. If not, return a zero. }

function Lookup(T: TabPtr; s: string; n: integer): integer;
var i: integer;
    found: boolean;
begin
   found := false;
   i := n;
   while (i > 0) and not found do
      if s = T^[i] then
         found := true
      else
         dec(i);
   Lookup := i;
end;
{----------------------------------------------}
\end{verbatim}

To test it, you  can  temporarily  change  the  main  program as follows:

\begin{verbatim}
{----------------------------------------------}
{ Main Program }


begin
   ReadLn(Token);
   WriteLn(Lookup(Addr(KWList), Token, 4));
end.
{----------------------------------------------}
\end{verbatim}

Notice how Lookup is called: The Addr function sets up  a pointer to KWList, which gets passed to Lookup.

OK, give this  a  try. Since we're bypassing Scan here, you'll have to type the keywords in upper case to get any matches.

Now that we can recognize keywords, the next thing is  to arrange to return codes for them.

So what kind of code should we return?  There are really only two reasonable choices. This seems like an ideal application for the Pascal enumerated type. For  example, you can define something like

\begin{verbatim}
  SymType = (IfSym, ElseSym, EndifSym,
        EndSym, Ident, Number, Operator);
\end{verbatim}

and arrange to return a variable of this type. Let's  give it a try. Insert the line above into your type definitions.

Now, add the two variable declarations:

\begin{verbatim}
    Token: Symtype;          { Current Token  }
    Value: String[16];       { String Token of Look }
\end{verbatim}

Modify the scanner to read:

\begin{verbatim}
{----------------------------------------------}
{ Lexical Scanner }

procedure Scan;
var k: integer;
begin
   while Look = CR do
      Fin;
   if IsAlpha(Look) then begin
      Value := GetName;
      k := Lookup(Addr(KWlist), Value, 4);
      if k = 0 then
         Token := Ident
      else
         Token := SymType(k - 1);
      end
   else if IsDigit(Look) then begin
      Value := GetNum;
      Token := Number;
      end
   else if IsOp(Look) then begin
      Value := GetOp;
      Token := Operator;
      end
   else begin
      Value := Look;
      Token := Operator;
      GetChar;
   end;
   SkipWhite;
end;
{----------------------------------------------}
\end{verbatim}

(Notice that Scan is now a procedure, not a function.)
Finally, modify the main program to read:

\begin{verbatim}
{----------------------------------------------}
{ Main Program }

begin
   Init;
   repeat
      Scan;
      case Token of
        Ident: write('Ident ');
        Number: Write('Number ');
        Operator: Write('Operator ');
        IfSym, ElseSym, EndifSym, EndSym: Write('Keyword ');
      end;
      Writeln(Value);
   until Token = EndSym;
end.
{----------------------------------------------}
\end{verbatim}

What we've done here is to replace the string Token  used earlier with an enumerated type. Scan returns the type in variable Token, and returns the string itself in the new variable Value.

OK, compile this and give it a whirl. If everything  goes right, you should see that we are now recognizing keywords.

What  we  have  now is working right, and it was easy to generate from what  we  had  earlier. However, it still seems a little ``busy'' to me. We can  simplify  things a bit by letting {\tt GetName, GetNum, GetOp}, and {\tt Scan} be  procedures  working  with  the global variables Token and Value, thereby eliminating the  local copies. It  also seems a little cleaner to move  the  table  lookup  into {\tt GetName}. The new form for the four procedures is, then:

\begin{verbatim}
{----------------------------------------------}
{ Get an Identifier }

procedure GetName;
var k: integer;
begin
   Value := '';
   if not IsAlpha(Look) then Expected('Name');
   while IsAlNum(Look) do begin
     Value := Value + UpCase(Look);
     GetChar;
   end;
   k := Lookup(Addr(KWlist), Value, 4);
   if k = 0 then
      Token := Ident
   else
      Token := SymType(k-1);
end;

{----------------------------------------------}
{ Get a Number }

procedure GetNum;
begin
   Value := '';
   if not IsDigit(Look) then Expected('Integer');
   while IsDigit(Look) do begin
     Value := Value + Look;
     GetChar;
   end;
   Token := Number;
end;

{----------------------------------------------}
{ Get an Operator }

procedure GetOp;
begin
   Value := '';
   if not IsOp(Look) then Expected('Operator');
   while IsOp(Look) do begin
     Value := Value + Look;
     GetChar;
   end;
   Token := Operator;
end;

{----------------------------------------------}
{ Lexical Scanner }

procedure Scan;
var k: integer;
begin
   while Look = CR do
      Fin;
   if IsAlpha(Look) then
      GetName
   else if IsDigit(Look) then
      GetNum
   else if IsOp(Look) then
      GetOp
   else begin
      Value := Look;
      Token := Operator;
      GetChar;
   end;
   SkipWhite;
end;
{----------------------------------------------}
\end{verbatim}

\section{Returning A Character}

Essentially  every scanner I've ever seen  that  was  written  in Pascal  used  the  mechanism of an enumerated type that I've just described. It is certainly  a workable mechanism, but it doesn't seem the simplest approach to me.

For one thing, the  list  of possible symbol types can get pretty long. Here, I've used just one symbol, ``Operator'',  to  stand for all of the operators, but I've seen other  designs  that actually return different codes for each one.

There is, of course, another simple type that can be  returned as a  code: the character. Instead  of  returning  the  enumeration value 'Operator' for a '+' sign, what's wrong with just returning the character itself?  A character is just as good a variable for encoding the different  token  types, it  can  be  used  in case statements  easily, and it's sure a lot easier  to  type. What could be simpler?

Besides, we've already  had  experience with the idea of encoding keywords as single characters. Our previous programs are already written  that  way, so  using  this approach will  minimize  the changes to what we've already done.

Some of you may feel that this idea of returning  character codes is too mickey-mouse. I must  admit  it gets a little awkward for multi-character operators like '<='. If you choose to stay with the  enumerated  type, fine. For the rest, I'd like to show you how to change what we've done above to support that approach.

First, you can delete the SymType declaration now ... we won't be needing that. And you can change the type of Token to char.

Next, to replace SymType, add the following constant string:

\begin{verbatim}
   const KWcode: string[5] = 'xilee';
\end{verbatim}

(I'll be encoding all idents with the single character `x'.)

Lastly, modify Scan and its relatives as follows:

\begin{verbatim}
{----------------------------------------------}
{ Get an Identifier }

procedure GetName;
begin
   Value := '';
   if not IsAlpha(Look) then Expected('Name');
   while IsAlNum(Look) do begin
     Value := Value + UpCase(Look);
     GetChar;
   end;
   Token := KWcode[Lookup(Addr(KWlist), Value, 4) + 1];
end;


{----------------------------------------------}
{ Get a Number }

procedure GetNum;
begin
   Value := '';
   if not IsDigit(Look) then Expected('Integer');
   while IsDigit(Look) do begin
     Value := Value + Look;
     GetChar;
   end;
   Token := '#';
end;


{----------------------------------------------}
{ Get an Operator }

procedure GetOp;
begin
   Value := '';
   if not IsOp(Look) then Expected('Operator');
   while IsOp(Look) do begin
     Value := Value + Look;
     GetChar;
   end;
   if Length(Value) = 1 then
      Token := Value[1]
   else
      Token := '?';
end;


{----------------------------------------------}
{ Lexical Scanner }

procedure Scan;
var k: integer;
begin
   while Look = CR do
      Fin;
   if IsAlpha(Look) then
      GetName
   else if IsDigit(Look) then
      GetNum
   else if IsOp(Look) then begin
      GetOp
   else begin
      Value := Look;
      Token := '?';
      GetChar;
   end;
   SkipWhite;
end;


{----------------------------------------------}
{ Main Program }


begin
   Init;
   repeat
      Scan;
      case Token of
        'x': write('Ident ');
        '#': Write('Number ');
        'i', 'l', 'e': Write('Keyword ');
        else Write('Operator ');
      end;
      Writeln(Value);
   until Value = 'END';
end.
{----------------------------------------------}
\end{verbatim}

This program should  work  the  same  as the previous version. A minor  difference  in  structure, maybe, but   it   seems  more straightforward to me.

\section{Distributed Vs Centralized Scanners}

The structure for the lexical scanner that I've just shown you is very conventional, and  about  99\% of all compilers use something very  close  to it. This is  not, however, the  only  possible structure, or even always the best one.

The problem with the  conventional  approach  is that the scanner has no knowledge of context. For example, it  can't distinguish between the assignment operator `=' and  the  relational operator `=' (perhaps that's why both C and Pascal  use  different strings for the  two). All  the scanner can do is to pass the operator along  to  the  parser, which can hopefully tell from the context which operator is meant. Similarly, a keyword like 'IF' has no place in the middle of a  math  expression, but if one happens to appear there, the scanner  will  see no problem with it, and will return it to the parser, properly encoded as an 'IF'.

With this  kind  of  approach, we  are  not really using all the information at our disposal. In the middle of an expression, for example, the parser  ``knows''  that  there  is no need to look for keywords, but it has no way of telling the scanner that. So the scanner  continues to do so. This, of  course, slows  down  the compilation.

In real-world compilers, the  designers  often  arrange  for more information  to be passed between parser  and  scanner, just  to avoid  this  kind of problem. But  that  can  get  awkward, and certainly destroys a lot of the modularity of the structure.

The  alternative  is  to seek some  way  to  use  the  contextual information that comes from knowing where we are  in  the parser. This leads us  back  to  the  notion of a distributed scanner, in which various portions  of  the scanner are called depending upon the context.

In KISS, as  in  most  languages, keywords  ONLY  appear  at the beginning of a statement. In places like  expressions, they are not allowed. Also, with one minor exception (the multi-character relops)  that  is  easily  handled, all  operators   are  single characters, which means that we don't need GetOp at all.

So it turns out  that  even  with  multi-character tokens, we can still always tell from the  current  lookahead  character exactly what kind of token is coming, except  at the very beginning of a statement.

Even at that point, the ONLY  kind  of  token we can accept is an identifier. We need only to determine if that  identifier  is  a keyword or the target of an assignment statement.

We end up, then, still needing only {\tt GetName} and {\tt GetNum}, which are used very much as we've used them in earlier installments.

It may seem  at first to you that this is a step backwards, and a rather  primitive  approach. In fact, it is an improvement over the classical scanner, since we're  using  the  scanning routines only where they're really needed. In places  where  keywords are not allowed, we don't slow things down by looking for them.

\section{Merging Scanner And Parser}

Now that we've covered  all  of the theory and general aspects of lexical scanning that we'll be needing, I'm FINALLY ready to back up my claim that  we  can  accommodate multi-character tokens with minimal change to our previous work. To keep  things  short  and simple I will restrict myself here to a subset of what we've done before; I'm allowing only one control construct (the  IF)  and no Boolean expressions. That's enough to demonstrate the parsing of both keywords and expressions. The extension to the full  set of constructs should be  pretty  apparent  from  what  we've already done.

All  the  elements  of  the  program to parse this subset, using single-character tokens, exist  already in our previous programs. I built it by judicious copying of these files, but  I  wouldn't dare try to lead you through that process. Instead, to avoid any confusion, the whole program is shown below:

\begin{verbatim}
{----------------------------------------------}
program KISS;

{----------------------------------------------}
{ Constant Declarations }

const TAB = ^I;
      CR  = ^M;
      LF  = ^J;

{----------------------------------------------}
{ Type Declarations  }

type Symbol = string[8];

     SymTab = array[1..1000] of Symbol;

     TabPtr = ^SymTab;

{----------------------------------------------}
{ Variable Declarations }

var Look  : char;              { Lookahead Character }
    Lcount: integer;           { Label Counter       }


{----------------------------------------------}
{ Read New Character From Input Stream }

procedure GetChar;
begin
   Read(Look);
end;

{----------------------------------------------}
{ Report an Error }

procedure Error(s: string);
begin
   WriteLn;
   WriteLn(^G, 'Error: ', s, '.');
end;

{----------------------------------------------}
{ Report Error and Halt }

procedure Abort(s: string);
begin
   Error(s);
   Halt;
end;


{----------------------------------------------}
{ Report What Was Expected }

procedure Expected(s: string);
begin
   Abort(s + ' Expected');
end;

{----------------------------------------------}
{ Recognize an Alpha Character }

function IsAlpha(c: char): boolean;
begin
   IsAlpha := UpCase(c) in ['A'..'Z'];
end;


{----------------------------------------------}
{ Recognize a Decimal Digit }

function IsDigit(c: char): boolean;
begin
   IsDigit := c in ['0'..'9'];
end;


{----------------------------------------------}
{ Recognize an AlphaNumeric Character }

function IsAlNum(c: char): boolean;
begin
   IsAlNum := IsAlpha(c) or IsDigit(c);
end;

{----------------------------------------------}
{ Recognize an Addop }

function IsAddop(c: char): boolean;
begin
   IsAddop := c in ['+', '-'];
end;


{----------------------------------------------}
{ Recognize a Mulop }

function IsMulop(c: char): boolean;
begin
   IsMulop := c in ['*', '/'];
end;


{----------------------------------------------}
{ Recognize White Space }

function IsWhite(c: char): boolean;
begin
   IsWhite := c in [' ', TAB];
end;


{----------------------------------------------}
{ Skip Over Leading White Space }

procedure SkipWhite;
begin
   while IsWhite(Look) do
      GetChar;
end;


{----------------------------------------------}
{ Match a Specific Input Character }

procedure Match(x: char);
begin
   if Look <> x then Expected('''' + x + '''');
   GetChar;
   SkipWhite;
end;


{----------------------------------------------}
{ Skip a CRLF }

procedure Fin;
begin
   if Look = CR then GetChar;
   if Look = LF then GetChar;
   SkipWhite;
end;


{----------------------------------------------}
{ Get an Identifier }

function GetName: char;
begin
   while Look = CR do
      Fin;
   if not IsAlpha(Look) then Expected('Name');
   Getname := UpCase(Look);
   GetChar;
   SkipWhite;
end;


{----------------------------------------------}
{ Get a Number }

function GetNum: char;
begin
   if not IsDigit(Look) then Expected('Integer');
   GetNum := Look;
   GetChar;
   SkipWhite;
end;


{----------------------------------------------}
{ Generate a Unique Label }

function NewLabel: string;
var S: string;
begin
   Str(LCount, S);
   NewLabel := 'L' + S;
   Inc(LCount);
end;


{----------------------------------------------}
{ Post a Label To Output }

procedure PostLabel(L: string);
begin
   WriteLn(L, ':');
end;


{----------------------------------------------}
{ Output a String with Tab }

procedure Emit(s: string);
begin
   Write(TAB, s);
end;


{----------------------------------------------}

{ Output a String with Tab and CRLF }

procedure EmitLn(s: string);
begin
   Emit(s);
   WriteLn;
end;


{-------------------------------------------------------}
{ Parse and Translate an Identifier }

procedure Ident;
var Name: char;
begin
   Name := GetName;
   if Look = '(' then begin
      Match('(');
      Match(')');
      EmitLn('BSR ' + Name);
      end
   else
      EmitLn('MOVE ' + Name + '(PC),D0');
end;


{-------------------------------------------------------}
{ Parse and Translate a Math Factor }

procedure Expression; Forward;

procedure Factor;
begin
   if Look = '(' then begin
      Match('(');
      Expression;
      Match(')');
      end
   else if IsAlpha(Look) then
      Ident
   else
      EmitLn('MOVE #' + GetNum + ',D0');
end;


{-------------------------------------------------------}
{ Parse and Translate the First Math Factor }


procedure SignedFactor;
var s: boolean;
begin
   s := Look = '-';
   if IsAddop(Look) then begin
      GetChar;
      SkipWhite;
   end;
   Factor;
   if s then
      EmitLn('NEG D0');
end;


{----------------------------------------------}
{ Recognize and Translate a Multiply }

procedure Multiply;
begin
   Match('*');
   Factor;
   EmitLn('MULS (SP)+,D0');
end;


{-----------------------------------------------------}
{ Recognize and Translate a Divide }

procedure Divide;
begin
   Match('/');
   Factor;
   EmitLn('MOVE (SP)+,D1');
   EmitLn('EXS.L D0');
   EmitLn('DIVS D1,D0');
end;


{-------------------------------------------------------}
{ Completion of Term Processing  (called by Term and FirstTerm }

procedure Term1;
begin
   while IsMulop(Look) do begin
      EmitLn('MOVE D0,-(SP)');
      case Look of
       '*': Multiply;
       '/': Divide;
      end;
   end;
end;


{-------------------------------------------------------}
{ Parse and Translate a Math Term }

procedure Term;
begin
   Factor;
   Term1;
end;


{-------------------------------------------------------}
{ Parse and Translate a Math Term with Possible Leading Sign }

procedure FirstTerm;
begin
   SignedFactor;
   Term1;
end;


{-------------------------------------------------------}
{ Recognize and Translate an Add }

procedure Add;
begin
   Match('+');
   Term;
   EmitLn('ADD (SP)+,D0');
end;


{-------------------------------------------------------}
{ Recognize and Translate a Subtract }

procedure Subtract;
begin
   Match('-');
   Term;
   EmitLn('SUB (SP)+,D0');
   EmitLn('NEG D0');
end;


{-------------------------------------------------------}
{ Parse and Translate an Expression }

procedure Expression;
begin
   FirstTerm;
   while IsAddop(Look) do begin
      EmitLn('MOVE D0,-(SP)');
      case Look of
       '+': Add;
       '-': Subtract;
      end;
   end;
end;


{-------------------------------------------------------}
{ Parse and Translate a Boolean Condition }
{ This version is a dummy }

Procedure Condition;
begin
   EmitLn('Condition');
end;


{-------------------------------------------------------}
{ Recognize and Translate an IF Construct }

procedure Block;
 Forward;

procedure DoIf;
var L1, L2: string;
begin
   Match('i');
   Condition;
   L1 := NewLabel;
   L2 := L1;
   EmitLn('BEQ ' + L1);
   Block;
   if Look = 'l' then begin
      Match('l');
      L2 := NewLabel;
      EmitLn('BRA ' + L2);
      PostLabel(L1);
      Block;
   end;
   PostLabel(L2);
   Match('e');
end;


{----------------------------------------------}
{ Parse and Translate an Assignment Statement }

procedure Assignment;
var Name: char;
begin
   Name := GetName;
   Match('=');
   Expression;
   EmitLn('LEA ' + Name + '(PC),A0');
   EmitLn('MOVE D0,(A0)');
end;


{----------------------------------------------}
{ Recognize and Translate a Statement Block }

procedure Block;
begin
   while not(Look in ['e', 'l']) do begin
      case Look of
       'i': DoIf;
       CR: while Look = CR do
              Fin;
       else Assignment;
      end;
   end;
end;


{----------------------------------------------}
{ Parse and Translate a Program }

procedure DoProgram;
begin
   Block;
   if Look <> 'e' then Expected('END');
   EmitLn('END')
end;


{----------------------------------------------}

{ Initialize }

procedure Init;
begin
   LCount := 0;
   GetChar;
end;


{----------------------------------------------}
{ Main Program }

begin
   Init;
   DoProgram;
end.
{----------------------------------------------}
\end{verbatim}

A couple of comments:

\begin{itemize}
\item	The form for the expression parser, using  FirstTerm, etc., is  a  little  different from what you've seen before. It's yet another variation on the same theme. Don't let it throw you: the change is not required for what follows.
\item	Note that, as usual, I had to add calls to Fin  at strategic spots to allow for multiple lines.
\end{itemize}

Before we proceed to adding the scanner, first copy this file and verify that it does indeed  parse things correctly. Don't forget the ``codes'': `i' for IF, `l' for ELSE, and `e' for END or ENDIF.

If the program works, then let's press on. In adding the scanner modules to the program, it helps  to  have a systematic plan. In all  the  parsers  we've  written  to  date, we've  stuck  to  a convention that the current lookahead character should  always be a non-blank character. We  preload  the  lookahead  character in Init, and keep the ``pump primed''  after  that. To keep the thing working right at newlines, we had to modify this a bit  and treat the newline as a legal token.

In the  multi-character version, the rule is similar: The current lookahead character should always be left at the BEGINNING of the next token, or at a newline.

The multi-character version is shown next. To get it, I've made the following changes:

\begin{itemize}
\item	Added the variables Token  and Value, and the type definitions needed by Lookup.
\item	Added the definitions of KWList and KWcode.
\item	Added Lookup.
\item	Replaced {\tt GetName} and {\tt GetNum} by their multi-character versions. (Note that the call  to  Lookup has been moved out of {\tt GetName}, so  that  it  will  not   be  executed  for  calls  within  an expression.)
\item	Created a new, vestigial  Scan that calls {\tt GetName}, then scans for keywords.
\item	Created  a  new  procedure, MatchString, that  looks  for  a specific keyword. Note that, unlike  Match, MatchString does NOT read the next keyword.
\item	Modified Block to call Scan.
\item	Changed the calls  to  Fin  a  bit. Fin is now called within {\tt GetName}.
\end{itemize}

Here is the program in its entirety:

\begin{verbatim}
{----------------------------------------------}
program KISS;

{----------------------------------------------}
{ Constant Declarations }

const TAB = ^I;
      CR  = ^M;
      LF  = ^J;

{----------------------------------------------}
{ Type Declarations  }

type Symbol = string[8];

     SymTab = array[1..1000] of Symbol;

     TabPtr = ^SymTab;


{----------------------------------------------}
{ Variable Declarations }

var Look  : char;              { Lookahead Character }
    Token : char;              { Encoded Token       }
    Value : string[16];        { Unencoded Token     }
    Lcount: integer;           { Label Counter       }


{----------------------------------------------}
{ Definition of Keywords and Token Types }

const KWlist: array [1..4] of Symbol =
              ('IF', 'ELSE', 'ENDIF', 'END');

const KWcode: string[5] = 'xilee';


{----------------------------------------------}
{ Read New Character From Input Stream }

procedure GetChar;
begin
   Read(Look);
end;

{----------------------------------------------}
{ Report an Error }

procedure Error(s: string);
begin
   WriteLn;
   WriteLn(^G, 'Error: ', s, '.');
end;


{----------------------------------------------}
{ Report Error and Halt }

procedure Abort(s: string);
begin
   Error(s);
   Halt;
end;


{----------------------------------------------}
{ Report What Was Expected }

procedure Expected(s: string);
begin
   Abort(s + ' Expected');
end;

{----------------------------------------------}
{ Recognize an Alpha Character }

function IsAlpha(c: char): boolean;
begin
   IsAlpha := UpCase(c) in ['A'..'Z'];
end;


{----------------------------------------------}
{ Recognize a Decimal Digit }

function IsDigit(c: char): boolean;
begin
   IsDigit := c in ['0'..'9'];
end;


{----------------------------------------------}
{ Recognize an AlphaNumeric Character }

function IsAlNum(c: char): boolean;
begin
   IsAlNum := IsAlpha(c) or IsDigit(c);
end;


{----------------------------------------------}
{ Recognize an Addop }

function IsAddop(c: char): boolean;
begin
   IsAddop := c in ['+', '-'];
end;


{----------------------------------------------}
{ Recognize a Mulop }

function IsMulop(c: char): boolean;
begin
   IsMulop := c in ['*', '/'];
end;


{----------------------------------------------}
{ Recognize White Space }

function IsWhite(c: char): boolean;
begin
   IsWhite := c in [' ', TAB];
end;


{----------------------------------------------}
{ Skip Over Leading White Space }

procedure SkipWhite;
begin
   while IsWhite(Look) do
      GetChar;
end;


{----------------------------------------------}
{ Match a Specific Input Character }

procedure Match(x: char);
begin
   if Look <> x then Expected('''' + x + '''');
   GetChar;
   SkipWhite;
end;


{----------------------------------------------}
{ Skip a CRLF }

procedure Fin;
begin
   if Look = CR then GetChar;
   if Look = LF then GetChar;
   SkipWhite;
end;

{----------------------------------------------}
{ Table Lookup }

function Lookup(T: TabPtr; s: string; n: integer): integer;
var i: integer;
    found: boolean;
begin
   found := false;
   i := n;
   while (i > 0) and not found do
      if s = T^[i] then
         found := true
      else
         dec(i);
   Lookup := i;
end;


{----------------------------------------------}
{ Get an Identifier }

procedure GetName;
begin
   while Look = CR do
      Fin;
   if not IsAlpha(Look) then Expected('Name');
   Value := '';
   while IsAlNum(Look) do begin
     Value := Value + UpCase(Look);
     GetChar;
   end;
   SkipWhite;
end;


{----------------------------------------------}
{ Get a Number }

procedure GetNum;
begin
   if not IsDigit(Look) then Expected('Integer');
   Value := '';
   while IsDigit(Look) do begin
     Value := Value + Look;
     GetChar;
   end;
   Token := '#';
   SkipWhite;
end;

{----------------------------------------------}
{ Get an Identifier and Scan it for Keywords }

procedure Scan;
begin
   GetName;
   Token := KWcode[Lookup(Addr(KWlist), Value, 4) + 1];
end;

{----------------------------------------------}
{ Match a Specific Input String }

procedure MatchString(x: string);
begin
   if Value <> x then Expected('''' + x + '''');
end;

{----------------------------------------------}
{ Generate a Unique Label }

function NewLabel: string;
var S: string;
begin
   Str(LCount, S);
   NewLabel := 'L' + S;
   Inc(LCount);
end;

{----------------------------------------------}
{ Post a Label To Output }

procedure PostLabel(L: string);
begin
   WriteLn(L, ':');
end;

{----------------------------------------------}
{ Output a String with Tab }

procedure Emit(s: string);
begin
   Write(TAB, s);
end;

{----------------------------------------------}
{ Output a String with Tab and CRLF }

procedure EmitLn(s: string);
begin
   Emit(s);
   WriteLn;
end;


{-------------------------------------------------------}
{ Parse and Translate an Identifier }

procedure Ident;
begin
   GetName;
   if Look = '(' then begin
      Match('(');
      Match(')');
      EmitLn('BSR ' + Value);
      end
   else
      EmitLn('MOVE ' + Value + '(PC),D0');
end;

{-------------------------------------------------------}
{ Parse and Translate a Math Factor }

procedure Expression; Forward;

procedure Factor;
begin
   if Look = '(' then begin
      Match('(');
      Expression;
      Match(')');
      end
   else if IsAlpha(Look) then
      Ident
   else begin
      GetNum;
      EmitLn('MOVE #' + Value + ',D0');
   end;
end;


{-------------------------------------------------------}
{ Parse and Translate the First Math Factor }

procedure SignedFactor;
var s: boolean;
begin
   s := Look = '-';
   if IsAddop(Look) then begin
      GetChar;
      SkipWhite;
   end;
   Factor;
   if s then
      EmitLn('NEG D0');
end;


{----------------------------------------------}
{ Recognize and Translate a Multiply }

procedure Multiply;
begin
   Match('*');
   Factor;
   EmitLn('MULS (SP)+,D0');
end;


{-----------------------------------------------------}
{ Recognize and Translate a Divide }

procedure Divide;
begin
   Match('/');
   Factor;
   EmitLn('MOVE (SP)+,D1');
   EmitLn('EXS.L D0');
   EmitLn('DIVS D1,D0');
end;


{-------------------------------------------------------}
{ Completion of Term Processing  (called by Term and FirstTerm }

procedure Term1;
begin
   while IsMulop(Look) do begin
      EmitLn('MOVE D0,-(SP)');
      case Look of
       '*': Multiply;
       '/': Divide;
      end;
   end;
end;
{-------------------------------------------------------}
{ Parse and Translate a Math Term }

procedure Term;
begin
   Factor;
   Term1;
end;


{-------------------------------------------------------}
{ Parse and Translate a Math Term with Possible Leading Sign }

procedure FirstTerm;
begin
   SignedFactor;
   Term1;
end;


{-------------------------------------------------------}
{ Recognize and Translate an Add }

procedure Add;
begin
   Match('+');
   Term;
   EmitLn('ADD (SP)+,D0');
end;


{-------------------------------------------------------}
{ Recognize and Translate a Subtract }

procedure Subtract;
begin
   Match('-');
   Term;
   EmitLn('SUB (SP)+,D0');
   EmitLn('NEG D0');
end;


{-------------------------------------------------------}
{ Parse and Translate an Expression }

procedure Expression;
begin
   FirstTerm;
   while IsAddop(Look) do begin
      EmitLn('MOVE D0,-(SP)');
      case Look of
       '+': Add;
       '-': Subtract;
      end;
   end;
end;


{-------------------------------------------------------}
{ Parse and Translate a Boolean Condition }
{ This version is a dummy }

Procedure Condition;
begin
   EmitLn('Condition');
end;


{-------------------------------------------------------}
{ Recognize and Translate an IF Construct }

procedure Block; Forward;


procedure DoIf;
var L1, L2: string;
begin
   Condition;
   L1 := NewLabel;
   L2 := L1;
   EmitLn('BEQ ' + L1);
   Block;
   if Token = 'l' then begin
      L2 := NewLabel;
      EmitLn('BRA ' + L2);
      PostLabel(L1);
      Block;
   end;
   PostLabel(L2);
   MatchString('ENDIF');
end;


{----------------------------------------------}
{ Parse and Translate an Assignment Statement }

procedure Assignment;
var Name: string;
begin
   Name := Value;
   Match('=');
   Expression;
   EmitLn('LEA ' + Name + '(PC),A0');
   EmitLn('MOVE D0,(A0)');
end;


{----------------------------------------------}
{ Recognize and Translate a Statement Block }

procedure Block;
begin
   Scan;
   while not (Token in ['e', 'l']) do begin
      case Token of
       'i': DoIf;
       else Assignment;
      end;
      Scan;
   end;
end;


{----------------------------------------------}

{ Parse and Translate a Program }

procedure DoProgram;
begin
   Block;
   MatchString('END');
   EmitLn('END')
end;


{----------------------------------------------}

{ Initialize }

procedure Init;
begin
   LCount := 0;
   GetChar;
end;


{----------------------------------------------}
{ Main Program }

begin
   Init;
   DoProgram;
end.
{----------------------------------------------}
\end{verbatim}

Compare this program with its  single-character  counterpart. I think you will agree that the differences are minor.

\section{Conclusion}

At this point, you have learned how to parse  and  generate  code for expressions, Boolean  expressions, and  control structures. You have now learned how to develop lexical scanners, and  how to incorporate their elements into a translator. You have still not seen ALL the elements combined into one program, but on the basis of  what  we've  done before you should find it a straightforward matter to extend our earlier programs to include scanners.

We are very  close  to  having  all  the elements that we need to build a real, functional compiler. There are still a  few things missing, notably procedure  calls  and type definitions. We will deal with  those  in  the  next  few  sessions. Before doing so, however, I thought it  would  be fun to turn the translator above into a true compiler. That's what we'll  be  doing  in  the next installment.

Up till now, we've taken  a rather bottom-up approach to parsing, beginning with low-level constructs and working our way  up. In the next installment, I'll  also  be  taking a look from the top down, and  we'll  discuss how the structure of the translator is altered by changes in the language definition.

See you then.
