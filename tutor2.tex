
\chapter{EXPRESSION PARSING}

24th July 1988

\section{Getting Started}

If you've read the introduction document to this series, you will already know what  we're  about. You will also have copied the cradle software  into your Turbo Pascal system, and have compiled it. So you should be ready to go.

The purpose of this article is for us to learn  how  to parse and translate mathematical expressions. What we would like to see as output is a series of assembler-language statements  that perform the desired actions. For purposes of definition, an expression is the right-hand side of an equation, as in

\begin{verbatim}
               x = 2*y + 3/(4*z)
\end{verbatim}

In the early going, I'll be taking things in {\bfseries very}  small steps. That's  so  that  the beginners among you won't get totally lost. There are also  some  very  good  lessons to be learned early on, that will serve us well later. For the more experienced readers: bear with me. We'll get rolling soon enough.

\section{Single Digits}

In keeping with the whole theme of this series (KISS, remember?), let's start with the absolutely most simple case we can think of. That, to me, is an expression consisting of a single digit.

Before starting to code, make sure you have a  baseline  copy  of the  ``cradle'' that I gave last time. We'll be using it again for other experiments. Then add this code:

\begin{verbatim}
{ Parse and Translate a Math Expression }

procedure Expression;
begin
   EmitLn('MOVE #' + GetNum + ',D0')
end;
\end{verbatim}

And add the line  ``Expression'';  to  the main program
so that it reads:


\begin{verbatim}
{-----------------------------------------------------------}
begin
   Init;
   Expression;
end.
\end{verbatim}

Now  run  the  program. Try any single-digit number as input. You should get a single line of assembler-language output. Now try any  other character as input, and you'll  see  that  the  parser properly reports an error.

CONGRATULATIONS! You have just written a working translator!

OK, I grant you that it's pretty limited. But don't brush  it off too  lightly. This little ``compiler'' does, on  a  very  limited scale, exactly  what  any larger compiler does:    it  correctly recognizes legal  statements in the input ``language'' that we have defined for it, and  it  produces  correct, executable assembler code, suitable  for  assembling  into  object  format. Just  as importantly, it correctly  recognizes  statements  that  are {\bfseries not} legal, and gives a  meaningful  error message. Who could ask for more?  As we expand our  parser, we'd better make sure those two characteristics always hold true.

There  are  some  other  features  of  this  tiny  program  worth mentioning. First, you  can  see that we don't separate  code generation from parsing ... as  soon as the parser knows what we want  done, it generates the object code directly. In  a  real compiler, of course, the reads in {\tt GetChar} would be  from  a  disk file, and the writes to another  disk  file, but this way is much easier to deal with while we're experimenting.

Also note that an expression must leave a result somewhere. I've chosen the  68000  register DO. I  could have made some other choices, but this one makes sense.

\section{Binary Expressions}

Now that we have that under our belt, let's  branch  out  a bit. Admittedly, an ``expression'' consisting of only  one  character is not going to meet our needs for long, so let's see what we can do to extend it. Suppose we want to handle expressions of the form:

\begin{verbatim}
                         1+2
     or                  4-3
     or, in general, <term> +/- <term>
\end{verbatim}

(That's a bit of Backus-Naur Form, or BNF.)

To do this we need a procedure that recognizes a term  and leaves its   result   somewhere, and  another   that   recognizes   and distinguishes  between   a  `+'  and  a  `-'  and  generates  the appropriate code. But if Expression is going to leave its result in DO, where should Term leave its result?    Answer:    the same place. We're  going  to  have  to  save the first result of Term somewhere before we get the next one.

OK, basically what we want to  do  is have procedure Term do what Expression was doing before. So just {\bfseries rename} procedure Expression as Term, and enter the following new version of Expression:

\begin{verbatim}
{-----------------------------------------------------------}
{ Parse and Translate an Expression }

procedure Expression;
begin
   Term;
   EmitLn('MOVE D0,D1');
   case Look of
    '+': Add;
    '-': Subtract;
   else Expected('Addop');
   end;
end;
{----------------------------------------------------------}
\end{verbatim}

Next, just above Expression enter these two procedures:

\begin{verbatim}
{----------------------------------------------------------}
{ Recognize and Translate an Add }

procedure Add;
begin
   Match('+');
   Term;
   EmitLn('ADD D1,D0');
end;


{---------------------------------------------------------}
{ Recognize and Translate a Subtract }

procedure Subtract;
begin
   Match('-');
   Term;
   EmitLn('SUB D1,D0');
end;
{---------------------------------------------------------}
\end{verbatim}

When you're finished with that, the order of the routines should be:

\begin{verbatim}
	Term (The OLD Expression)
	Add
	Subtract
	Expression
\end{verbatim}

Now run the program. Try any combination you can think of of two single digits, separated  by  a  `+' or a `-'. You should get a series of four assembler-language instructions out  of  each run. Now  try  some  expressions with deliberate errors in them. Does the parser catch the errors?

Take  a  look  at the object  code  generated. There  are  two observations we can make. First, the code generated is {\bfseries not} what we would write ourselves. The sequence

\begin{verbatim}
        MOVE #n,D0
        MOVE D0,D1
\end{verbatim}

is inefficient. If we were  writing  this code by hand, we would probably just load the data directly to D1.

There is a  message  here:  code  generated by our parser is less efficient  than the code we would write by hand. Get used to it. That's going to be true throughout this series. It's true of all compilers to some extent. Computer scientists have devoted whole lifetimes to the issue of code optimization, and there are indeed things that can be done to improve the quality  of  code  output. Some compilers do quite well, but  there  is a heavy price to pay in complexity, and it's  a losing  battle  anyway ... there will probably never come a time when  a  good  assembler-language pro- grammar can't out-program a compiler. Before  this  session is over, I'll briefly mention some ways that we can do a  little optimization, just  to  show you that we can indeed improve things without too much trouble. But remember, we're here to learn, not to see how tight we can make  the  object  code. For  now, and really throughout  this  series  of  articles, we'll  studiously ignore optimization and  concentrate  on  getting  out  code that works.

Speaking of which: ours {\bfseries doesn't!}  The code is {\bfseries wrong!}  As things are working  now, the subtraction process subtracts D1 (which has the {\bfseries first} argument in it) from D0 (which has the second). That's the wrong way, so we end up with the wrong  sign  for the result. So let's fix up procedure Subtract with a  sign-changer, so that it reads

\begin{verbatim}
{---------------------------------------------------------}
{ Recognize and Translate a Subtract }

procedure Subtract;
begin
   Match('-');
   Term;
   EmitLn('SUB D1,D0');
   EmitLn('NEG D0');
end;
{---------------------------------------------------------}
\end{verbatim}

Now  our  code  is even less efficient, but at least it gives the right answer!  Unfortunately, the  rules that give the meaning of math expressions require that the terms in an expression come out in an inconvenient  order  for  us. Again, this is just one of those facts of life you learn to live with. This  one will come back to haunt us when we get to division.

OK, at this point we have a parser that can recognize the sum or difference of two digits. Earlier, we  could only recognize a single digit. But  real  expressions can have either form (or an infinity of others). For kicks, go back and run the program with the single input line `1'.

Didn't work, did it? And  why  should  it?  We  just finished telling  our  parser  that the only kinds of expressions that are legal are those  with  two  terms. We  must  rewrite procedure Expression to be a lot more broadminded, and this is where things start to take the shape of a real parser.

\section{General Expressions}

In the  {\bfseries real} world, an  expression  can  consist of one or more terms, separated  by  ``addops''  (`+'  or  `-'). In BNF, this is written

\begin{verbatim}
          <expression> ::= <term> [<addop> <term>]*
\end{verbatim}

We  can  accommodate  this definition of an  expression  with  the addition of a simple loop to procedure Expression:

\begin{verbatim}
{-----------------------------------------------------------}
{ Parse and Translate an Expression }

procedure Expression;
begin
   Term;
   while Look in ['+', '-'] do begin
      EmitLn('MOVE D0,D1');
      case Look of
       '+': Add;
       '-': Subtract;
      else Expected('Addop');
      end;
   end;
end;
{----------------------------------------------------------}
\end{verbatim}

{\bfseries now} we're getting somewhere!   This version handles any number of terms, and it only cost us two extra lines of code. As we go on, you'll discover that this is characteristic  of  top-down parsers ... it only takes a few lines of code to accommodate extensions to the  language. That's  what  makes  our  incremental  approach possible. Notice, too, how well the code of procedure Expression matches the BNF definition. That, too, is characteristic of the method. As you get proficient in the approach, you'll  find that you can turn BNF into parser code just about as  fast  as you can type!

OK, compile the new version of our parser, and give it a try. As usual, verify  that  the  ``compiler''   can   handle   any  legal expression, and  will  give a meaningful error  message  for  an illegal one. Neat, eh?  You might note that in our test version, any error message comes  out  sort of buried in whatever code had already been  generated. But remember, that's just because we are using  the  CRT  as  our  ``output  file''  for   this   series  of experiments. In a production version, the two  outputs  would be separated ... one to the output file, and one to the screen.

\section{Using The Stack}

At  this  point  I'm going to  violate  my  rule  that  we  don't introduce any complexity until  it's  absolutely  necessary, long enough to point out a problem with the code we're generating. As things stand now, the parser  uses D0 for the ``primary'' register, and D1 as  a place to store the partial sum. That works fine for now, because  as  long as we deal with only the ``addops'' `+' and `-', any new term can be added in as soon as it is found. But in general that isn't true. Consider, for example, the expression

\begin{verbatim}
             1+(2-(3+(4-5)))
\end{verbatim}

If we put the `1' in D1, where  do  we  put  the  `2'?    Since a general expression can have any degree of complexity, we're going to run out of registers fast!

Fortunately, there's  a  simple  solution. Like  every modern microprocessor, the 68000 has a stack, which is the perfect place to save a variable number of items. So instead of moving the term in D0 to  D1, let's just push it onto the stack. For the benefit of  those unfamiliar with 68000 assembler  language, a  push  is written

\begin{verbatim}
             -(SP)

and a pop, (SP)+ 
\end{verbatim}

So let's change the EmitLn in Expression to read:

\begin{verbatim}
            EmitLn('MOVE D0,-(SP)');
\end{verbatim}

and the two lines in Add and Subtract to

\begin{verbatim}
           EmitLn('ADD (SP)+,D0')

and        EmitLn('SUB (SP)+,D0'),
\end{verbatim}

respectively. Now try the parser again and make sure  we haven't broken it.

Once again, the generated code is less efficient than before, but it's a necessary step, as you'll see.

\section{Multiplication And Division}

Now let's get down to some REALLY serious business. As  you  all know, there  are  other  math   operators   than   ``addops''  ... expressions can also have  multiply  and  divide operations. You also  know  that  there  is  an implied operator  PRECEDENCE, or hierarchy, associated with expressions, so that in  an expression like

\begin{verbatim}
              2 + 3 * 4,
\end{verbatim}

we know that we're supposed to multiply FIRST, then  add. (See why we needed the stack?)

In the early days of compiler technology, people used some rather complex techniques to insure that the  operator  precedence rules were  obeyed. It turns out, though, that  none  of  this  is necessary ... the rules can be accommodated quite  nicely  by our top-down  parsing technique. Up till now, the  only  form  that we've considered for a term is that of a  single  decimal  digit.

More generally, we  can  define  a  term as a PRODUCT of FACTORS; i.e.,

\begin{verbatim}
        <term> ::= <factor>  [ <mulop> <factor ]*
\end{verbatim}

What  is  a factor?  For now, it's what a term used to be  ... a single digit.

Notice the symmetry: a  term  has the same form as an expression. As a matter of fact, we can  add  to  our  parser  with  a little judicious  copying and renaming. But  to  avoid  confusion, the listing below is the complete set of parsing routines. (Note the way we handle the reversal of operands in Divide.)

\begin{verbatim}
{--------------------------------------------------------------}
{ Parse and Translate a Math Factor }

procedure Factor;
begin
   EmitLn('MOVE #' + GetNum + ',D0')
end;

{--------------------------------------------------------------}
{ Recognize and Translate a Multiply }

procedure Multiply;
begin
   Match('*');
   Factor;
   EmitLn('MULS (SP)+,D0');
end;

{-------------------------------------------------------------}
{ Recognize and Translate a Divide }

procedure Divide;
begin
   Match('/');
   Factor;
   EmitLn('MOVE (SP)+,D1');
   EmitLn('DIVS D1,D0');
end;


{-------------------------------------------------------------}
{ Parse and Translate a Math Term }

procedure Term;
begin
   Factor;
   while Look in ['*', '/'] do begin
      EmitLn('MOVE D0,-(SP)');
      case Look of
       '*': Multiply;
       '/': Divide;
      else Expected('Mulop');
      end;
   end;
end;

{------------------------------------------------------------}
{ Recognize and Translate an Add }

procedure Add;
begin
   Match('+');
   Term;
   EmitLn('ADD (SP)+,D0');
end;

{-----------------------------------------------------------}
{ Recognize and Translate a Subtract }

procedure Subtract;
begin
   Match('-');
   Term;
   EmitLn('SUB (SP)+,D0');
   EmitLn('NEG D0');
end;

{-----------------------------------------------------------}
{ Parse and Translate an Expression }

procedure Expression;
begin
   Term;
   while Look in ['+', '-'] do begin
      EmitLn('MOVE D0,-(SP)');
      case Look of
       '+': Add;
       '-': Subtract;
      else Expected('Addop');
      end;
   end;
end;
\end{verbatim}

Hot dog!  A {\bfseries nearly} functional parser/translator, in only 55 lines of Pascal!  The output is starting to look really useful, if you continue to overlook the inefficiency, which  I  hope  you will. Remember, we're not trying to produce tight code here.

\section{Parentheses}

We  can  wrap  up this part of the parser with  the  addition  of parentheses with  math expressions. As you know, parentheses are a  mechanism to force a desired operator  precedence. So, for example, in the expression

\begin{verbatim}
               2*(3+4),
\end{verbatim}

the parentheses force the addition  before  the  multiply. Much more importantly, though, parentheses  give  us  a  mechanism for defining expressions of any degree of complexity, as in

\begin{verbatim}
               (1+2)/((3+4)+(5-6))
\end{verbatim}

The  key  to  incorporating  parentheses  into our parser  is  to realize that  no matter how complicated an expression enclosed by parentheses may be, to  the  rest  of  the world it looks like a simple factor. That is, one of the forms for a factor is:

\begin{verbatim}
          <factor> ::= (<expression>)
\end{verbatim}

This is where the recursion comes in. An expression can contain a factor which contains another expression which contains a factor, etc., ad infinitum.

Complicated or not, we can take care of this by adding just a few lines of Pascal to procedure Factor:

\begin{verbatim}
{-----------------------------------------------------------}
{ Parse and Translate a Math Factor }

procedure Expression; Forward;

procedure Factor;
begin
   if Look = '(' then begin
      Match('(');
      Expression;
      Match(')');
      end
   else
      EmitLn('MOVE #' + GetNum + ',D0');
end;
{----------------------------------------------------------}
\end{verbatim}

Note again how easily we can extend the parser, and how  well the Pascal code matches the BNF syntax.

As usual, compile the new version and make sure that it correctly parses  legal sentences, and flags illegal  ones  with  an  error message.

\section{Unary Minus}

At  this  point, we have a parser that can handle just about any expression, right?  OK, try this input sentence:

\begin{verbatim}
                       -1
\end{verbatim}

WOOPS!  It doesn't work, does it?   Procedure  Expression expects everything to start with an integer, so it coughs up  the leading minus  sign. You'll find that +3 won't  work  either, nor  will something like

\begin{verbatim}
                    -(3-2).
\end{verbatim}

There  are  a  couple of ways to fix the problem. The  easiest (although not necessarily the best)  way is to stick an imaginary leading zero in  front  of  expressions  of this type, so that -3 becomes 0-3. We can easily patch this into our  existing version of Expression:

\begin{verbatim}
{-------------------------------------------------------}
{ Parse and Translate an Expression }

procedure Expression;
begin
   if IsAddop(Look) then
      EmitLn('CLR D0')
   else
      Term;
   while IsAddop(Look) do begin
      EmitLn('MOVE D0,-(SP)');
      case Look of
       '+': Add;
       '-': Subtract;
      else Expected('Addop');
      end;
   end;
end;
{------------------------------------------------------}
\end{verbatim}

I {\bfseries told} you that making changes  was  easy!   This time it cost us only  three  new lines of Pascal. Note  the  new  reference  to function {\tt IsAddop}. Since the test for an {\tt addop} appeared  twice, I chose  to  embed  it in the new function. The  form  of  {\tt IsAddop} should be apparent from that for {\tt IsAlpha}. Here it is:

\begin{verbatim}
{------------------------------------------------------}
{ Recognize an Addop }

function IsAddop(c: char): boolean;
begin
   IsAddop := c in ['+', '-'];
end;
{------------------------------------------------------}
\end{verbatim}

OK, make these changes to the program and recompile. You should also include {\tt IsAddop} in your baseline copy of the cradle. We'll be needing  it  again  later. Now try the input -1 again. Wow!  The efficiency of the code is  pretty  poor ... six lines of code just for loading a simple constant ... but at least it's correct. Remember, we're not trying to replace Turbo Pascal here.

At this point we're just about finished with the structure of our expression parser. This version of the program should correctly parse and compile just about any expression you care to  throw at it. It's still limited in that  we  can  only  handle  factors involving single decimal digits. But I hope that by now you're starting  to  get  the  message  that we can  accommodate  further extensions  with  just  some  minor  changes to the parser. You probably won't be  surprised  to  hear  that a variable or even a function call is just another kind of a factor.

In  the next session, I'll show you just how easy it is to extend our parser to take care of  these  things too, and I'll also show you just  how easily we can accommodate multicharacter numbers and variable names. So you see, we're  not  far at all from a truly useful parser.

\section{A Word About Optimization}

Earlier in this session, I promised to give you some hints  as to how we can improve the quality of the generated code. As I said, the  production of tight code is not the  main  purpose  of  this series of articles. But you need to at least know that we aren't just  wasting our time here ... that we  can  indeed  modify  the parser further to  make  it produce better code, without throwing away everything we've done to date. As usual, it turns  out that SOME optimization is not that difficult to do ... it simply takes some extra code in the parser.

There are two basic approaches we can take:

\subsection{Try to fix up the code after it's generated}

This is  the concept of ``peephole'' optimization. The general idea it that we  know  what  combinations of instructions the compiler  is  going  to generate, and we also know which ones are pretty bad (such as the code for -1, above). So all we do  is  to   scan   the  produced  code, looking  for  those combinations, and replacing  them  by better ones. It's sort of   a   macro   expansion, in   reverse, and   a  fairly straightforward  exercise  in   pattern-matching. The  only complication, really, is that there may be  a {\bfseries lot}  of  such combinations to look for. It's called  peephole optimization simply because it only looks at a small group of instructions at a time. Peephole  optimization can have a dramatic effect on  the  quality  of the code, with  little  change  to  the structure of the compiler  itself. There is a price to pay, though, in  both  the  speed, size, and complexity of  the compiler. Looking for all those combinations calls for a lot of IF tests, each one of which is a source of error. And, of course, it takes time.

In  the  classical  implementation  of a peephole optimizer, it's done as a second pass to the compiler. The  output code is  written  to  disk, and  then  the  optimizer  reads  and processes the disk file again. As a matter of fact, you can see that the optimizer could  even be a separate PROGRAM from the compiler proper. Since the optimizer only  looks  at the code through a  small  ``window''  of  instructions  (hence the name), a better implementation would be to simply buffer up a few lines of output, and scan the buffer after each {\tt EmitLn}.

\subsection{Try to generate better code in the first place}

This approach calls for us to look for  special  cases {\bfseries before} we Emit them. As a trivial example, we  should  be  able to identify a constant zero, and  Emit a {\tt CLR} instead of a load, or even do nothing at all, as in an add of zero, for example. Closer to home, if we had chosen to recognize the unary minus in Factor  instead of in Expression, we could treat constants like -1 as ordinary constants, rather  then  generating them from  positive  ones. None of these things are difficult to deal with ... they only add extra tests in the code, which is why  I  haven't  included them in our program. The way I see it, once we get to the point that we have a working compiler, generating useful code  that  executes, we can always go back and tweak the thing to tighten up the code produced. That's why there are Release 2.0's in the world.

There {\bfseries is} one more type  of  optimization  worth  mentioning, that seems to promise pretty tight code without too much hassle. It's my ``invention'' in the  sense  that I haven't seen it suggested in print anywhere, though I have  no  illusions  that  it's original with me.

This  is to avoid such a heavy use of the stack, by making better use of the CPU registers. Remember back when we were  doing only addition  and  subtraction, that we used registers  D0  and  D1, rather than the stack?  It worked, because with  only  those  two operations, the ``stack'' never needs more than two entries.

Well, the 68000 has eight data registers. Why not use them as a privately managed stack?  The key is to recognize  that, at  any point in its processing, the  parser {\bfseries knows} how many items are on the  stack, so it can indeed manage it properly. We can define a private ``stack pointer'' that keeps  track  of  which  stack level we're at, and addresses the  corresponding  register. Procedure Factor, for  example, would  not  cause data to be loaded  into register  D0, but   into  whatever  the  current  ``top-of-stack'' register happened to be.

What we're doing in effect is to replace the CPU's RAM stack with a  locally  managed  stack  made  up  of  registers. For  most expressions, the stack level  will  never  exceed eight, so we'll get pretty good code out. Of course, we also  have  to deal with those  odd cases where the stack level  DOES  exceed  eight, but that's no problem  either. We  simply let the stack spill over into the CPU  stack. For  levels  beyond eight, the code is no worse  than  what  we're generating now, and for levels less than eight, it's considerably better.

For the record, I  have  implemented  this  concept, just to make sure  it  works  before  I  mentioned  it to you. It does. In practice, it turns out that you can't really use all eight levels ... you need at least one register free to  reverse  the  operand order for division  (sure  wish  the  68000 had an XTHL, like the 8080!). For expressions  that  include  function calls, we would also need a register reserved for them. Still, there  is  a  nice improvement in code size for most expressions.

So, you see, getting  better  code  isn't  that difficult, but it does add complexity to the our translator ... complexity  we can do without at this point. For that reason, I  STRONGLY  suggest that we continue to ignore efficiency issues for the rest of this series, secure  in  the knowledge that we can indeed improve the code quality without throwing away what we've done.

Next lesson, I'll show you how to deal with variables factors and function calls. I'll also show you just how easy it is to handle multicharacter tokens and embedded white space.
